\documentclass[final]{siamltex}

% for red MarginPars
\usepackage{color}

% for \boldsymbol
\usepackage{amsmath}
\usepackage{latexsym}
\usepackage{graphicx}
\usepackage{geometry}
\usepackage{hyperref}

% total number of floats allowed on a page
\setcounter{totalnumber}{100}

% float page fractions
\renewcommand{\topfraction}{0.9}
\renewcommand{\bottomfraction}{0.9}
\renewcommand{\textfraction}{0.2}

% MarginPar
\setlength{\marginparwidth}{0.75in}
\newcommand{\MarginPar}[1]{\marginpar{\vskip-\baselineskip\raggedright\tiny\sffamily\hrule\smallskip{\color{red}#1}\par\smallskip\hrule}}

% for non-stacked fractions
\newcommand{\sfrac}[2]{\mathchoice
  {\kern0em\raise.5ex\hbox{\the\scriptfont0 #1}\kern-.15em/
   \kern-.15em\lower.25ex\hbox{\the\scriptfont0 #2}}
  {\kern0em\raise.5ex\hbox{\the\scriptfont0 #1}\kern-.15em/
   \kern-.15em\lower.25ex\hbox{\the\scriptfont0 #2}}
  {\kern0em\raise.5ex\hbox{\the\scriptscriptfont0 #1}\kern-.2em/
   \kern-.15em\lower.25ex\hbox{\the\scriptscriptfont0 #2}}
  {#1\!/#2}}

\def\1b {{\bf 1}}
\def\Ab {{\bf A}}
\def\bb {{\bf b}}
\def\Eb {{\bf E}}
\def\fb {{\bf f}}
\def\Fb {{\bf F}}
\def\gb {{\bf g}}
\def\Ib {{\bf I}}
\def\Lb {{\bf L}}
\def\mb {{\bf m}}
\def\vb {{\bf v}}
\def\wb {{\bf w}}
\def\Wb {{\bf W}}
\def\xb {{\bf x}}
\def\zb {{\bf z}}

\def\chib   {\boldsymbol{\chi}}
\def\deltab {\boldsymbol{\delta}}
\def\Gammab {\boldsymbol{\Gamma}}
\def\phib   {\boldsymbol{\phi}}
\def\Phib   {\boldsymbol{\Phi}}
\def\Psib   {\boldsymbol{\Psi}}
\def\rhob   {\boldsymbol{\rho}}
\def\sigmab {\boldsymbol{\sigma}}
\def\Sigmab {\boldsymbol{\Sigma}}
\def\taub   {\boldsymbol{\tau}}
\def\zetab  {\boldsymbol{\zeta}}

\def\half   {\frac{1}{2}}
\def\myhalf {\sfrac{1}{2}}

\begin{document}

%==========================================================================
% Title
%==========================================================================
\title{Applications}

\maketitle

\section{Electroosmotic Flow}

Using the code from \cite{LowMachElectro}, the problem setup is as follows.

The computational domain is square with $L=1.28\times 10^{-4}$~cm.
These tests were performed in two-dimensions.  We typically use
$128^2$ grid cells so $\Delta x = 1\times 10^{-6}$~cm.

We model a saltwater solution with 3 species: Na, Cl, and Water.
The viscosity is $1.05\times 10^{-2}$~g/cm$^2$.

The applied electric field in the flow direction is $10^{11}$~V/cm.

Boundary conditions are inhomogeneous Neumann, with a surface charge
density set to exactly balance the lack of charge neutrality on the interior.
In fact, if the Neumann condition is not specified correctly to a large number
of digits, the electrostatic solver will fail to converge.

We have a temporary fix in the code to handle the Lorentz force using the
discretization $qE$.  For the original formulation work we somehow need to
subtract off the charge at the boundary.

\subsection{Test 1: Debye length much smaller than the channel width}
We initialize $w_{\rm Na} = 2.088\times 10^{-6}$ and
$w_{\rm Cl} = 1.68\times 10^{-6}$ so the Debye length of the initial
configuration is $3.66\times 10^{-6}$~cm.  The total charge in the interior
of the domain is $6.8812977493198297\times 10^{-11}$~C.  The surface charge density
is $\sigma = q_{\rm tot} / (2L\cdot (1~{\rm cm})) = 2.68800693333\times 10^{-7}$~g/cm$^2$
(this is 2D but we say the thickness is 1~cm so the units work out)

As the simulation reaches
steady-state, there are four Debye lengths to consider.
\begin{itemize}
\item The initial state ($3.66\times 10^{-6}$~cm).
\item The state against the wall ($1.21\times 10^{-6}$~cm).
\item The state 1 Debye length from the wall ($2.95\times 10^{-6}$~cm).
\item The state at the centerline ($4.16\times 10^{-6}$~cm).
\end{itemize}
The peak velocity (at the centerline) is given by
\begin{equation}
v_0 = \frac{\lambda\sigma E_0}{\mu} ~ {\rm cm/s}.\label{eq:v0}
\end{equation}
Plugging in the parameters (using the smalles and largest of the 4 Debye lengths),
the predicted peak velocity is between 
3.10~cm/s and 10.65~cm/s.  The computationally measured peak velocity 
is $v_0 = 6.54$~cm/s

The theoretical result for the velocity one Debye length from the wall
is $v_0 (1-e^{-1}) = (6.54)(0.63) = 4.13$~cm/s.
This comes from the equation valid for $\lambda \ll L$:
\begin{equation}
v(y) = -v_0(1-e^{-(|y|-L/2)/\lambda})
\end{equation}

The computationally measured velocity one Debye length from the wall is 4.3~cm/s.  

\subsection{Test 2: Debye length comparable to the channel width}
We initialize $w_{\rm Na} = 2.088\times 10^{-9}$ and
$w_{\rm Cl} = 1.68\times 10^{-9}$ so the Debye length is
$1.16\times 10^{-4}$~cm.

The surface charge density
is $\sigma = q_{\rm tot} / (2L\cdot (1~{\rm cm})) = 2.68800000693\times 10^{-10}$~g/cm$^2$

We have zeta potential $\zeta = \lambda\sigma/\epsilon = 44379$.

The equation for velocity in general 
\begin{equation}
v(y) = v_0\left(1 - \frac{\phi_0(y)}{\phi_0(L/2)}\right)\label{eq:vgeneral}
\end{equation}
where we believe $\zeta$ is the potential at the wall, i.e., $\zeta=\phi_0(L/2)$.
Our maximum velocity at the center line in the computation is 0.08~cm/s.
The potential varies from $\zeta$ at the wall to $\zeta+11953$ at the centerline.
We compute $v_0=0.30$~cm/s using (\ref{eq:v0}) above, and thus the velocity at the 
centerline using (\ref{eq:vgeneral}) is $0.08$, matching the computation.

\subsection{Conductance Test}
I instrumented the code to all up the total advective and diffusive mass fluxes on a face
that spans the vertical direction.  As a proxy for measuring current, I subtract
the total flux of the postively charged species from the total flux of the negatively
charged species.

The idea is to vary only the salt concentration and measure the current.  Experiments
show that as the concentration decreases, the conductance does not linearly decrease once
the concentration falls below a certain value, where the conductance reaches
a constant value.  Thus as the concentration becomes smaller, the current should be
proportional to the applied electric field and independent of the concentration.

I have tried tests where I have kept the Na and Cl at a constant ratio (that has
a net charge), and tests where I have kept the amount of Cl on the boundary
fixed and decreased the Na and Cl in a way to maintain overall charge neutrality.

\section{Induced Charge Electrokinetic / Induced Charge Electro-osmosis}
The setup is a neutral salt on the interior.
The lo and hi x-boundary conditions on electric potential
are $\phi=-1$ and $\phi=1$.
On the hi-y wall we use a homogeneous Neumann electric potential.
On the lo-y wall we use a homogeneous Dirichlet electric potential,
but outside the central 50\% of the domain we set $\epsilon$ on the
boundary face to zero, which has the net effect of a homogeneous
Neumann electric potential boundary condition.

\section{Electrophoretic Flow}
The setup is a salt with unequal ions so there is a net interior charge.
The x-direction is periodic with an applied electric field.
On the hi-y wall we homogeneous Neumann electric potential.
On the lo-y wall we use a Neumann electric potential boundary condition, where the
central 50\% of the domain has a net surface charge density to match
the amount of charge on the interior, and the rest of the wall is homogeneous.

\section{Strip/Square Diffusion Testing}

\begin{verbatim}
We model a mixture of NaCl and KCl in water using 3 different approaches.
In each case we have a horizontal strip of higher concentration NaCl,
and a square region of higher concentration KCl centered in the domain.
In this strip and square, the concentration of each salt is double the ambient.

We use periodic boundary conditions.  Since we use a Boussinesq 
approximation, all the rhobars are equal to 1.

Simulation parameters:

Domain Length = 2.56d-3 [cm]

Approach 1.  4-species (Na, Cl, K, H2O), explicit electrodiffusion.
             (inputs\_trinary\_test with electroneutral=F)

Approach 2.  4-species, electroneutral.
             (inputs\_trinary\_test with electroneutral=T)

Approach 3.  3-species (NaCl, KCl, H2O), ambipolar approximation (no charges).
             The initial conditions in Approach 3 have been calibrated to
             match the amount of Na, Cl, and K used in Approach 1 and 2.
             (inputs\_ambipolar\_trinary\_test with optional hacked 
              src\_lowMach/write\_plotfile.f90->write\_plotfile\_ambi.f90
              to see the individual species)

For Approach 1 and 2, the Debye length is 4.42e-6cm (44.2nm).
Thus, the explicit electrodiffusion stability limit is ~9.63d-7s.

All of the binary diffusion coefficients are consistent with each other across
the 3 approaches to 15 digits of accuracy.

  -The "free" parameters are binary Na-water, K-water, Cl-water, (found in the 4-species input file)
   and self-water (self-water does not appear in inputs file but is 2.3d-5 as reported in electrolye paper)
  -The Na-Cl, K-Cl, Na-K binary coefficients are computed from eq (91) in the electrolyte paper
  -The NaCl-water and KCl-water binary coefficients are computed from eq (63) in the electrolyte paper
  -Then the NaCl-KCl binary coefficient is computed from eq (91)

The mass diffusion stability limit is a function of dx

n\_cell       dx       dt\_diff
32           8.e-5    7.88e-5
64           4.e-5    1.97e-5
128          2.e-5    4.93e-6
256          1.e-5    1.23e-6
512 (ref)    5.e-6    3.08e-7

Setup: (tfinal = 6.4d-4)

We coarsen both space and time by factors of 2, except for the reference solution which has
a very small time step so it can overcome the mass diffusion stability limit.

n\_cell       dt       max\_step   massdiff\_cfl (1 is stability limit)  electrodiff\_cfl
32           6.4d-6   100        0.08                                 6.65
64           3.2d-6   200        0.16                                 3.32
128          1.6d-6   400        0.32                                 1.66
256          8.0d-7   800        0.65                                 0.83
512 (ref)    2.0d-7   3200       0.65                                 0.21

We run Approaches 1 and 2 with 256 cells to show they give very similar results.
Next, we show that Approach 3 with 256 cells gives results that do not match
Approach 1 nearly as well as Approach 2 does:

(EN to EE) L1 norm of Absolute and Relative Error in Each Component
--------------------------------------------------------------------------------
Level:  0
                                c1:      3.634660615e-17     9.568324806e-11
                                c2:      1.413736678e-16     1.334739775e-10
                                c3:      6.689817041e-17     1.295370978e-10
                                c4:      3.854876397e-17     3.854884239e-17

(AMBI to EE) L1 norm of Absolute and Relative Error in Each Component
--------------------------------------------------------------------------------
Level:  0
                                c1:      1.195011321e-15     3.136804244e-09
                                c2:      5.234618438e-16     4.912427224e-10
                                c3:      2.331950865e-15     4.496367586e-09
                                c4:      1.660083259e-15     1.660086688e-15

We also demonstrate that we cannot run Approach 1 using any coarse resolution
since we violate the electrodiffusion stability limit.  However we can run
Approach 2 coarser, demonstrating the code gives convergent results with 32 cells:

-----
Error (top) and Convergence Rates (bottom) for EN algorithm

c1		c2		c3		c4
3.63E-015	8.71E-015	3.82E-015	1.58E-014  (32-64)
9.36E-016	2.26E-015	1.01E-015	4.11E-015  (64-128)
2.34E-016	5.63E-016	2.52E-016	1.02E-015  (128-256)
-----
1.9541749602	1.9484281296	1.9146316615	1.9464200824
1.9973750393	2.0013118237	2.0047483301	2.002534713
-----

For completeness, here is the AMBI convergence rates (to obtain convergence 
rates for the EE algorithm we need to shrink the time step; in my earlier 
testing it was second-order, although I do not report the numbers here).

-----
Error (top) and Convergence Rates (bottom) for AMBI algorithm

c1		c2		c3		c4
3.64E-015	8.70E-015	3.77E-015	1.58E-014
9.38E-016	2.26E-015	9.98E-016	4.11E-015
2.35E-016	5.63E-016	2.48E-016	1.02E-015
-----			
1.9575453343	1.9480125666	1.9198041446	1.9455387386
1.9977691699	2.0013748132	2.0073818783	2.0024779391
-----

REFERENCE SOLUTION
To obtain a reference solution, we use Approach 1 using a very small dx and dt.
We use 1024 cells, and a time step of 2.d-8.  Both the explicit electrodiffusion
and electroneutral algorithm at 256 cells seem to give comparable answers with
respect to the reference solution, which is expected since I already stated above
that EE and EN give essentially the same answer.

Next we compare the EN solutions at all resolutions to the reference solution, and then
we compare the AMBI solutions at all resolutions to the reference solution.
We can see that the EN solution is converging toward the reference solution whereas the
AMBI solution asymptotically approaches a different solution.

L1 Errors of EN algorithm compared to reference solution

c1		c2		c3		c4
4.64E-009	1.85E-008	1.23E-008	3.54E-008
1.16E-009	4.47E-009	2.94E-009	8.40E-009
2.82E-010	1.26E-009	7.32E-010	2.24E-009
5.45E-011	3.80E-010	1.38E-010	5.58E-010
			
Convergence Rate			
2.0043502586	2.0469087378	2.0684559857	2.0766326693
2.0363042065	1.8330929776	2.0035237126	1.9077447356
2.3695077669	1.7255142777	2.4126391262	2.0032054748
-----

L1 Errors of AMBI algorithm compared to reference solution

c1		c2		c3		c4
3.89E-009	1.87E-008	1.38E-008	3.64E-008
5.45E-010	4.67E-009	4.53E-009	9.32E-009
6.70E-010	1.48E-009	2.40E-009	3.32E-009
8.82E-010	6.18E-010	1.93E-009	1.65E-009
			
Convergence Rate		
2.8362413505	2.0029799987	1.6097501077	1.9663045457
-0.2979170902	1.6536953736	0.917136087	1.4900618972
-0.3971062067	1.2639758383	0.3133001509	1.0117594677
-----

Follow up: I changed the diffusion coefficients so D(Cl) = 2*D(K) = 4*D(Na) = 2.d-5

(EN to EE) L1 norm of Absolute and Relative Error in Each Component
L1 norm of Absolute and Relative Error in Each Component
--------------------------------------------------------------------------------
Level:  0
                                c1:      7.634765955e-17     1.967261293e-10
                                c2:      6.945209422e-16       6.3844666e-10
                                c3:       1.92814393e-16      3.58636414e-10
                                c4:       4.27874198e-16     4.278751025e-16

(AMBI to EE) L1 norm of Absolute and Relative Error in Each Component
L1 norm of Absolute and Relative Error in Each Component
--------------------------------------------------------------------------------
Level:  0
                                c1:      1.422469878e-15     3.586513422e-09
                                c2:      1.838189849e-15     1.714736948e-09
                                c3:      3.571952675e-15     6.974175322e-09
                                c4:       3.96370133e-15     3.963709584e-15


Now instead of the L1 norm, let's consider the L0 norm (which may actually be
the better way to look at the differences - consider the maximum difference)

Using the 'fake' diffusion coefficients, at early time

(EN to EE) L0 norm of Absolute and Relative Error in Each Component
L0 norm of Absolute and Relative Error in Each Component
--------------------------------------------------------------------------------
Level:  0
                                c1:      8.814952934e-11     0.0002280886335
                                c2:       7.18089591e-10      0.000636396231
                                c3:      2.345145084e-10     0.0003971919597
                                c4:      4.160143341e-10     4.160154816e-10


(AMBI to EE) L0 norm of Absolute and Relative Error in Each Component
L0 norm of Absolute and Relative Error in Each Component
--------------------------------------------------------------------------------
Level:  0
                                c1:      1.705761866e-09        0.0037975753
                                c2:      1.741987167e-09      0.001584005311
                                c3:       4.31884101e-09      0.008495029476
                                c4:      4.375728335e-09     4.375738142e-09


At later time (5x later)

(EN to EE) L0 norm of Absolute and Relative Error in Each Component
--------------------------------------------------------------------------------
Level:  0
                                c1:      6.346989821e-11     0.0001770828472
                                c2:       3.22143482e-10     0.0003017736847
                                c3:      1.354736963e-10     0.0002386451304
                                c4:      1.727238352e-10     1.727243426e-10


(AMBI to EE) L0 norm of Absolute and Relative Error in Each Component
--------------------------------------------------------------------------------
Level:  0
                                c1:      4.737941285e-09       0.01086536039
                                c2:      3.394453866e-09      0.002755756522
                                c3:      1.126567092e-08       0.02265134035
                                c4:      9.967068615e-09     9.967091038e-09

With the 'realisitc' diffusion coefficients,

(EN to EE) L0 norm of Absolute and Relative Error in Each Component

--------------------------------------------------------------------------------
Level:  0
                                c1:      3.192426265e-11     0.0001006612942
                                c2:      1.094870349e-10      0.000121569243
                                c3:      5.308008371e-11     0.0001073714369
                                c4:       2.92439406e-11     2.924398951e-11


(AMBI to EE) L0 norm of Absolute and Relative Error in Each Component
--------------------------------------------------------------------------------
Level:  0
                                c1:      9.481640704e-10      0.002990511672
                                c2:      3.754793045e-10     0.0003999217323
                                c3:      1.825540537e-09      0.003540720257
                                c4:       1.24642674e-09     1.246429031e-09

At longer time:

(EN to EE) L0 norm of Absolute and Relative Error in Each Component
--------------------------------------------------------------------------------
Level:  0
                                c1:      1.572249737e-11     5.018801799e-05
                                c2:      4.707963446e-11     5.254320089e-05
                                c3:      2.329893947e-11     4.689002448e-05
                                c4:      1.064148769e-11     1.064151096e-11


(AMBI to EE) L0 norm of Absolute and Relative Error in Each Component
--------------------------------------------------------------------------------
Level:  0
                                c1:      2.039734145e-09      0.006525267577
                                c2:      5.435538885e-10     0.0005937740117
                                c3:      3.920152971e-09       0.00759369096
                                c4:       2.40708431e-09     2.407088624e-09
\end{verbatim}

\section{Electroconvection Boundary Conditions}

We model 3 species, $(w_+, w_-, w_0)$ (think Na$^+$, Cl$^-$, and water).
The mass fractions are something like $(10^{-4}, 10^{-4}, \sim 1)$.\\ \\

To simplify things (for now), we set each $\bar\rho=0$, so $\rho=1$.
We set the charges per mass for $w_+$ and $w_-$ to be equal and
opposite, and use the same molecular mass for all species.\\ \\

The x-walls are periodic.\\ \\

The upper y-wall is a no-slip, no-flow reservoir with fixed concentrations
held at fixed electric potential.\\ \\

The lower y-wall is the complicated case.  This wall 
has no-slip, no-flow boundary conditions, but has a Dirichlet condition
on $w_+$ and Dirichlet electric potential, and does not allow $w_-$ to
pass through at all.
To determine the boundary conditions for $w_-$ and $w_0$,
we need to invoke physical conditions.  Ignoring fluctuations for now,
the total flux is
\begin{eqnarray}
\bf{J} &=& \rho W \chi \left( \Gamma \nabla x + \underbrace{\frac{\bar{m} W z}{k_B T}}_{a}\nabla \Phi \right) \nonumber\\
&=& \rho W \chi {\bf d}\label{eq:Jflux}
\end{eqnarray}
What we essentially do is multiply out (\ref{eq:Jflux})
\begin{equation}
\left(\begin{array}{c}
J_+ \\
J_- \\
J_0
\end{array}\right)
=
\left(\begin{array}{ccc}
w_+(\chi_{++}d_+ + \chi_{+-}d_- + \chi_{+0}d_0) \\
w_-(\chi_{-+}d_+ + \chi_{--}d_- + \chi_{-0}d_0) \\
w_0(\chi_{0+}d_+ + \chi_{0-}d_- + \chi_{00}d_0)
\end{array}\right)\label{eq:Jmult}
\end{equation}
We observe that we can obtain $d_+$ if we extrapolate the transport coefficients from
the interior to the boundary and use a simple gradient stencil for $\nabla x$.
The boundary does not allow $w_-$ to pass through at all, so we can invoke $J_-=0$.
We use $J_-=0$ along with the condition
\begin{equation}
d_+ + d_- + d_0 = \sum a_i \nabla\phi
\end{equation}
to form a linear system for $d_-$ and $d_0$.
Once we know $d_-$ and $d_0$ we can compute $J_-$ and $J_0$ using (\ref{eq:Jmult}).

\section{Micropump}
We model 5 species, H$^+$, OH$^-$, H$_2$O$_2$, and H$_2$O.

The binary diffusion coefficients into water (in cgs units [cm$^2$/s]) are given by 
$D_{15}=9.3\times 10^{-5}, D_{25}=5.3\times 10^{-5}, D_{35}=6.6\times 10^{-6}, 
D_{45}=2.0\times 10^{-5}$ \cite{Esplandiu16,MoranPosner11}.  The self-diffusion
coefficient for water is $D_5=2.3\times 10^{-5}$ \cite{LowMachMulti}.
We can evaluate the remaining binary diffusion coefficients by using the
infinite dilution approxmation in \cite{LowMachMulti},
\begin{equation}
D_{jk} = \frac{D_{ij}D_{ik}}{D_i}
\end{equation}
giving
$D_{12}=2.143\times 10^{-4},
D_{13}=2.669\times 10^{-4}, D_{23}=1.521\times 10^{-4}
D_{14}=8.09\times 10^{-5}, D_{24}=4.61\times 10^{-5}, D_{34}=5.74\times 10^{-5}$.

We assume the density of each species as well as the the solution is 1~g/cm$^3$,
even though pure hydrogen peroxide is 1.45~g/cm$^3$.  For the initial conditions,
we assume a 0.3 molar (moles/liter) solution of H$_2$O$_2$, which is approximately
a mass fraction of 0.01.  We assume neutral water, so that OH$^-$ and H$^+$ have
a molarity of $10^{-7}$, corresponding to mass fractions of $1.7\times 10^{-9}$
and $10^{-10}$.  The initial O$_2$ molarity is $2\times 10^{-4}$, corresponding
to a mass fraction of $6.4\times 10^{-6}$.  The remaining solution is H$_2$O.

The chemical rates are found in \cite{MoranPosner11} as $k_{\rm anode}=5.5\times 10^{-9}$~m/s
and $k_{\rm cathode}=1$~m$^7$s$^{-1}$/mol$^2$.  Converting to our CGS formulation where
reactions are computed using number densities, and expressed as a change over a volume
rather than a flux over a surface (encapsulated by a final division by $\Delta x$) gives
$\tilde k_{\rm anode} = k_{\rm anode} 100 / \Delta x$ and
$\tilde k_{\rm cathode} = k_{\rm cathode} 100^7 / (N_A^2 \Delta x)$.

\bibliographystyle{plain}
\bibliography{Applications}

\end{document}
