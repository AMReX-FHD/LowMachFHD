%% LyX 2.0.5.1 created this file.  For more info, see http://www.lyx.org/.
%% Do not edit unless you really know what you are doing.
\documentclass[english,11pt]{article}
\usepackage[T1]{fontenc}
\usepackage[latin9]{inputenc}
\usepackage{geometry}
\geometry{verbose,tmargin=1in,bmargin=1in,lmargin=1in,rmargin=1in}
\usepackage{xcolor}
\usepackage{pdfcolmk}
\usepackage{babel}
\usepackage{amstext}
\PassOptionsToPackage{normalem}{ulem}
\usepackage{ulem}
\usepackage[unicode=true,pdfusetitle,
 bookmarks=true,bookmarksnumbered=false,bookmarksopen=false,
 breaklinks=false,pdfborder={0 0 1},backref=false,colorlinks=false]
 {hyperref}

\makeatletter

%%%%%%%%%%%%%%%%%%%%%%%%%%%%%% LyX specific LaTeX commands.
\providecolor{lyxadded}{rgb}{1,0,0}
\providecolor{lyxdeleted}{rgb}{0,0,1}
%% Change tracking with ulem
\newcommand{\lyxadded}[3]{{\texorpdfstring{\color{lyxadded}{}}{}#3}}
\newcommand{\lyxdeleted}[3]{{\texorpdfstring{\color{lyxdeleted}\sout{#3}}{}}}

%%%%%%%%%%%%%%%%%%%%%%%%%%%%%% User specified LaTeX commands.
\usepackage{ae,aecompl}

%\usepackage[normalmargins,normalbib,normaltitle,normalsections,normalindent]{savetrees}
%\usepackage[normalbib]{savetrees}

\providecommand{\onlinecite}{\cite}
\providecommand{\onlineref}{\ref}
\providecommand{\FToth}{Fejes T\'oth}

\makeatother

\begin{document}

\title{Mixed-Mode Instability in a Ternary Mixture}

\maketitle
\global\long\def\V#1{\boldsymbol{#1}}
\global\long\def\M#1{\boldsymbol{#1}}
\global\long\def\Set#1{\mathbb{#1}}


\global\long\def\D#1{\Delta#1}
\global\long\def\d#1{\delta#1}


\global\long\def\norm#1{\left\Vert #1\right\Vert }
\global\long\def\abs#1{\left|#1\right|}


\global\long\def\grad{\M{\nabla}}
\global\long\def\avv#1{\langle#1\rangle}
\global\long\def\av#1{\left\langle #1\right\rangle }


\global\long\def\ki{k}
\global\long\def\wi{\omega}


This is a summary of the experimental setup from \cite{MixedDiffusiveInstability}
for getting a nice diffusion-driven gravitational instability in a
simple ternary mixture: a solution of sugar on top of a solution of
salt (in the beginning the density should be higher on the bottom
so it starts off stable). I suggest our target to be to reproduce
some aspects (at least the pretty picture) in Fig. 4 of \cite{MixedDiffusiveInstability}
-- this is the mixed-mode instability. This is when one starts with
an unstable (heavy on top of light fluid) situation but the differential
diffusion effects also affect the instability. One can also consider
the simpler DLC instability (we did this in the PRE for compressible
flow) or the classical salt-fingering DD instability, but the mixed-mode
one seems most interesting and complex.

We consider a ternary mixture of salt (KCl, species 1, molar mass
$74.55\, g\text{�}mol^{-1}$), sugar (sucrose, species 2, molar mass
$342.3\, g\text{�}mol^{-1}$) and water (species 3, molar mass $18.02\, g\text{�}mol^{-1}$).
The initial configuration is salt solution on top of sugar solution.

Our equation of state is defined as

\begin{equation}
\sum_{i}\frac{\rho_{i}}{\bar{\rho}_{i}}=\rho\sum_{i}\frac{w_{i}}{\bar{\rho}_{i}}=1,\label{eq:EOS}
\end{equation}
where $w_{i}=\rho_{i}/\rho$ is the mass fraction. We can write this
in the form
\begin{equation}
\frac{\rho-\rho_{1}-\rho_{2}}{\bar{\rho}_{3}}+\frac{\rho_{1}}{\bar{\rho}_{1}}+\frac{\rho_{2}}{\bar{\rho}_{2}}=1.\label{eq:eos_1}
\end{equation}
In the paper, the approximate formula for the density is
\begin{equation}
\rho=\rho_{0}\left(1+\alpha_{1}Z_{1}+\alpha_{2}Z_{2}\right)=\rho_{0}\left(1+\frac{\alpha_{1}}{M_{1}}\rho_{1}+\frac{\alpha_{2}}{M_{2}}\rho_{2}\right),\label{eq:eos_2}
\end{equation}
where $\rho_{0}=\bar{\rho}_{3}=10^{3}\, g/l$ (I use liter here since
they use it in the paper, we can use $cm^{3}$ in the actual runs)
is the density of water, $\alpha_{k}$ are constants and $Z_{k}$
is the number of moles of each component, related to the partial density
via $\rho_{k}=Z_{k}M_{k},$ where $M_{k}$ is the molar mass. By comparing
(\ref{eq:eos_1}) and (\ref{eq:eos_2}) we get
\[
1-\frac{\rho_{0}}{\bar{\rho}_{k}}=\rho_{0}\frac{\alpha_{k}}{M_{k}},
\]
which now gives us the value of 
\[
\bar{\rho}_{1}=2.15\text{ and }\bar{\rho}_{2}=1.55
\]
from the values of $\alpha$ tabulated in table I of the paper, $\alpha=4\cdot10^{-2}\, l/mol$
for KCl and $\alpha=12.2\cdot10^{-2}\, l/mol$ for sucrose. BTW, when
used for glycerol ($\alpha=2.3$ from Table I and molar mass $92.2$),
this gives $\bar{\rho}=1.33$, which is close to the correct value
of 1.26.

For the diffusion coefficients, these mixtures are at what can be
considered ``infinite dilution'' so there is essentially no coupling
between the different components (from this pespective it is a bit
boring). In this limit the approximation proposed in \cite{Diffusion_InfiniteDilution}
suggest setting the Maxwell-Stefan diffusion coefficients to
\[
D_{13}=D_{1},\, D_{23}=D_{2},\, D_{12}=\frac{D_{1}D_{2}}{D_{3}},
\]
where from table I we read the self-diffusion coefficient of low-dilution
KCl in water as $D_{1}=1.91\cdot10^{-5}\, cm^{2}/s$, and for sucrose
$D_{2}=0.52\cdot10^{-5}\, cm^{2}/s$. Here $D_{3}$ is the self-diffusion
coefficient of pure water, $D_{3}=2.3\cdot10^{-5}\, cm^{2}/s$.

The initial concentrations on the top and bottom are determined from
the dimensionless number
\[
R=\frac{\alpha_{2}Z_{2}}{\alpha_{1}Z_{1}}=0.89.
\]
The actual values of the concentrations are not stated but can be
calculated as follows. They start from stock solutions of $10^{-2}\, mol/l$
and then dilute one of the solutions to reach the desired $R$. In
our case this gives
\[
Z_{2}^{0}=2.9\cdot10^{-3}\text{ and }Z_{1}^{0}=10^{-2}\, mol/l,
\]
which gives initial densities 
\begin{eqnarray}
\rho_{\text{top}} & = & \rho_{0}\left(1+\alpha_{1}Z_{1}\right)=\left(1.0+4\cdot10^{-2}\times10^{-2}\right)\, g/cm^{3}=\left(1.0+4\cdot10^{-4}\right)\, g/cm^{3},\label{eq:ics}\\
\rho_{\text{bottom}} & = & \rho_{0}\left(1+\alpha_{2}Z_{2}\right)=\left(1.0+12.2\cdot10^{-2}\times2.9\cdot10^{-3}\right)\, g/cm^{3}=\left(1+3.54\cdot10^{-4}\right)\, g/cm^{3}.\nonumber 
\end{eqnarray}


The plots in the bottom panel of Fig. 1 are for times $40,50,60\, s,$
so we hope to be able to run the simulation up to $100\, s$. The
geometry is a cell that it should be at least $1cm^{2}$ in the $x-y$
plane (gravity along $y$ here), since this is the size of the snapshots
in Fig. 1. The thickness in the $z$ direction is $0.25cm$ (so not
so thin). Based on these length and timescales I estimate that we
can use the inertial code (but the overdamped may good a give approximation
also) -- the momentum diffusion time across $1cm$ length is $\tau_{\nu}\sim1cm^{2}/2\cdot10^{-2}\left(cm^{2}/s\right)\sim50s$.
We can use periodic BCs along $x$, and reservoir along $y$ (with
reservoir values set equal to the initial values), and no-slip along
$z$. They use some special experimental procedure to get a very flat
interface at the beginning, not sure however how close to a jump profile
it is (with bds advection we can do a jump also). They state the wavelength
of the Y-shaped instability fingers they get is on the order of $1.3mm$,
which is an important number to observe. To start off the instability
in deterministic simulations we can make the concentrations random
in the middle layer of cells like we did for the Kevin-Helmholtz instability.

\bibliographystyle{unsrt}
\bibliography{References,GarciaGeneralBibFile,MScaleProp}

\end{document}
