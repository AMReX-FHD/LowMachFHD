\documentclass[final]{siamltex}

% for red MarginPars
\usepackage{color}

% for \boldsymbol
\usepackage{amsmath}
\usepackage{latexsym}
\usepackage{graphicx}
\usepackage{geometry}
\usepackage{hyperref}

% total number of floats allowed on a page
\setcounter{totalnumber}{100}

% float page fractions
\renewcommand{\topfraction}{0.9}
\renewcommand{\bottomfraction}{0.9}
\renewcommand{\textfraction}{0.2}

% MarginPar
\setlength{\marginparwidth}{0.75in}
\newcommand{\MarginPar}[1]{\marginpar{\vskip-\baselineskip\raggedright\tiny\sffamily\hrule\smallskip{\color{red}#1}\par\smallskip\hrule}}

% for non-stacked fractions
\newcommand{\sfrac}[2]{\mathchoice
  {\kern0em\raise.5ex\hbox{\the\scriptfont0 #1}\kern-.15em/
   \kern-.15em\lower.25ex\hbox{\the\scriptfont0 #2}}
  {\kern0em\raise.5ex\hbox{\the\scriptfont0 #1}\kern-.15em/
   \kern-.15em\lower.25ex\hbox{\the\scriptfont0 #2}}
  {\kern0em\raise.5ex\hbox{\the\scriptscriptfont0 #1}\kern-.2em/
   \kern-.15em\lower.25ex\hbox{\the\scriptscriptfont0 #2}}
  {#1\!/#2}}

\def\1b {{\bf 1}}
\def\Ab {{\bf A}}
\def\bb {{\bf b}}
\def\Eb {{\bf E}}
\def\fb {{\bf f}}
\def\Fb {{\bf F}}
\def\gb {{\bf g}}
\def\Ib {{\bf I}}
\def\Lb {{\bf L}}
\def\mb {{\bf m}}
\def\vb {{\bf v}}
\def\wb {{\bf w}}
\def\Wb {{\bf W}}
\def\xb {{\bf x}}
\def\zb {{\bf z}}

\def\chib   {\boldsymbol{\chi}}
\def\deltab {\boldsymbol{\delta}}
\def\Gammab {\boldsymbol{\Gamma}}
\def\phib   {\boldsymbol{\phi}}
\def\Phib   {\boldsymbol{\Phi}}
\def\Psib   {\boldsymbol{\Psi}}
\def\rhob   {\boldsymbol{\rho}}
\def\sigmab {\boldsymbol{\sigma}}
\def\Sigmab {\boldsymbol{\Sigma}}
\def\taub   {\boldsymbol{\tau}}
\def\zetab  {\boldsymbol{\zeta}}

\def\half   {\frac{1}{2}}
\def\myhalf {\sfrac{1}{2}}

\begin{document}

%==========================================================================
% Title
%==========================================================================
\title{Electroosmosis Notes}

\maketitle

\section{Problem Setup}
Using the code from \cite{LowMachElectro}, the problem setup is as follows.

The computational domain is square with $L=1.28\times 10^{-4}$~cm.
These tests were performed in two-dimensions.  We typically use
$128^2$ grid cells so $\Delta x = 1\times 10^{-6}$~cm.

We model a saltwater solution with 3 species: Na, Cl, and Water.
The viscosity is $1.05\times 10^{-2}$~g/cm$^2$.

The applied electric field in the flow direction is $10^{11}$~V/cm.

Boundary conditions are inhomogeneous Neumann, with a surface charge
density set to exactly balance the lack of charge neutrality on the interior.
In fact, if the Neumann condition is not specified correctly to a large number
of digits, the electrostatic solver will fail to converge.

We have a temporary fix in the code to handle the Lorentz force using the
discretization $qE$.  For the original formulation work we somehow need to
subtract off the charge at the boundary.

\subsection{Test 1: Debye length much smaller than the channel width}
We initialize $w_{\rm Na} = 2.088\times 10^{-6}$ and
$w_{\rm Cl} = 1.68\times 10^{-6}$ so the Debye length of the initial
configuration is $3.66\times 10^{-6}$~cm.  The total charge in the interior
of the domain is $6.8812977493198297\times 10^{-11}$~C.  The surface charge density
is $\sigma = q_{\rm tot} / (2L\cdot (1~{\rm cm})) = 2.68800693333\times 10^{-7}$~g/cm$^2$
(this is 2D but we say the thickness is 1~cm so the units work out)

As the simulation reaches
steady-state, there are four Debye lengths to consider.
\begin{itemize}
\item The initial state ($3.66\times 10^{-6}$~cm).
\item The state against the wall ($1.21\times 10^{-6}$~cm).
\item The state 1 Debye length from the wall ($2.95\times 10^{-6}$~cm).
\item The state at the centerline ($4.16\times 10^{-6}$~cm).
\end{itemize}
The peak velocity (at the centerline) is given by
\begin{equation}
v_0 = \frac{\lambda\sigma E_0}{\mu} ~ {\rm cm/s}.
\end{equation}
Plugging in the parameters (using the smalles and largest of the 4 Debye lengths),
the predicted peak velocity is between 
3.10~cm/s and 10.65~cm/s.  The computationally measured peak velocity 
is $v_0 = 6.54$~cm/s

The theoretical result for the velocity one Debye length from the wall
is $v_0 (1-e^{-1}) = (6.54)(0.63) = 4.13$~cm/s.
The computationally measured velocity one Debye length from the wall is 4.3~cm/s.  



\subsection{Test 2: Debye length comparable to the channel width}
We initialize $w_{\rm Na} = 2.088\times 10^{-9}$ and
$w_{\rm Cl} = 1.68\times 10^{-9}$ so the Debye length is
$1.16\times 10^{-4}$~cm.

\bibliographystyle{plain}
\bibliography{Electroosmosis}

\end{document}
