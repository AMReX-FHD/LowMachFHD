\documentclass[final]{siamltex}

% for red MarginPars
\usepackage{color}

% for \boldsymbol
\usepackage{amsmath}
\usepackage{latexsym}
\usepackage{graphicx}
\usepackage{geometry}
\usepackage{hyperref}

% total number of floats allowed on a page
\setcounter{totalnumber}{100}

% float page fractions
\renewcommand{\topfraction}{0.9}
\renewcommand{\bottomfraction}{0.9}
\renewcommand{\textfraction}{0.2}

% MarginPar
\setlength{\marginparwidth}{0.75in}
\newcommand{\MarginPar}[1]{\marginpar{\vskip-\baselineskip\raggedright\tiny\sffamily\hrule\smallskip{\color{red}#1}\par\smallskip\hrule}}

% for non-stacked fractions
\newcommand{\sfrac}[2]{\mathchoice
  {\kern0em\raise.5ex\hbox{\the\scriptfont0 #1}\kern-.15em/
   \kern-.15em\lower.25ex\hbox{\the\scriptfont0 #2}}
  {\kern0em\raise.5ex\hbox{\the\scriptfont0 #1}\kern-.15em/
   \kern-.15em\lower.25ex\hbox{\the\scriptfont0 #2}}
  {\kern0em\raise.5ex\hbox{\the\scriptscriptfont0 #1}\kern-.2em/
   \kern-.15em\lower.25ex\hbox{\the\scriptscriptfont0 #2}}
  {#1\!/#2}}

\def\1b {{\bf 1}}
\def\Ab {{\bf A}}
\def\bb {{\bf b}}
\def\Eb {{\bf E}}
\def\fb {{\bf f}}
\def\Fb {{\bf F}}
\def\gb {{\bf g}}
\def\Ib {{\bf I}}
\def\Lb {{\bf L}}
\def\mb {{\bf m}}
\def\vb {{\bf v}}
\def\wb {{\bf w}}
\def\Wb {{\bf W}}
\def\xb {{\bf x}}
\def\zb {{\bf z}}

\def\chib   {\boldsymbol{\chi}}
\def\deltab {\boldsymbol{\delta}}
\def\Gammab {\boldsymbol{\Gamma}}
\def\phib   {\boldsymbol{\phi}}
\def\Phib   {\boldsymbol{\Phi}}
\def\Psib   {\boldsymbol{\Psi}}
\def\rhob   {\boldsymbol{\rho}}
\def\sigmab {\boldsymbol{\sigma}}
\def\Sigmab {\boldsymbol{\Sigma}}
\def\taub   {\boldsymbol{\tau}}
\def\zetab  {\boldsymbol{\zeta}}

\def\half   {\frac{1}{2}}
\def\myhalf {\sfrac{1}{2}}

\begin{document}

%==========================================================================
% Title
%==========================================================================
\title{Electroosmosis Notes}

\maketitle

\cite{LowMachElectro} The problem setup is as follows.

The computational domain is square with $L=1.28\times 10^{-4}$~cm.
These tests were performed in two-dimensions.  We typically use
$128^2$ grid cells so $\Delta x = 1\times 10^{-6}$~cm.

We model a saltwater solution with 3 species: Na, Cl, and Water.
The viscosity is $1.05\times 10^{-2}$~g/cm$^2$.

The applied electric field in the flow direction is $10^{11}$~V/cm.


Boundary conditions are inhomogeneous Neumann, with a surface charge
density set to exactly balance the lack of charge neutrality on the interior.
In fact, if the Neumann condition is not specified correctly to a large number
of digits, the electrostatic solver will fail to converge.

We have a temporary fix in the code to handle the Lorentz force using the
discretization $qE$.  For the original formulation work we somehow need to
subtract off the charge at the boundary.


Test 1: Debye length much smaller than the channel width.
We initialize $w_{\rm Na} = 2.088\times 10^{-6}$ and
$w_{\rm Cl} = 1.68\times 10^{-6}$ so the Debye length of the initial
configuration is $3.66\times 10^{-6}$~cm.  As the simulation reaches
steady-state, there are two other Debye lengths to consider.  The first is
found by taking the state against the wall, whereas the other is the
centerline.

Theory says the steady velocity profile should be
plug-flow like with maximum velocity $XXX$, with
a velocity of $XXX (1-1/e)$ one Debye length from the wall

Test 2: Debye length comparable to the channel width.
We initialize $w_{\rm Na} = 2.088\times 10^{-9}$ and
$w_{\rm Cl} = 1.68\times 10^{-9}$ so the Debye length is
$1.16\times 10^{-4}$~cm.

\bibliographystyle{plain}
\bibliography{Electroosmosis}

\end{document}
