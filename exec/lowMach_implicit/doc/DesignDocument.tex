\documentclass[final]{siamltex}

% for red MarginPars
\usepackage{color}

% for \boldsymbol
\usepackage{amsmath}
\usepackage{latexsym}
\usepackage{graphicx}
\usepackage{geometry}
\usepackage{hyperref}

% total number of floats allowed on a page
\setcounter{totalnumber}{100}

% MarginPar
\setlength{\marginparwidth}{0.75in}
\newcommand{\MarginPar}[1]{\marginpar{\vskip-\baselineskip\raggedright\tiny\sffamily\hrule\smallskip{\color{red}#1}\par\smallskip\hrule}}

% for non-stacked fractions
\newcommand{\sfrac}[2]{\mathchoice
  {\kern0em\raise.5ex\hbox{\the\scriptfont0 #1}\kern-.15em/
   \kern-.15em\lower.25ex\hbox{\the\scriptfont0 #2}}
  {\kern0em\raise.5ex\hbox{\the\scriptfont0 #1}\kern-.15em/
   \kern-.15em\lower.25ex\hbox{\the\scriptfont0 #2}}
  {\kern0em\raise.5ex\hbox{\the\scriptscriptfont0 #1}\kern-.2em/
   \kern-.15em\lower.25ex\hbox{\the\scriptscriptfont0 #2}}
  {#1\!/#2}}

\def\Ab {{\bf A}}
\def\bb {{\bf b}}
\def\Fb {{\bf F}}
\def\gb {{\bf g}}
\def\vb {{\bf v}}
\def\Wb {{\bf W}}
\def\xb {{\bf x}}

\def\deltab {\boldsymbol{\delta}}
\def\Psib   {\boldsymbol{\Psi}}
\def\Sigmab {\boldsymbol{\Sigma}}
\def\taub   {\boldsymbol{\tau}}

\def\half   {\frac{1}{2}}
\def\myhalf {\sfrac{1}{2}}

\begin{document}

%==========================================================================
% Title
%==========================================================================
\title{Implicit Low Mach Number Binary Mixing Notes}

\maketitle

\section{Equations}
We have two incompressible fluids with densities $\bar\rho_1$ and $\bar\rho_2$, where 
each fluid does not change volume upon mixing.  Defining $\rho$ to be the density of
a mixture of these fluids, the densities of each fluid are $\rho_1 \equiv \rho c$ and 
$\rho_2 \equiv \rho(1-c)$, where $c$ is the {\it mass} fraction of the first fluid.  
Thus, $\rho = \rho_1 + \rho_2$.  Since the fluids do not change volume upon mixing, 
the equation of state will enforce that the {\it volume} fractions must sum to 1:
\begin{equation}
\frac{\rho_1}{\bar\rho_1} + \frac{\rho_2}{\bar\rho_2} =
\frac{\rho c}{\bar\rho_1} + \frac{\rho(1-c)}{\bar\rho_2} = 1 
\quad \rightarrow \quad
\rho = \left(\frac{c}{\bar\rho_1} + \frac{1-c}{\bar\rho_2}\right)^{-1}.
\end{equation}
The isothermal low Mach number equations of motion are:
\begin{eqnarray}
\partial_t\rho &=& -\nabla\cdot(\rho\vb),\\
\partial_t\rho_1 &=& -\nabla\cdot(\rho_1\vb) + \underbrace{\nabla\cdot\rho\chi\nabla c + \nabla\cdot\Psib}_{\nabla\cdot\Fb},\\
\partial_t(\rho\vb) &=& -\nabla\cdot(\rho\vb\vb^T) - \nabla\pi + \mathcal{A}_0\vb + \nabla\cdot\Sigmab + \rho\gb,
\end{eqnarray}
subject to
\begin{equation}
\nabla\cdot\vb = S \equiv \underbrace{\left(\frac{1}{\bar\rho_1}-\frac{1}{\bar\rho_2}\right)}_{-(\rho^{-1}\beta)}\nabla\cdot\Fb.
\end{equation}
Thus, $S \equiv S(\Fb)$.  The stochastic terms are:
\begin{eqnarray}
\Psib &=& \sqrt{\frac{2\chi\rho\mu_c^{-1}k_BT}{\Delta t\Delta V}}\widetilde\Wb,\label{eq:rho1 stoch}\\
\Sigmab &=& \sqrt{\frac{\eta k_B T}{\Delta t\Delta V}}\left(\Wb + \Wb^T\right),\label{eq:mom stoch}
\end{eqnarray}
where $\Wb$ and $\widetilde\Wb$ are standard white-noise random Gaussian tensor
and vector fields with uncorrelated components.  In our implementation, 
we use an ideal gas mixture for (\ref{eq:rho1 stoch}) with $\mu_c^{-1}k_BT \equiv c(1-c)$ so that
\begin{equation}
\Psib = \sqrt{\frac{2\chi\rho c(1-c)}{\Delta t\Delta V}}\widetilde\Wb
\end{equation}
Also, in (\ref{eq:mom stoch}) we use
\begin{equation}
\widehat\Wb = \sqrt{2}\left(\Wb + \Wb^T\right) \quad \rightarrow \quad
\Sigmab = \sqrt{\frac{2\eta k_B T}{\Delta t\Delta V}}\widehat\Wb,
\end{equation}
where $\widehat\Wb$ is a symmetric Gaussian random tensor field, and the
extra factor of $\sqrt{2}$ accounts for the reduction in variance due to the averaging.\\

The code has several options for the viscous stress tensor.  For ease of exposition,
we define $\mathcal{A}_0$ as a function of a velocity vector, $\vb$ (the superscript
for $\mathcal{A}_0$ indicates the temporal discretization for the transport coefficients):\\
\begin{itemize}
\item $|${\tt visc\_type}$|$=1 $\quad\rightarrow\quad \mathcal{A}_0^n\vb \equiv \nabla\cdot\beta^n\nabla\vb$.\\
\item $|${\tt visc\_type}$|$=2 $\quad\rightarrow\quad \mathcal{A}_0^n\vb \equiv \nabla\cdot\left\{\beta^n[\nabla\vb + (\nabla\vb)^T]\right\}$.\\
\item $|${\tt visc\_type}$|$=3 $\quad\rightarrow\quad \mathcal{A}_0^n\vb \equiv \nabla\cdot\left\{\beta^n[\nabla\vb + (\nabla\vb)^T] + \mathcal{I}\left(\gamma^n - \frac{2}{3}\beta^n\right)(\nabla\cdot\vb)\right\}$.\\
\end{itemize}

\section{GMRES Solver}
We also have a standalone GMRES solver of the form,
\begin{equation}
\underbrace{
\left(\begin{array}{cc}
\mathcal{A} & \mathcal{G} \\
-\mathcal{D} & 0
\end{array}\right)
}_{\Ab}
\underbrace{
\left(\begin{array}{c}
\xb_{\vb} \\
x_p
\end{array}\right)
}_{\xb}
=
\underbrace{
\left(\begin{array}{c}
\bb_{\vb}\\
b_p
\end{array}\right)}_{\bb},
\end{equation}
where $\xb_\vb$ and $\bb_\vb$ are staggered quantities and $x_p$ and $b_p$ are
cell-centered quantities.  The gradient operator, $\mathcal{G}$, operates on
cell-centered data and returns a staggered field.  The divergence operator, 
$\mathcal{D}$, operates on staggered data and returns a cell-centered field.  
The Helmholtz-like operator, $\mathcal{A}$, has the general form,
$\mathcal{A} = \Theta\alpha\mathcal{I} - \mathcal{A}_0$, where
$\Theta$ is a constant parameter,
$\alpha$ is a cell-centered quantity,
$\mathcal{I}$ is the identity matrix, 
and $\mathcal{A}_0$ represents the viscous stress tensor.
Both $\mathcal{A}$ and $\mathcal{A}_0$ operate on staggered data and return
a staggered field.\\

\section{Algorithm}
Here is the time-advancement scheme.\\ \\
{\bf Step 0: Initialization:}\\

Initialize $\vb^{\rm init}, \rho^0, \rho_1^0, \chi^0, \eta^0, \kappa^0$, and $\pi^0$.
Then, perform projection to obtain an initial velocity field, $\vb^0$, that satisfies
\begin{equation}
\nabla\cdot\vb^0 = S(\Fb^0); \qquad 
\Fb^0 = (\rho\chi\nabla c)^0 + \underbrace{\sqrt{\frac{2\left[\chi\rho c(1-c)\right]^0}{\Delta t\Delta V}}\widetilde\Wb^0}_{\Psib^0}.
\end{equation}
We do this by solving for $\phi$ and updating $\vb^{\rm init}$ as follows:
\begin{equation}
\nabla\cdot\frac{1}{\rho^0}\nabla\phi = \nabla\cdot\vb^{\rm init} - S^0,
\end{equation}
\begin{equation}
\vb^0 = \vb^{\rm init} - \frac{1}{\rho}\nabla\phi.
\end{equation}
{\bf Step 1: Forward Euler Scalar Predictor:}
\begin{eqnarray}
\rho^{*,n+1} &=& \rho^n + \Delta t\nabla\cdot(-\rho^n\vb^n),\\
\rho_1^{*,n+1} &=& \rho_1^n + \Delta t\nabla\cdot(-\rho_1^n\vb^n + \Fb^n),
\end{eqnarray}
with
\begin{equation}
\Fb^n = (\rho\chi\nabla c)^n + \underbrace{\sqrt{\frac{2\left[\chi\rho c(1-c)\right]}{\Delta t\Delta V}}\widetilde\Wb^n}_{\Psib^n}.
\end{equation}
Note that $\Fb^n$ has already been computed during {\bf Step 4} of the previous time step
(or the initialization {\bf Step 0} if $n=0$).\\ \\
{\bf Step 2: Crank-Nicolson Velocity Predictor:}\\ \\
First, define
\begin{equation}
S^{*,n+1} \equiv S(\Fb^{*,n+1});
\qquad
\Fb^{*,n+1} = (\rho\chi\nabla c)^{*,n+1} + \Psib^n.
\end{equation}
Then, define $\vb^{*,n+1} = \vb^n + \deltab\vb, \pi^{*,n+1} = \pi^n + \delta\pi$ and solve
for $\deltab\vb$ and $\delta\pi$:
\begin{eqnarray}
\frac{\rho^{*,n+1}(\vb^n + \deltab\vb) - \rho^n\vb^n}{\Delta t} + \nabla(\pi^n+\delta\pi) &=&\nonumber\\
&&\hspace{-1.5in}\nabla\cdot(-\rho^n\vb^n\vb^n) + \half\left[\mathcal{A}_0^n\vb^n + \mathcal{A}_0^{*,n+1}(\vb^n + \deltab\vb)\right] + \nabla\cdot\underbrace{\sqrt{\frac{2\eta^n k_B T}{\Delta t\Delta V}}\widehat\Wb^n}_{\Sigmab^n}.
\end{eqnarray}
\begin{equation}
\nabla\cdot(\vb^n+\deltab\vb) = S^{*,n+1}.
\end{equation}
We rewrite this system as
\begin{eqnarray}
\left(\frac{\rho^{*,n+1}}{\Delta t} - \half\mathcal{A}_0^{*,n+1}\right)\deltab\vb + \nabla\delta\pi &=& \frac{(\rho^n-\rho^{*,n+1})\vb^n}{\Delta t} -\nabla\pi^n\nonumber\\
&&\hspace{-0.5in}+ \nabla\cdot(-\rho^n\vb^n\vb^n) + \half\left(\mathcal{A}_0^n\vb^n + \mathcal{A}_0^{*,n+1}\vb^n\right) + \nabla\cdot\Sigmab^n,\label{eq:CN Vel Pred}
\end{eqnarray}
\begin{equation}
-\nabla\cdot\deltab\vb = \nabla\cdot\vb^n - S^{*,n+1}.
\end{equation}
Relating this to the GMRES solver, we can see that we are solving for 
$(\xb_\vb,x_p) = (\deltab\vb,\delta\pi)$ with $b_p = \nabla\cdot\vb^n-S^{*,n+1}$ (note the change in sign!) 
and $\bb_\vb$ equal to the right-hand-side of (\ref{eq:CN Vel Pred}).  For the Helmholtz-like operator, 
$\mathcal{A}=\Theta\alpha\mathcal{I} - \mathcal{A}_0$, we have $\Theta=1/\Delta t, \alpha=\rho^{*,n+1}, 
\beta=\eta/2$, and $\gamma=\kappa/2$.\\ \\
{\bf Step 3: Trapezoidal Scalar Corrector:}
\begin{eqnarray}
\rho^{n+1} &=& \half\rho^n + \half\left[\rho^{*,n+1} + \Delta t\nabla\cdot(-\rho^{*,n+1}\vb^{*,n+1})\right],\\
\rho_1^{n+1} &=& \half\rho_1^n + \half\left[\rho_1^{*,n+1} + \Delta t\nabla\cdot(-\rho_1^{*,n+1}\vb^{*,n+1} + \Fb^{*,n+1})\right].
\end{eqnarray}
Note that $\Fb^{*,n+1}$ has already been computed during {\bf Step 2}.\\ \\
{\bf Step 4: Crank-Nicolson Velocity Corrector:}\\ \\
First, define
\begin{equation}
S^{n+1} \equiv S(\Fb^{n+1});
\qquad
\Fb^{n+1} = (\rho\chi\nabla c)^{n+1} + \underbrace{\sqrt{\frac{2\left[\chi\rho c(1-c)\right]^{n+1}}{\Delta t\Delta V}}\widetilde\Wb^{n+1}}_{\Psib^{n+1}}.
\end{equation}
Note the use of time-advanced stochastic concentration fluxes for use in $S$.  Finally, 
define $\vb^{n+1} = \vb^n + \deltab\vb$ and $\pi^{n+1} = \pi^n + \delta\pi$ and
solve the following system for $(\deltab\vb,\delta\pi)$:
\begin{eqnarray}
\left(\frac{\rho^{n+1}}{\Delta t} - \half\mathcal{A}_0^{n+1}\right)\deltab\vb + \nabla\delta\pi &=& \frac{(\rho^n-\rho^{n+1})\vb^n}{\Delta t} -\nabla\pi^n\nonumber\\
&&\hspace{-1.25in}+ \half\nabla\cdot(-\rho^n\vb^n\vb^n - \rho^{n+1}\vb^{*,n+1}\vb^{*,n+1}) + \half\left(\mathcal{A}^n_0\vb^n + \mathcal{A}^{n+1}_0\vb^n\right) + \nabla\cdot\Sigmab^n,
\end{eqnarray}
\begin{equation}
-\nabla\cdot\deltab\vb = \nabla\cdot\vb^n - S^{n+1}.
\end{equation}

\section{Boundary Conditions}
The types of non-periodic boundary conditions we consider are (i) no-slip impermeable walls,
(ii) free-slip impermeable walls, and (iii) a reservoir conditions.  We impose boundary
conditions on the primitive variables, i.e., $\rho,c$, and $\vb$.\\

{\bf No-Slip Impermeable Wall:}  For impermeable walls, there can be no diffusive
flux, $F_n=0$, and zero advective flux, $\rho v_n = \rho_1 v_n = 0$.  The zero
diffusive flux implies a homogeneous Neumann condition on $c$, and also $\rho$ since it
can be derived from $c$ and the EOS.  The zero advective flux implies that $v_n=0$ on the
boundary.  The no-slip condition implies that $v_{\taub}=0$ on the boundary.\\

{\bf Free-Slip Impermeable Wall:}  This is the same situation as the no-slip impermeable
wall, except that the tangential velocities is subject to the condition that the
tangential compoment of the normal viscous stres vanishes.  In other words, for each 
tangential direction $\tau$,
\begin{equation}
\eta\left(\frac{\partial v_n}{\partial\tau} + \frac{\partial v_{\tau}}{\partial n}\right) = 0
\end{equation}
Note that for impermeable walls, $v_n=0$ and therefore we simply require $\partial v_{\tau}/\partial n=0$.\\

{\bf Reservoir:} We specify $c$ at the boundary, and use the EOS to obtain $\rho$ at the
boundary.  The normal velocity must be equal to $-(\rho^{-1}\beta)F_n$ in order for
the EOS to remain satisfied.  NOTE: For now we will use $v_{\tau}=0$ but must write the
code to handle the free-slip condition.

\end{document}
