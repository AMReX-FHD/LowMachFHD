%% LyX 2.0.8.1 created this file.  For more info, see http://www.lyx.org/.
%% Do not edit unless you really know what you are doing.
\documentclass[english,11pt,aps,letter,superscriptaddress,nofootinbib,pre]{revtex4}
\usepackage[T1]{fontenc}
\usepackage[latin9]{inputenc}
\setcounter{secnumdepth}{3}
\usepackage{xcolor}
\usepackage{amsmath}
\usepackage{esint}
\PassOptionsToPackage{normalem}{ulem}
\usepackage{ulem}

\makeatletter

%%%%%%%%%%%%%%%%%%%%%%%%%%%%%% LyX specific LaTeX commands.
\providecolor{lyxadded}{rgb}{1,0,0}
\providecolor{lyxdeleted}{rgb}{0,0,1}
%% Change tracking with ulem
\DeclareRobustCommand{\lyxadded}[3]{{\color{lyxadded}{}#3}}
\DeclareRobustCommand{\lyxdeleted}[3]{{\color{lyxdeleted}\sout{#3}}}

%%%%%%%%%%%%%%%%%%%%%%%%%%%%%% Textclass specific LaTeX commands.
\@ifundefined{textcolor}{}
{%
 \definecolor{BLACK}{gray}{0}
 \definecolor{WHITE}{gray}{1}
 \definecolor{RED}{rgb}{1,0,0}
 \definecolor{GREEN}{rgb}{0,1,0}
 \definecolor{BLUE}{rgb}{0,0,1}
 \definecolor{CYAN}{cmyk}{1,0,0,0}
 \definecolor{MAGENTA}{cmyk}{0,1,0,0}
 \definecolor{YELLOW}{cmyk}{0,0,1,0}
}

%%%%%%%%%%%%%%%%%%%%%%%%%%%%%% User specified LaTeX commands.
\usepackage{ae,aecompl}

%\usepackage[normalmargins,normalbib,normaltitle,normalsections,normalindent]{savetrees}
%\usepackage[normalmargins,normalbib,normaltitle]{savetrees}

\providecommand{\onlinecite}{\cite}
\providecommand{\onlineref}{\ref}
\providecommand{\FToth}{Fejes T\'oth}

% for non-stacked fractions
\newcommand{\sfrac}[2]{\mathchoice
  {\kern0em\raise.5ex\hbox{\the\scriptfont0 #1}\kern-.15em/
   \kern-.15em\lower.25ex\hbox{\the\scriptfont0 #2}}
  {\kern0em\raise.5ex\hbox{\the\scriptfont0 #1}\kern-.15em/
   \kern-.15em\lower.25ex\hbox{\the\scriptfont0 #2}}
  {\kern0em\raise.5ex\hbox{\the\scriptscriptfont0 #1}\kern-.2em/
   \kern-.15em\lower.25ex\hbox{\the\scriptscriptfont0 #2}}
  {#1\!/#2}}

\makeatletter
\DeclareMathSizes{\@xipt}{10}{6}{5}
\makeatother

\setlength{\abovedisplayskip}{0.25ex}\setlength{\belowdisplayskip}{0.25ex}\setlength{\abovedisplayshortskip}{0.1ex}\setlength{\belowdisplayshortskip}{0.1ex}

\makeatother

\usepackage{babel}
\begin{document}
\global\long\def\V#1{\boldsymbol{#1}}
\global\long\def\M#1{\boldsymbol{#1}}
\global\long\def\Set#1{\mathbb{#1}}


\global\long\def\D#1{\Delta#1}
\global\long\def\d#1{\delta#1}


\global\long\def\norm#1{\left\Vert #1\right\Vert }
\global\long\def\abs#1{\left|#1\right|}


\global\long\def\grad{\M{\nabla}}
\global\long\def\avv#1{\langle#1\rangle}
\global\long\def\av#1{\left\langle #1\right\rangle }


\global\long\def\myhalf{\sfrac{1}{2}}
\global\long\def\mythreehalves{\sfrac{3}{2}}



\title{Fluctuating Hydrodynamics of Isothermal Reactive Binary Mixtures}

\maketitle

\section{Compressible Equations}

The starting point of our investigations is the system of isothermal
compressible equations of fluctuating hydrodynamics for the density
$\rho(\V r,t)$, velocity $\V v(\V r,t)$, and mass concentration
$c(\V r,t)$ for a mixture of two fluids, A and B, in $d$ dimensions.
The two components undergo a chemical reaction
\[
\nu_{1}^{+}A+\nu_{2}^{+}B\leftrightarrow\nu_{1}^{-}A+\nu_{2}^{-}B
\]
The standard stochiometric coefficients are $\V{\nu}=\V{\nu}^{-}-\V{\nu}^{+}$
(negative for reactants), and mass conservation requires that $\nu_{1}m_{1}+\nu_{2}m_{2}=0$,
where $\V m$ are the molecular masses. For a binary reaction we can
think of the effective reaction as
\[
\left(\nu_{1}^{+}-\nu_{1}^{-}\right)A\leftrightarrow\left(\nu_{2}^{-}-\nu_{2}^{+}\right)B\quad\equiv\quad\left(-\nu_{1}\right)A\leftrightarrow\nu_{2}B.
\]


In a nonlinear Langevin FHD formalism for multispecies reactive mixtures
\cite{MultispeciesChemistry,GENERIC_Chemical}, the additional contribution
to the mass density equation from this reaction can be written in
the form (in the kinetic or Klimontovich interpretation \cite{KineticStochasticIntegral_Ottinger})
{[}\textbf{Donev: In }\cite{MultispeciesChemistry}\textbf{ we used
a reaction time and some other prefactors instead of a rate, but it
is all just irrelevant constants for our purpose.}{]}
\begin{align}
\left(\partial_{t}\V{\rho}\right)_{\text{react}} & =\left(\V{\nu}\odot\V m\right)\left[-r\tilde{\mathcal{A}}+\left(2r\frac{\tilde{\mathcal{A}}}{\mathcal{A}}\right)^{\frac{1}{2}}\diamond\check{\mathcal{W}}\right],\label{eq:GENERIC}
\end{align}
where $\odot$ denote element-wise product and $r\left(P,T\right)$
is a reaction rate parameter assumed to be independent of the composition.
The expression in the square brackets is the number flux. The nonlinear
expression 
\[
\tilde{\mathcal{A}}=\exp\left(\sum_{k}\tilde{\mu}_{k}^{-}\right)-\exp\left(\sum_{k}\tilde{\mu}_{k}^{+}\right)=\left(\prod_{k}e^{\tilde{\mu}_{k}^{-}}-\prod_{k}e^{\tilde{\mu}_{k}^{+}}\right),
\]
is related to the reaction \emph{affinity} 
\[
\mathcal{A}=\sum_{k}\tilde{\mu}_{k}^{-}-\sum_{k}\tilde{\mu}_{k}^{+}=\left(k_{B}T\right)^{-1}\sum_{k}\nu_{k}m_{k}\mu_{k}.
\]
Here $\tilde{\mu}_{k}^{\pm}=\left(k_{B}T\right)^{-1}\nu_{k}^{\pm}m_{k}\mu_{k}$
are related to the (mass-density-based) chemical potentials $\mu_{k}$.
Note that the variance of the stochastic forcing in this nonlinear
Langevin equation can be seen to be positive because $\tilde{\mathcal{A}}$
and $\mathcal{A}$ always have the same sign \cite{GENERIC_Chemical}.

Note that both $\mathcal{A}$ and $\tilde{\mathcal{A}}$ are equal
to zero at chemical equilibrium. If we linearize (\ref{eq:GENERIC})
around a steady state to first order in the affinity $\mathcal{A}$,
then we can alternatively write the noise term as a sum over the forward
and the reverse reaction
\begin{equation}
\mbox{variance}\sim2\frac{\tilde{\mathcal{A}}}{\mathcal{A}}\approx\exp\left(\sum_{k}\tilde{\mu}_{k}^{+}\right)+\exp\left(\sum_{k}\tilde{\mu}_{k}^{-}\right),\label{eq:linearized}
\end{equation}
which is nonlinear and evidently positive. This is the form found
in most works on the subject \cite{KeizerBook,NonEqThermo_Bedeaux,FluctuatingReactionDiffusion,LLNS_ReactionDiffusion,LLNS_ReactionDiffusion2},
and when applied to a nonlinear setting it gives the CLE \cite{MultispeciesChemistry},
\begin{align}
\left(\partial_{t}\V{\rho}\right)_{\text{react}}= & \left(\V{\nu}\odot\V m\right)r\left(\prod_{k}e^{\tilde{\mu}_{k}^{+}}-\prod_{k}e^{\tilde{\mu}_{k}^{-}}\right)\label{eq:CLE}\\
+ & \left(\V{\nu}\odot\V m\right)r^{\frac{1}{2}}\left[\left(\prod_{k}e^{\tilde{\mu}_{k}^{-}/2}\right)\check{\mathcal{W}}^{-}+\left(\prod_{k}e^{\tilde{\mu}_{k}^{+}/2}\right)\check{\mathcal{W}}^{+}\right].\nonumber 
\end{align}
In the linearized setting at thermodynamic equilibrium the LME (\ref{eq:GENERIC})
and (\ref{eq:CLE}) are indistinguishable, but they are different
in the nonlinear setting even at equilibrium, as well as in the linearized
setting outside of equilibrium \cite{MultispeciesChemistry}.

For a binary mixture, we have that affinity is related to the reduced
chemical potential of the mixture $\mu$ as used in \cite{LowMachExplicit},
\[
\mathcal{A}=\left(k_{B}T\right)^{-1}\left(\nu_{1}m_{1}\mu_{1}+\nu_{2}m_{2}\mu_{2}\right)=\left(k_{B}T\right)^{-1}\nu_{1}m_{1}\left(\mu_{1}-\mu_{2}\right)=\left(k_{B}T\right)^{-1}\nu_{1}m_{1}\mu.
\]
We also have that at equilibrium $\tilde{\mathcal{A}}=0$ which gives
\[
\exp\left(\sum_{k}\tilde{\mu}_{k}^{+}\right)+\exp\left(\sum_{k}\tilde{\mu}_{k}^{-}\right)=2\exp\left(\sum_{k}\tilde{\mu}_{k}^{+}\right)=2\exp\left(\sum_{k}\tilde{\mu}_{k}^{-}\right)=2\Upsilon,
\]
i.e., close to equilibrium $\tilde{\mathcal{A}}\approx\Upsilon\mathcal{A}$
where $\Upsilon$ is a constant, i.e., the nonlinear and linear affinities
are equivalent up to a constant. Therefore, if we redefine the chemical
rate as $\kappa=\nu_{1}^{2}m_{1}^{2}\left(k_{B}T\right)^{-1}r\Upsilon$,
near equilibrium we have that \textcolor{red}{
\begin{align}
\left(\partial_{t}\rho_{1}\right)_{\text{react}}^{\text{eq}} & \approx\nu_{1}m_{1}\left[-r\Upsilon\mathcal{A}+\left(2r\Upsilon\right)^{\frac{1}{2}}\check{\mathcal{W}}\right]=-\kappa\mu+\left(2\kappa k_{B}T\right)^{\frac{1}{2}}\check{\mathcal{W}}.\label{eq:LinearizedChemistry}
\end{align}
}This matches the expressions derived by Grossman from equilibrium
fluctuation-dissipation balance \cite{FluctChemistry_Grossmann},
and will be useful for the linearized analysis performed later.

In terms of mass and momentum densities the equations can be written
as conservation laws \cite{OttingerBook,FluctHydroNonEq_Book,LowMachExplicit},
\begin{align}
\partial_{t}\rho+\grad\cdot\left(\rho\V v\right)= & \;0\nonumber \\
\partial_{t}\left(\rho\V v\right)+\grad\cdot\left(\rho\V v\V v^{T}\right)= & -\grad P+\grad\cdot\left[\eta\left(\grad\V v+\grad^{T}\V v\right)+\M{\Sigma}\right]+\rho\V g\nonumber \\
\partial_{t}\left(\rho_{1}\right)+\grad\cdot\left(\rho_{1}\V v\right)= & \grad\cdot\left[\rho\chi\left(\grad c+K_{P}\grad P\right)+\M{\Psi}\right]+\left(\partial_{t}\rho_{1}\right)_{\text{react}},\label{LLNS_primitive}
\end{align}
where $\rho_{1}=\rho c$ is the density of the first component, $\rho_{2}=(1-c)\rho$
is the density of the second component, $P(\rho,c;T)$ is the equation
of state for the pressure at the reference temperature $T=T_{0}=\mbox{const.}$,
and $\V g$ is the gravitational acceleration. Here we neglected bulk
viscosity since this is not expected to enter in the low Mach limit.
The shear viscosity $\eta$, bulk viscosity $\kappa$, mass diffusion
coefficient $\chi$, reaction rate $r$, and baro-diffusion coefficient
$K_{P}$, in general, depend on the state. 

The capital Greek letters denote stochastic momentum and mass fluxes
that are formally modeled as \cite{LLNS_Staggered} 
\begin{align}
\M{\Sigma}=\sqrt{\eta k_{B}T}\left(\M{\mathcal{W}}+\M{\mathcal{W}}^{T}\right)\mbox{ and } & \M{\Psi}=\sqrt{2\chi\rho\mu_{c}^{-1}k_{B}T}\;\widetilde{\M{\mathcal{W}}},\label{stoch_flux_covariance}
\end{align}
where $k_{B}$ is Boltzmann's constant, and $\M{\mathcal{W}}(\V r,t)$
and $\widetilde{\M{\mathcal{W}}}(\V r,t)$ are standard zero mean,
unit variance random Gaussian tensor and vector fields with uncorrelated
components,
\[
\avv{\mathcal{W}_{ij}(\V r,t)\mathcal{W}_{kl}(\V r^{\prime},t')}=\delta_{ik}\delta_{jl}\;\delta(t-t^{\prime})\delta(\V r-\V r^{\prime}),
\]
and similarly for $\widetilde{\M{\mathcal{W}}}$.


\subsection{Thermodynamic Identities}

The various coefficients appearing in the equations are not unrelated
to each other. Let us define the following coefficients, which can
be taken as arbitrary:
\begin{eqnarray*}
c_{T}^{2} & = & \left(\partial P/\partial\rho\right)_{c}\quad\text{(isothermal speed of sound)}\\
\mu_{c} & = & \left(\partial\mu/\partial c\right)_{P}\quad\text{(second derivative of Gibbs free energy)}\\
\beta & = & \frac{1}{\rho}\left(\frac{\partial\rho}{\partial c}\right)_{P}\quad\text{(solutal expansion coefficient)}.
\end{eqnarray*}
Thermodynamics then gives expressions for all other derivatives and
coefficients in terms of these three coefficients, which depend on
the type of fluid under consideration.

First, we have the Maxwell identity \cite{Landau:StatPhys1}
\begin{equation}
\left(\frac{\partial\mu}{\partial P}\right)_{c}=-\frac{1}{\rho^{2}}\left(\frac{\partial\rho}{\partial c}\right)_{P}=-\frac{\beta}{\rho}.\label{eq:MaxwellId}
\end{equation}
The baro-diffusion coefficient $K_{P}$ is not a transport coefficient
but rather determined from thermodynamics via (\ref{eq:MaxwellId}),
\begin{equation}
K_{P}=\frac{\left(\partial\mu/\partial P\right)_{c}}{\left(\partial\mu/\partial c\right)_{P}}=-\rho^{-2}\frac{\left(\partial\rho/\partial c\right)_{P}}{\left(\partial\mu/\partial c\right)_{P}}=-\frac{\beta}{\rho\mu_{c}}.\label{eq:K_P}
\end{equation}


We will now use the mathematical identities for derivatives in the
presence of an EOS constraint relating three variables (in our case
$\rho$, $c$ and $P$), $z=z(x,y)$ and $w=w(x,y)$,
\[
\left(\frac{\partial x}{\partial y}\right)_{z}=-\left(\frac{\partial x}{\partial z}\right)_{y}\left(\frac{\partial z}{\partial y}\right)_{x}
\]
\[
\left(\frac{\partial z}{\partial x}\right)_{w}=\left(\frac{\partial z}{\partial x}\right)_{y}+\left(\frac{\partial z}{\partial y}\right)_{x}\left(\frac{\partial y}{\partial x}\right)_{w},
\]
which follow from the chain rule. From these and (\ref{eq:MaxwellId})
we can derive
\[
\left(\frac{\partial P}{\partial c}\right)_{\rho}=-\left(\frac{\partial P}{\partial\rho}\right)_{c}\left(\frac{\partial\rho}{\partial c}\right)_{P}=\beta\rho c_{T}^{2}
\]


\[
\left(\frac{\partial\mu}{\partial c}\right)_{\rho}=\left(\frac{\partial\mu}{\partial c}\right)_{P}+\left(\frac{\partial\mu}{\partial P}\right)_{c}\left(\frac{\partial P}{\partial c}\right)_{\rho}=\mu_{c}+\beta^{2}c_{T}^{2},
\]
and, using the standard chain rule in one variable and (\ref{eq:MaxwellId}),
\[
\left(\frac{\partial\mu}{\partial\rho}\right)_{c}=\left(\frac{\partial\mu}{\partial P}\right)_{c}\left(\frac{\partial P}{\partial\rho}\right)_{c}=-\frac{\beta c_{T}^{2}}{\rho}.
\]
These will be crucial in the linearized analysis to follow.


\subsection{Linearized Analysis}

Some of the most important quantities predicted by the fluctuating
hydrodynamics equations are the equilibrium structure factors (static
covariances) of the fluctuating fields. These can be obtained by linearizing
the compressible equations (\ref{LLNS_primitive}) around a uniform
reference state, $\rho=\rho_{0}+\d{\rho}$, $c=c_{0}+\d c$, $\V v=\d{\V v}$,
$P=P_{0}+\d P$, $\mu=\mu_{0}+\d{\mu}$, where 
\begin{eqnarray}
\d P & = & \left(\frac{\partial P}{\partial\rho}\right)_{c}\left(\d{\rho}\right)+\left(\frac{\partial P}{\partial c}\right)_{\rho}\left(\d c\right)=c_{T}^{2}\left(\d{\rho}\right)-\beta\rho c_{T}^{2}\left(\d c\right),\nonumber \\
\delta\mu & = & \left(\frac{\partial\mu}{\partial\rho}\right)_{c}\left(\d{\rho}\right)+\left(\frac{\partial\mu}{\partial c}\right)_{\rho}\left(\d c\right)=-\frac{\beta c_{T}^{2}}{\rho}\left(\d{\rho}\right)+\left(\mu_{c}+\beta^{2}c_{T}^{2}\right)\left(\d c\right).\label{eq:dmu_isothermal}
\end{eqnarray}
Using the linearized form of the chemical production (\ref{eq:LinearizedChemistry}),
we get the linearized equations at thermodynamic equilibrium {[}\textbf{Donev:
I did not add subscripts zero here for the base state}{]}:
\begin{align}
\partial_{t}\left(\d{\rho}\right)= & -\rho\grad\cdot\V v\nonumber \\
\partial_{t}\left(\d{\V v}\right)= & -\frac{c_{T}^{2}}{\rho}\grad\left(\d{\rho}\right)+\beta c_{T}^{2}\grad\left(\d c\right)+\nu\grad\cdot\left(\grad\left(\d{\V v}\right)+\grad^{T}\left(\d{\V v}\right)\right)+\sqrt{\nu\rho^{-1}k_{B}T}\,\grad\cdot\left(\M{\mathcal{W}}+\M{\mathcal{W}}^{T}\right)\nonumber \\
\partial_{t}\left(\d c\right)= & \chi\grad^{2}\left(\d c\right){\color{magenta}-\frac{\chi\beta}{\rho\mu_{c}}\left(c_{T}^{2}\grad^{2}\left(\d{\rho}\right)-\beta\rho c_{T}^{2}\grad^{2}\left(\d c\right)\right)}+\sqrt{2\chi\rho^{-1}\mu_{c}^{-1}k_{B}T}\;\grad\cdot\widetilde{\M{\mathcal{W}}}\label{eq:LinearizedCompressible}\\
 & {\color{red}-\frac{\kappa}{\rho}\left(-\frac{\beta c_{T}^{2}}{\rho}\left(\d{\rho}\right)+\left(\mu_{c}+\beta^{2}c_{T}^{2}\right)\left(\d c\right)\right)}+\left(2\kappa\rho^{-2}k_{B}T\right)^{\frac{1}{2}}\check{\mathcal{W}},\nonumber 
\end{align}
where the (rather non-trivial!) deterministic barodiffusion terms
are shown in magenta and the reactive terms are shown in red.

Owing to fluctuation-dissipation balance the static structure factors
are independent of the wavevector $\V k$ at thermodynamic equilibrium,
\begin{eqnarray}
S_{\rho,\rho}\left(\V k\right)= & \av{\left(\widehat{\d{\rho}}\right)\left(\widehat{\d{\rho}}\right)^{\star}} & =\frac{\rho_{0}k_{B}T_{0}}{c_{T}^{2}}+\beta^{2}\frac{\rho_{0}k_{B}T_{0}}{\mu_{c}}\nonumber \\
\M S_{\V v,\V v}\left(\V k\right)= & \av{(\widehat{\delta\V v})(\widehat{\d{\V v}})^{\star}} & =\rho_{0}^{-1}k_{B}T_{0}\,\M I\nonumber \\
S_{c,c}\left(\V k\right)= & \av{\left(\widehat{\d c}\right)\left(\widehat{\d c}\right)^{\star}} & =\frac{k_{B}T_{0}}{\rho_{0}\mu_{c}}.\label{eq:S_equilibrium}
\end{eqnarray}
Note that density fluctuations do not vanish even in the incompressible
limit $c_{T}\rightarrow\infty$ unless $\beta=0$.


\subsection{Dimerization reaction for $\beta=0$}

Let us first take the supposedly easy case of $\beta=0$, i.e., $\rho=\text{const.}$
independent of concentration. In this case the momentum and density
equations decouple from the concentration equation in the linearization,
and we get the single reaction-diffusion equation for the concentration,
\begin{eqnarray}
\partial_{t}\left(\d c\right) & = & \chi\grad^{2}\left(\d c\right)+\sqrt{2\chi\rho^{-1}\mu_{c}^{-1}k_{B}T}\;\grad\cdot\widetilde{\M{\mathcal{W}}}\nonumber \\
 &  & -\frac{\kappa}{\rho}\mu_{c}\left(\d c\right)+\left(2\kappa\rho^{-2}k_{B}T\right)^{\frac{1}{2}}\check{\mathcal{W}}.\label{eq:beta_zero}
\end{eqnarray}
This equation is thermodynamically consistent, because both reaction
and diffusion are separately consistent with $S_{c,c}=k_{B}T/\left(\rho\mu_{c}\right)$.
However, notice that in the chemical reaction rate what enters is
$\mu_{c}$; in particular, we see that the reaction part of the linearized
equation must be written as
\begin{equation}
\partial_{t}\left(\d c\right)_{\text{react}}=-r\left(\d c\right)+\sqrt{2\frac{k_{B}T}{\rho\mu_{c}}r}\,\check{\mathcal{W}},\label{eq:react_diff_thermo}
\end{equation}
for some constant $r$ in order to be thermodynamically consistent.

In the compressible chemistry paper \cite{MultispeciesChemistry}
we considered a dimerization reaction $2A\leftrightarrow A_{2}$ in
an ideal gas mixture, $m_{1}=m$ and $m_{2}=2m$. But, note that for
an ideal gas mixture, it is \emph{not} true that $\beta=0$. For an
ideal gas, we took $n_{0}=\rho/m=\rho_{1}/m_{1}+2\rho_{2}/m_{2}=n_{1}+2n_{2}$,
the total number density of A particles contained in both monomers
and dimers, to be spatially constant in the reaction-diffusion approximation.
This is identical to taking density to be constant consistent with
our assumption $\beta=0$, giving
\begin{eqnarray*}
n_{0} & = & n_{1}+2n_{2}=\text{const.}
\end{eqnarray*}
For an ideal mixture (not necessarily an ideal gas {[}\textbf{Donev:
Worth checking this claim explicitly}{]}), we have
\[
\mu_{c}=\frac{k_{B}T}{c\left(1-c\right)\left[cm_{2}+\left(1-c\right)m_{1}\right]}=\frac{k_{B}T}{mc\left(1-c^{2}\right)}.
\]



\subsubsection{Number-density based LMA}

Assume now that we use the usual law of mass action and take the chemical
production to be given by (the noise here is chosen to follow the
CLE form but this is not important in the linearization)
\begin{equation}
\left(\partial_{t}c\right)_{\text{react}}=-2\frac{m_{1}}{\rho}\left(k^{+}n_{1}^{2}-k^{-}n_{2}\right)+2\frac{m_{1}}{\rho}\left(k^{+}n_{1}^{2}+k^{-}n_{2}\right)^{\frac{1}{2}}\check{\mathcal{W}}.\label{eq:LMA_n}
\end{equation}
If we now eliminate $n_{2}$ in terms of $n_{1}$, and express the
result back in terms of mass fraction, and set the deterministic term
to zero when $c=c_{0}$, we get that the forward rate should be
\[
k^{+}=\frac{k^{-}}{n_{0}}\frac{1-c_{0}}{2c_{0}^{2}}.
\]
If we now linearize the reaction around $c_{0}$, we get
\begin{equation}
\partial_{t}\left(\d c\right)_{\text{react}}=-\frac{c_{0}\left(2-c_{0}\right)}{c_{0}^{2}}k^{-}\left(\d c\right)+\left(\frac{4\left(1-c_{0}\right)}{n_{0}}k^{-}\right)^{\frac{1}{2}}\check{\mathcal{W}}.\label{eq:LMA_dimer_n}
\end{equation}


But the LMA equation (\ref{eq:LMA_dimer_n}) is \emph{not} of the
required form (\ref{eq:react_diff_thermo}). For this to be true we
require that
\[
\frac{\frac{c_{0}\left(2-c_{0}\right)}{c_{0}^{2}}k^{-}}{\frac{4\left(1-c_{0}\right)}{n_{0}}k^{-}}=\frac{n_{0}\left(2-c_{0}\right)}{4c_{0}(1-c_{0})}\overset{?}{=}\frac{\rho\mu_{c}}{2k_{B}T}=\frac{n_{0}}{2c_{0}\left(1-c_{0}^{2}\right)},
\]
which is not true for general $c_{0}$. In particular, for $c_{0}=1/2$,
which is the case we studied in \cite{MultispeciesChemistry}, we
get $3/2\neq4/3$. This means that the law of mass action kinetics
assumed is not consistent with thermodynamics, even in the compressible
case, if we take $\beta=0$! In particular, the LMA formulation (\ref{eq:LMA_dimer_n})
for $c_{0}=1/2$ combined with diffusion gives
\[
\partial_{t}\left(\d c\right)=\chi\grad^{2}\left(\d c\right)+\sqrt{2\chi\left(\frac{3}{8n_{0}}\right)}\;\grad\cdot\widetilde{\M{\mathcal{W}}}-3k^{-}\left(\d c\right)+\sqrt{\frac{2k^{-}}{n_{0}}}\,\check{\mathcal{W}},
\]
which gives the non-constant structure factor (see identical formula
in (34) in\cite{MultispeciesChemistry}) 
\[
S(k)=\frac{3}{8n_{0}}\frac{8/9+k^{2}d^{2}}{1+k^{2}d^{2}},
\]
where the penetration depth is $d^{2}=\chi/\left(3k^{-}\right)$.


\subsubsection{Mole-fraction based LMA}

As shown in Eq. (18) in \cite{MultispeciesChemistry}, the correct
formulation of the LMA for general ideal mixtures (not necessarily
ideal gas) uses \emph{mole fractions} and not number densities. This
gives the CLE, with $n=n_{1}+n_{2}$,
\begin{equation}
\left(\partial_{t}c\right)_{\text{react}}=-2\frac{m_{1}}{\rho}\left(k^{+}\left(\frac{n_{1}}{n}\right)^{2}-k^{-}\frac{n_{2}}{n}\right)+2\frac{m_{1}}{\rho}\left(k^{+}\left(\frac{n_{1}}{n}\right)^{2}+k^{-}\frac{n_{2}}{n}\right)^{\frac{1}{2}}\check{\mathcal{W}}.\label{eq:LMA_mole}
\end{equation}
Repeating the calculation now gives
\[
k^{+}=\frac{k^{-}}{n_{0}}\frac{1-c_{0}^{2}}{2c_{0}^{2}},
\]
and a linearized equation with reactive part 
\begin{equation}
\partial_{t}\left(\d c\right)_{\text{react}}=-\frac{4}{n_{0}c_{0}\left(1+c_{0}\right)^{2}}k^{-}\left(\d c\right)+\left(\frac{8\left(1-c_{0}\right)}{n_{0}^{2}\left(1+c_{0}\right)}k^{-}\right)^{\frac{1}{2}}\check{\mathcal{W}}.\label{eq:LMA_dimer_mole}
\end{equation}
This mole-fraction-based LMA equation (\ref{eq:LMA_dimer_n}) \emph{is}
of the required form (\ref{eq:react_diff_thermo}), since
\[
\frac{\frac{4}{n_{0}c_{0}\left(1+c_{0}\right)^{2}}k^{-}}{\frac{8\left(1-c_{0}\right)}{n_{0}^{2}\left(1+c_{0}\right)}k^{-}}=\frac{n_{0}}{2c_{0}\left(1-c_{0}^{2}\right)}=\frac{\rho\mu_{c}}{2k_{B}T},
\]
and now everything is thermodynamically consistent and the structure
factor is flat.




\section{Low Mach Equations}

The asymptotic low Mach analysis of (\ref{LLNS_primitive}) is standard
{[}\textbf{Donev: Not really, we don't actually know how to precisely
derive these equations with barodiffusion included...}{]} and follows
the procedure outlined in Ref. \cite{IncompressibleLimit_Majda},
formally treating the stochastic forcing as smooth. This analysis
leads to the \emph{isothermal reactive low Mach number} equations
for a binary mixture of fluids in conservation form,
\begin{align}
\partial_{t}\rho+\grad\cdot\left(\rho\V v\right)= & 0\label{eq:rho_eq}\\
\partial_{t}\left(\rho\V v\right)+\nabla\pi=-\grad\cdot\left(\rho\V v\V v^{T}\right) & +\grad\cdot\left[\eta\left(\grad\V v+\grad^{T}\V v\right)+\M{\Sigma}\right]+\rho\V g\label{eq:momentum_eq}\\
\partial_{t}\left(\rho_{1}\right)=-\grad\cdot\left(\rho_{1}\V v\right)+ & \grad\cdot\V F+\left(\partial_{t}\rho_{1}\right)_{\text{react}}^{P_{0}}\label{eq:rho1_eq}\\
\mbox{such that }\grad\cdot\V v= & -\left(\rho^{-1}\beta\right)\,\grad\cdot\V F,\label{eq:div_v_constraint}
\end{align}
where the non-advective (diffusive and stochastic) fluxes are denoted
with
\[
\V F=\rho\chi\grad c+\M{\Psi}.
\]
The gradient of the non-thermodynamic component of the pressure $\pi$
(Lagrange multiplier) appears in the momentum equation as a driving
force that ensures the EOS constraint (\ref{eq:div_v_constraint})
is obeyed. The superscript $P_{0}$ on the reactive term means here
that the activities need to be evaluated at constant pressure.

In this work we consider a specific linear EOS, 
\begin{equation}
\frac{\rho_{1}}{\bar{\rho}_{1}}+\frac{\rho_{2}}{\bar{\rho}_{2}}=\frac{c\rho}{\bar{\rho}_{1}}+\frac{(1-c)\rho}{\bar{\rho}_{2}}=1,\label{eq:EOS_quasi_incomp}
\end{equation}
where $\bar{\rho}_{1}$ and $\bar{\rho}_{2}$ are the densities of
the pure component fluids ($c=1$ and $c=0$, respectively), giving
\begin{equation}
\beta=\rho\left(\frac{1}{\bar{\rho}_{2}}-\frac{1}{\bar{\rho}_{1}}\right)=\frac{\bar{\rho}_{1}-\bar{\rho}_{2}}{c\bar{\rho}_{2}+(1-c)\bar{\rho}_{1}}.\label{eq:beta_simple}
\end{equation}
It is important that for this specific form of the EOS $\beta/\rho$
is a material constant independent of the concentration. The density
dependence (\ref{eq:beta_simple}) on concentration arises if one
assumes that the two fluids do not change volume upon mixing.


\subsection{Linearized Analysis}

We now examine the spatio-temporal correlations of the steady-state
fluctuations in the low Mach number equations (\ref{eq:momentum_eq},\ref{eq:rho1_eq},\ref{eq:div_v_constraint},\ref{eq:rho_eq}).
Note that a density gradient will accompany a concentration gradient,
and this can introduce some additional terms in $\V F$ depending
on how $\rho\chi$ depends on concentration. For simplicity, we assume
$\rho\chi$ is a constant so that the diffusive term $\grad\cdot\V F$
in (\ref{eq:rho1_eq}) is simply $\rho\chi\grad^{2}c$. We also assume
the viscosity $\eta$ is spatially constant.

Because we are evaluating things at constant pressure, the activity,
which for the binary mix is the same as the chemical potential (really
difference of chemical potentials), can be linearized as
\begin{equation}
\delta\mu=\left(\frac{\partial\mu}{\partial c}\right)_{P}\left(\d c\right)=\mu_{c}\left(\d c\right),\label{eq:dmu_isobaric}
\end{equation}
which no longer contains $\beta$ or $c_{T}$; this is identical to
(\ref{eq:dmu_isothermal}) but using the fact that $\rho=\rho(c)$
is given by (\ref{eq:EOS_quasi_incomp}) and therefore $\delta\rho=\rho\beta\,\delta c$
. We thus obtain the simplified linearized reactive low Mach equations,
\begin{align}
\partial_{t}\left(\d{\V v}\right)= & -\rho^{-1}\grad\pi+\nu\grad^{2}\left(\d{\V v}\right)+\sqrt{\nu\rho^{-1}k_{B}T}\,\grad\cdot\left(\M{\mathcal{W}}+\M{\mathcal{W}}^{T}\right)\nonumber \\
\partial_{t}\left(\d c\right)= & \chi\grad^{2}\left(\d c\right)+\sqrt{2\chi\rho^{-1}\mu_{c}^{-1}k_{B}T}\;\grad\cdot\widetilde{\M{\mathcal{W}}}\nonumber \\
 & {\color{red}-\frac{\kappa}{\rho}\mu_{c}\left(\d c\right)+\left(2\kappa\rho^{-2}k_{B}T\right)^{\frac{1}{2}}\check{\mathcal{W}}}\nonumber \\
\grad\cdot\left(\d{\V v}\right)= & -\beta\partial_{t}\left(\d c\right).\label{eq:simpl_primitive_eqs}
\end{align}


Observe that the concentration equation is uncoupled from the velocity
equation, and is \emph{identical} to (\ref{eq:beta_zero}), i.e.,
to the full compressible equation with $\beta=0$. Therefore, it is
thermodynamically consistent in the sense that it will give the expected
structure factor for concentration even in the presence of reactions.
Specifically, it gives the following dynamic and static structure
factors:
\begin{equation}
S_{c,c}\left(\V k,\omega\right)=\frac{2k_{B}T\,\left(\chi\,{k}^{2}\rho+\kappa\,\mu_{{c}}\right)}{\left({\kappa}^{2}{\mu_{{c}}}^{2}+2\,\chi\,{k}^{2}\kappa\,\rho\,\mu_{{c}}+{\rho}^{2}\left({\chi}^{2}{k}^{4}+{\omega}^{2}\right)\right)\mu_{{c}}}\label{eq:S_kw_c_LM}
\end{equation}
\[
S_{c,c}\left(\V k\right)=\frac{1}{2\pi}\int_{\infty}^{\infty}S_{c,c}\left(\V k,\omega\right)d\omega=\frac{k_{B}T}{\rho\mu_{c}}.
\]



\subsection{The Bad News: Low Mach Limit?}

The only bad news is that it is still unclear how to go from the full
compressible equations to the low Mach equations. I computed the dynamic
structure factor from the full compressible equations and got a very
complicated expression that contained $\beta$ and $\mu_{c}$ and
$c_{T}$ in nontrivial ways. This seems to contradict a statement
in Appendix A in \cite{LowMachExplicit} where we claim that our low
Mach equations give the same result for $S_{c,c}\left(\V k,\omega\right)$
as the full compressible equations. I am not sure where we got that
statement and whether some approximations were involved.

I was able to get the much simpler result (\ref{eq:S_kw_c_LM}) by
taking the following limit:
\begin{eqnarray*}
c_{T} & \rightarrow & \infty\\
\mu_{c} & = & C\, c_{T}^{2}\\
\kappa\mu_{c} & = & K,
\end{eqnarray*}
where $C$ and $K$ are some (finite) constants. This limit is somewhat
sensible, in that for a binary mixture we know that $\mu_{c}\sim c_{T}^{2}\sim k_{B}T/m$,
where $m$ is a typical molecular mass. Recall also the thermodynamic
relation
\[
\left(\frac{\partial\mu}{\partial c}\right)_{\rho}=\mu_{c}+\beta^{2}c_{T}^{2}.
\]


A better way to justify the limit would be to think a bit more about
what it means physically. Physically, $c_{T}$ and $\mu_{c}$ are
constants and don't go to infinity. Instead, what we are assuming
here is that $k$ and $\omega$ are such that the two side Brillouin
peaks are far separated from the central peak. Perhaps there is an
additional assumption that the peaks are also in some sense narrow.
I am not really sure how to make it precise and how to connect $\mu_{c}$
back into it, since the side peaks depend on $c_{T}$ only and not
on $\mu_{c}$.




\subsection{Nonlinear Formulation of Chemistry}

To go beyond the linearized equations and still be consistent with
a traditional LMA we need to use the full formalism described in \cite{MultispeciesChemistry},
and use tau-leaping/SSA with forward/backward reaction rate (see (\ref{eq:CLE}))
\[
r^{\pm}=r\prod_{k}e^{\tilde{\mu}_{k}^{\pm}},
\]
which we can convert this into the more standard LMA form based on
mole fractions (but \emph{not} number densities, as explained above)
\cite{MultispeciesChemistry}. Specifically, general multispecies
mixtures, we can express chemical potentials as a sum of ideal and
excess contributions, 
\[
\mu_{k}\left(\V x,T,P\right)=\left(\mu_{k}^{0}\left(T,P\right)+\frac{k_{B}T}{m_{k}}\ln\left(x_{k}\right)\right)+\frac{k_{B}T}{m_{k}}\ln\left(\gamma_{k}\right),
\]
where $\mu_{k}^{0}\left(T,P\right)$ is a reference chemical potential
(e.g., pure liquid state at standard conditions), and $\gamma_{k}\left(\V x,T,P\right)$
is the activity coefficient of species $k$; for an ideal mixture
$\gamma_{k}=1$. In the low Mach number setting we consider here,
the chemical potentials depends on pressure \emph{only} through the
reference state. This is always true for an ideal mixture, but may
be assumed more generally so long as the activities only depend on
composition and \emph{not} on pressure.

For chemistry it is more convenient to work with a dimensionless chemical
potential per particle, 
\[
\hat{\mu}_{s}=\frac{m_{s}\mu_{s}}{k_{B}T}=\ln(x_{s}\gamma_{s})+\hat{\mu}_{s}^{o},
\]
where $\hat{\mu}_{s}^{o}=(m_{s}\mu_{s}^{o})/(k_{B}T)$, and $x_{s}\gamma_{s}$
is the activity (i.e., effective concentration). This gives 
\[
\exp\left(\tilde{\mu}_{s}^{\pm}\right)=\exp\left(\nu_{k}^{\pm}\hat{\mu}_{s}^{o}\right)\left(x_{s}\gamma_{s}\right)^{\nu_{s}^{\pm}}\;\;,
\]
which leads to a generalized law-of-mass action (LMA) of the form
\begin{eqnarray}
\left(\partial_{t}\rho_{s}\right)_{\text{react}} & = & \nu_{s}m_{s}\left(\kappa_{r}^{+}\prod_{s^{\prime}}\left(x_{s^{\prime}}\gamma_{s^{\prime}}\right)^{\nu_{{s^{\prime}}}^{+}}-\kappa_{r}^{-}\prod_{s^{\prime}}\left(x_{s^{\prime}}\gamma_{s^{\prime}}\right)^{\nu_{{s^{\prime}}}^{-}}\right),\label{eq:LMA_generalized}
\end{eqnarray}
where $\kappa_{r}^{\pm}(T,p,\V X)$ are the more familiar forward/reverse
reaction rates (per unit time and per unit volume). Since there is
only one independent timescale parameter $r$, the forward and reverse
rates are not independent and the LMA gives the ratio to be the equilibrium
constant,
\begin{equation}
K_{r}(T,P)=\frac{\kappa_{r}^{+}}{\kappa_{r}^{-}}=\left[\frac{\prod_{s}\left(x_{s}\gamma_{s}\right)^{\nu_{{k}}^{-}}}{\prod_{s}\left(x_{s}\gamma_{s}\right)^{\nu_{{k}}^{+}}}\right]_{\mathrm{eq}}=\exp\left(-\sum_{s}\nu_{s}\hat{\mu}_{s}^{o}\right),\label{eq:eq_const-1}
\end{equation}
which is a purely thermodynamic quantity (closely related to the dimensionless
reference Gibbs energy for the reaction at a unit reference pressure)
that can be calculated from pure component data~\cite{zemansky1981heat,NonEqThermo_Bedeaux}.


\subsection{Equilibrium Free Energy beyond Gaussian Approximation}

If everything is done right, for an ideal mixture with $\beta=0$
{[}\textbf{Donev: Handling beta nonzero requires a lot more thought}{]},
the thermodynamic equilibrium state should be consistent with the
Einstein distribution with an ideal entropy of mixing {[}\textbf{Donev:
I think this is right because the exponential has the form} $(N_{k}m_{k}\mu_{k})/k_{B}T$\textbf{
which seems right}{]} at constant mass, i.e., constant density,
\begin{equation}
P\left(\V n\right)\sim\exp\left[-\left(n\D V\right)\sum_{k}x_{k}\,\left(\ln\left(x_{k}/x_{k}^{\text{eq}}\right)-1\right)\right]=\exp\left[-\D V\sum_{k}n_{k}\,\left(\ln\left(x_{k}/x_{k}^{\text{eq}}\right)-1\right)\right],\label{eq:eq_distro_LM}
\end{equation}
including going beyond the Gaussian approximation. Here $x_{k}^{\text{eq}}$
is the equilibrium mole fraction, which is determined by the equilibrium
constant. For the dimerization reaction in a single cell with $c_{\text{eq}}=1/2$,
we have the mass conservation $n_{1}+2n_{2}=n_{0}$ relating $n_{1}$
and $n_{2}$, where $n_{0}=\rho/m$ is the number of $A$ atoms, and
$n_{1}^{\text{eq}}=n_{0}/2$, $n_{2}^{\text{eq}}=n_{0}/4$, $n_{eq}=3n_{0}/4$.
The distribution (\ref{eq:eq_distro_LM}) should be the equilibrium
density for the mole fraction in each cell (and different cells should
be independent). It is related to but different from the Poisson distribution
for the number of molecules of each species which applies in the ideal
gas case.

Note that for an ideal gas mixture, as we considered in \cite{MultispeciesChemistry},
we used the ideal-gas entropy, which would give 
\begin{equation}
P\left(\V n\right)\sim\exp\left[-\D V\sum_{k}n_{k}\,\left(\ln\left(n_{k}/n_{k}^{\text{eq}}\right)-1\right)\right].\label{eq:eq_distro_gas}
\end{equation}
Expressing the argument of the exponential in terms of $c=n_{1}m_{1}/\rho=n_{1}/n_{0}$,
this gives exactly the formula (28) in \cite{MultispeciesChemistry},
which is derived in Appendix B of that paper using combinatorics without
using the known entropy of a mixture. But the important thing is that
(\ref{eq:eq_distro_gas}) is \emph{different} from (\ref{eq:eq_distro_LM})
because $x_{k}=n_{k}/n$ and $n_{k}$ are different because $n$ varies.
For an ideal gas constant pressure means constant $n$ so the two
are then the same.

In the previous \cite{MultispeciesChemistry} we showed that (\ref{eq:eq_distro_gas})
is the steady-state solution of the Fokker-Planck equation for the
LME equation when the LMA is expressed in terms of number densities.
\textcolor{red}{So I think that we can convince ourselves that (\ref{eq:eq_distro_LM})
is the right answer in the Boussinesq case $\beta=0$ if Changho can
confirm that (\ref{eq:eq_distro_LM}) is the steady-state solution
of the Fokker-Planck equation for the LME equation when the LMA is
expressed in terms of }\textcolor{red}{\emph{mole}}\textcolor{red}{{}
fractions. If one does not want to use the LMA one can start from
the CME itself (with rates based on mole fractions!) and then use
Stirling's approximation in the solution; this is a bit more work
though and I am pretty sure it will give the same answer as the CME
to leading order. But, solving the CME is easy and we can regard that
answer as the ``truth'' to compare to since that does not even use
Stirling's approximation.}

\bibliographystyle{unsrt}
\bibliography{References,GarciaGeneralBibFile,MScaleProp}

\end{document}
