\documentclass[final]{siamltex}

% for red MarginPars
\usepackage{color}

% for \boldsymbol
\usepackage{amsmath}

% for \mathfrak
\usepackage{amsfonts}

% total number of floats allowed on a page
\setcounter{totalnumber}{100}

% float page fractions
\renewcommand{\topfraction}{0.9}
\renewcommand{\bottomfraction}{0.9}
\renewcommand{\textfraction}{0.2}

% MarginPar
\setlength{\marginparwidth}{0.75in}
\newcommand{\MarginPar}[1]{\marginpar{\vskip-\baselineskip\raggedright\tiny\sffamily\hrule\smallskip{\color{red}#1}\par\smallskip\hrule}}

% for non-stacked fractions
\newcommand{\sfrac}[2]{\mathchoice
  {\kern0em\raise.5ex\hbox{\the\scriptfont0 #1}\kern-.15em/
   \kern-.15em\lower.25ex\hbox{\the\scriptfont0 #2}}
  {\kern0em\raise.5ex\hbox{\the\scriptfont0 #1}\kern-.15em/
   \kern-.15em\lower.25ex\hbox{\the\scriptfont0 #2}}
  {\kern0em\raise.5ex\hbox{\the\scriptscriptfont0 #1}\kern-.2em/
   \kern-.15em\lower.25ex\hbox{\the\scriptscriptfont0 #2}}
  {#1\!/#2}}

\def\1b {{\bf 1}}
\def\bb {{\bf b}}
\def\Fb {{\bf F}}
\def\gb {{\bf g}}
\def\mb {{\bf m}}
\def\Qb {{\bf Q}}
\def\vb {{\bf v}}
\def\wb {{\bf w}}
\def\Wb {{\bf W}}
\def\xb {{\bf x}}

\def\chib   {\boldsymbol{\chi}}
\def\deltab {\boldsymbol{\delta}}
\def\Gammab {\boldsymbol{\Gamma}}
\def\phib   {\boldsymbol{\phi}}
\def\Phib   {\boldsymbol{\Phi}}
\def\Psib   {\boldsymbol{\Psi}}
\def\Sigmab {\boldsymbol{\Sigma}}
\def\taub   {\boldsymbol{\tau}}
\def\zetab  {\boldsymbol{\zeta}}

\def\half   {\frac{1}{2}}
\def\myhalf {\sfrac{1}{2}}

\begin{document}

%==========================================================================
% Title
%==========================================================================
\title{Notes for Adding Energy Equation to the Implicit Low Mach Number Multispecies Code}

\maketitle

\section{Equations}
The governing conservation equations for momentum, mass, and energy for our system are:
\begin{eqnarray}
\frac{\partial\rho\vb}{\partial t} &=& - \nabla\cdot(\rho\vb\vb) - \nabla\pi + \nabla\cdot\taub + \rho\gb,\label{eq:momentum}\\
\frac{\partial\rho_i}{\partial t} &=& -\nabla\cdot(\rho_i\vb) + \nabla\cdot F_i,\label{eq:mass}\\
\frac{\partial\rho h}{\partial t} &=& -\nabla\cdot(\rho h\vb) + \frac{DP}{Dt} + \nabla\cdot\Qb + \sum_k h_k F_k,\label{eq:enthalpy}
\end{eqnarray}
where $h_k = \partial h/\partial w_k$, and
$\taub = \overline\taub + \tilde\taub$,
$\Fb = \overline\Fb + \tilde\Fb$, and 
$\Qb = \overline\Qb + \tilde\Qb$ are the stress tensors, mass fluxes, and heat fluxes 
respectively, separated into deterministic and stochastic components.
Starting from the full compressible equations and expanding in Mach number shows that the leading
order pressure terms in Mach number must be constant in space. Thus, the perturbational pressure $\pi$ is
$O(M^2)$.
The low Mach number system is then obtained by ignoring the effect of terms of order $M^2$ or higher in pressure 
on the thermodynamic state of the system.
The equation of state then becomes a constraint on the evolution; namely,
the evolution of the system is constrained so that
\begin{equation}
P_0 (t) = P(\rho,\wb,T).\label{eq:EOS}
\end{equation}

The deterministic stress tensor ignores the effect of bulk viscosity, 
$\overline\taub = \eta\bar\nabla\vb = \eta[\nabla\vb + (\nabla\vb)^T]$, which 
can be aborbed into $\pi$.
The deterministic heat fluxes have the form 
$\overline\Qb = \lambda\nabla T$.
The deterministic part of the mass fluxes, $\overline{\Fb}$, contains contributions from 
compositional gradients, barodiffusion, and temperature gradients,
\begin{equation}
\overline{F} = -\rho\Wb\chib\left[\Gammab\nabla\xb + (\phib - \wb)\frac{\nabla P}{n k_B T} + \zetab\frac{\nabla T}{T}\right].
\end{equation}

Equations (\ref{eq:momentum}), (\ref{eq:mass}), (\ref{eq:enthalpy})
with $DP/dt$ in the enthalpy equation replaced by $dP_0/dt$, and (\ref{eq:EOS})
form the actual system that we would like to solve.
This corresponds to evolution subject to a constraint.  From the analysis below it will become
apparent that this system is an index 3 differential algebraic system.  Before looking in detail
at the structure of the system, it is, perhaps, helpful to consider a physical interpretation of the
system.  The perturbational pressure $\pi$ constrains velocity field so that
the evolution so that the evolution of $h$ and $\rho_k$
preserves a thermodynamic pressure that is a function of $t$ only.
When the system is open, the pressure
equilibrates to the ambient pressure and $P_0$ is independent of time and known.
When the system is closed
$P_0$ represents the global pressure
required force the fluid to occupy the available volume.

We also note that $\rho, h, T, P$ are linked by the equation of state in a sense that given
$\wb$ and any two of these variables, the equation of state uniquely specifies the other
two variables.  In particular, we will thing of enthalpy as either
$h = h(\rho,\wb,T)$, or the enthalpy $h = h(P,\wb,T)$.
\MarginPar{need to think about this one for general case . . .may not need the first version}
We view these relations as defining temperature in terms of enthalpy.

\subsection{Velocity Constraint and Thermodynamic Pressure Update}

Directly attacking the evolution of the constrained system is not tractable.
For DAE's the standard approach to understanding the structure is to differentiate the constraint.
If we start on the constraint and recast the evolution in terms of the derivative of the constraint
being satisfied, the resulting evolution of the system is analytically equivalent.  In the theory
of DAE's the number of times the system needs to be differentiated to recast it as a pure initial
value problem is referred to as the index.
The derivative of the constraint can also be cast as a characterization of the tangent plane to the
constraint manifold but it is not clear what to do with that observation.

The key things we want to get a handle on are how the two different elements that go into the
constraint interact to control the evolution and whether they are, in some sense, uniquely specified.

We begin by differentiating the 
right-hand side of equation (\ref{eq:EOS}) along particle paths,
\begin{equation}
\frac{DP}{Dt} = P_\rho\frac{D\rho}{Dt} + P_T\frac{DT}{Dt} + \sum_kP_{w_k}\frac{Dw_k}{Dt}.
\end{equation}
$P_0$ is spatially constant
so we can replace $DP/Dt$ 
with $d P_0/ dt$:
\begin{equation}
\frac{d P_0}{dt} = P_\rho\frac{D\rho}{Dt} + P_T\frac{DT}{Dt} + \sum_kP_{w_k}\frac{Dw_k}{Dt}.\label{eq:particle paths}
\end{equation}
As noted above we are also make the
substitution $DP/Dt = d P_0/ d t$ in the enthalpy equation.  (Since we
are currently assuming $P_0$ is constant in space, we are not allowing
for pressure stratification due to gravity.)\\

Next, by noting that by continuity $D\rho/Dt = -\rho\nabla\cdot\vb$, we can rewrite 
(\ref{eq:particle paths}) as
\begin{equation}
\nabla\cdot\vb = \frac{1}{\rho P_\rho}\left(-\frac{\partial P_o}{\partial t} + P_T\frac{DT}{Dt} + \sum_kP_{w_k}\frac{Dw_k}{Dt}\right).\label{eq:constraint1}
\end{equation}
To obtain an expression for $DT/Dt$, we differentiate the enthalpy, $h=h(P,\wb,T)$, 
along particle paths to obtain
\begin{equation}
\frac{Dh}{Dt} = 
c_p \frac{DT}{Dt} 
+ h_P\frac{d P_0}{d t} + \sum_k h_{k}\frac{Dw_k}{Dt}  \;\;  ,
\end{equation}
which then gives
\begin{equation}
\frac{DT}{Dt} = \frac{1}{c_p}\left(\frac{Dh}{Dt} - h_P\frac{d P_0}{d t} - \sum_kh_{k}\frac{Dw_k}{Dt}\right).\label{eq:DTDt1}
\end{equation}
From equation (\ref{eq:mass}) and 
continuity we have,
\begin{equation}
\frac{Dw_k}{Dt} = \frac{1}{\rho}\nabla\cdot\Fb_k,\label{eq:DwDt}
\end{equation}
and from equation (\ref{eq:enthalpy}) we have,
\begin{equation}
\frac{Dh}{Dt} = \frac{1}{\rho}\left(\frac{\partial P_0}{\partial t} + \nabla\cdot\Qb\right).\label{eq:DhDt}
\end{equation}
Combining (\ref{eq:DTDt1}), (\ref{eq:DwDt}), and (\ref{eq:DhDt}) we have
\begin{equation}
\frac{DT}{Dt} = \frac{1}{\rho c_p}\left[\left(1 - \rho h_P\right)\frac{\partial P_0}{\partial t} + \nabla\cdot\Qb - \sum_kh_{w_k}\nabla\cdot\Fb_k\right].\label{eq:DTDt}
\end{equation}
Combining equations (\ref{eq:constraint1}) and (\ref{eq:DTDt}) gives
\footnote{These are the same equations in MAESTRO Paper III, Almgren et al., ApJ, 2008, except that paper includes a convective derivative of $P_0$ due to stratification}
\begin{equation}
\nabla\cdot\vb + \alpha\frac{\partial P_0}{\partial t} = \sum_k \beta_k \nabla\cdot\Fb_k + \gamma \nabla\cdot\Qb \equiv S,\label{eq:constraint}
\end{equation}
\begin{equation}
\alpha = -\frac{1}{\rho^2 P_\rho}\left[\frac{(1-\rho h_p)P_T - \rho c_p}{c_p}\right], \quad
\beta_k = \frac{1}{\rho^2 P_\rho}\left(P_{w_k} - \frac{P_T h_{w_k}}{c_p}\right), \quad
\gamma = \frac{1}{\rho^2 P_\rho}\frac{P_T}{c_P}.
\end{equation}

The unknowns in equation (\ref{eq:constraint}) are $\vb$ and $d P_0/d t$.
For open containers where $\vb$ is allowed to be non-zero on one or more walls $P_0$
is constant and the constraint equation reduces to
\begin{equation}
\nabla\cdot\vb = S.
\end{equation}
This is a generalization of standard low Mach number approaches (at least our version) and fits into
a framework we have dealt with deterministically before.

For closed chambers with no-flow walls the constraint is somewhat more complex.
The equation \ref{eq:constraint} then represents the derivative of the original
constraint.  We now want to see how
decomposition of this equation into mean and fluctuating (not in the stochastic sense)
components allows us to recover the evolution of $P_0$ and $\pi$.

For a closed system, $\nabla \cdot \vb = g$ only has a solution if $\int g dx = 0$
We can use this fact to simultaneously solve for the unknowns
$\vb$ and $\partial P_0/\partial t$.
To do this, we split up $S$ into an average and a perturbational component,
\begin{equation}
S = \bar{S} + \delta S.
\end{equation}
We also
split $\alpha$ into an average and a perturbational component,
\begin{equation}
\alpha = \bar{\alpha} + \delta\alpha.
\end{equation}
By definition,
\begin{equation}
\int_{\Omega} \delta S ~\partial\Omega = \int_{\Omega} \delta \alpha ~\partial\Omega = 0.\label{eq:zero int}
\end{equation}
So equation (\ref{eq:constraint}) can be rewritten as,
\begin{equation}
\nabla\cdot\vb + \bar{\alpha}\frac{d P_0}{d t} 
= \bar{S} + \delta S -
\delta\alpha\frac{d P_0}{d t}.\label{eq:constraint2}
\end{equation}
Since $\nabla \cdot \vb$ must integrate to zero and $P_0$ must be only a function of time,
equation \ref{eq:constraint2} can be uniquely composed to give
\begin{equation}
\bar{\alpha}\frac{d P_0}{d t} = \bar{S}.\label{eq:P0}
\end{equation}
and
\begin{equation}
\nabla\cdot\vb = \delta S - \delta\alpha\frac{\partial P_0}{\partial t}.\label{eq:constraint3}
\end{equation}
Thus, the original system is analytically equivalent to solving the system (\ref{eq:momentum}), (\ref{eq:mass}), and (\ref{eq:enthalpy})
subject to (\ref{eq:P0}) and (\ref{eq:constraint3}).

\section{Basic Time Stepping Strategy}

The form of the system given above can be fit into the framework of low Mach number algorithms we have
developed based on the generalized Stokes solver. 
The additional issue that arises is that since the core numerical algorithm is based on the derivative
of the constraint, we can numerically drift off of the actual algebraic constraint. Thus, we augment the
basic algorithm with iterative 
``volume discrepancy'' scheme described below to solve 
(\ref{eq:momentum}), (\ref{eq:mass}), and (\ref{eq:enthalpy})
while enforcing condition (\ref{eq:EOS}) to a specified tolerance.

At the most basic level, the time-stepping strategy looks like this:
\begin{itemize}
\item {\bf Step A}: Compute $\partial P_0/\partial t$ with equation (\ref{eq:P0}) and 
then solve for velocity using the Stokes system GMRES solver coupling equations 
(\ref{eq:momentum}) and (\ref{eq:constraint3}).\\
\item {\bf Step B}: Update thermodynamic variables using equations (\ref{eq:mass}), 
and (\ref{eq:enthalpy}).  This involves the explicit computation of advective fluxes,
explicit computation of mass diffusion, and implicit treatment of heat fluxes.\\
\item {\bf Step C}: Check to see how the updated $P(\rho,\wb,T)$ compares to the 
updated $P_0$.  If the drift 
is unacceptable, add a correction to the right-hand-side of equations
(\ref{eq:P0}) and (\ref{eq:constraint3})
designed to drive the thermodynamic variables into equilibrium with 
$P_0$ and then return to {\bf Step A}.  Details on the form of this 
correction are given below.\\
\end{itemize}
The actual time-advancement algorithm will be a bit more complicated, as we 
incorporate a predictor-corrector formulism to obtain second-order advective 
fluxes, and also incorporate an iterative scheme for heat fluxes, etc.

\section{Equation of State Drift}
Recall equation (\ref{eq:particle paths}), where we analytically enforce that
thermodynamic pressure will evolve to be consistent with the evolution of $P_0$:
\begin{equation}
\frac{\partial P_0}{\partial t} = P_\rho\frac{D\rho}{Dt} + P_T\frac{DT}{Dt} + \sum_kP_{w_k}\frac{Dw_k}{Dt}.
\end{equation}
This equation was shown to be analytically equivalent to equation (\ref{eq:constraint}):
\begin{equation}
\nabla\cdot\vb + \alpha\frac{\partial P_0}{\partial t} = S.
\end{equation}

This divergence constraint represents a linearized approximation for the velocity 
field required to ensure that the thermodynamic variables remain thermodnamically
consistent with $P_0$.  Since in general the EOS is non-linear, the updated 
thermodynamic variables will not
satisfy this condition.  We propose an iterative strategy that is already
used in our low Mach number terrestrial and astrophysical combustion codes, that
uses a modified divergence constraint designed to drive the updated 
thermodynamic variables closer to thermodynamic equilibrium with $P_0$ with each iteration.
For closed chambers, we also iteratively update $P_0$ as well.\\

After computing a velocity field with the GMRES solver, and updating the thermodynamic
variables with advective and diffusive fluxes, the EOS will no longer be satisfied.
The amount by which the EOS is not satisfied can be easily quantified by defining
the drift, $\Delta P = P_{\rm EOS}^{n+1} - P_0^{n+1}$ with $P_{\rm EOS} = P(\rho,\wb,T)$.\\

If we were able to ``redo'' the velocity computation, the rate at which we want the 
pressure to change in each cell is no longer given
by simply $\partial P_0/\partial t$, but is now given by:
\begin{equation}
\frac{DP}{Dt} = \frac{\partial P_0}{\partial t} - \frac{\Delta P}{\Delta t}.
\end{equation}
We include the $\Delta p/\Delta t$ ``volume discrepancy'' correction term here because 
now we have a sense for what the numerics does to the local thermodynamic pressure in 
each cell without such a correction.  Physically, by including this term, we are enforce 
an additional amount of expansion or contraction within each cell in order to modify 
the pressure.  Now, equation (\ref{eq:constraint}) is replaced by
\begin{equation}
\nabla\cdot\vb + \alpha\left(\frac{\partial P_0}{\partial t} - \frac{\Delta P}{\Delta t}\right) = S.
\end{equation}
or
\begin{equation}
\nabla\cdot\vb + \alpha\frac{\partial P_0}{\partial t} = S + \underbrace{\alpha\frac{\Delta P}{\Delta t}}_{S_{\rm corr}}.
\end{equation}
Similar to above, in order for this sytem to be solvable we split the update into a
equations for $P_0$ and $\vb$,
\begin{equation}
\nabla\cdot\vb = \delta S + \delta S_{\rm corr} - \delta\alpha\frac{\partial P_0}{\partial t},
\end{equation}
\begin{equation}
\bar{\alpha}\frac{\partial P_0}{\partial t} = \bar{S} + \bar{S}_{\rm corr}.
\end{equation}
As we iterate, we increment (rather than reset) $\delta S_{\rm corr}$ based on how much 
the current solution drifts using $\delta S_{\rm corr}$ the previous iteration.

\section{Heat Flux Formulation}
Here's the basic idea.  Recall (\ref{eq:enthalpy}):
\begin{equation}
\frac{\partial(\rho h)}{\partial t} = -\nabla\cdot(\rho h\vb) + \frac{\partial P_0}{\partial t} + \nabla\cdot\lambda\nabla T + \sum_k\nabla\cdot h_kF_k.
\end{equation}
We want to handle the thermal diffusion semi-implicitly.  We use an iterative method that
linearizes the relationship between how $h$ changes with respect to changes in $T$.
Note that we have already updated the densities, but unfortunately the time-advanced
mass fluxes also depend on the enthalpy.  In the full time-advancement algorithm,
the idea shall be to iterate these fluxes as well, with an initial guess coming from
an earlier step in the algorithm.  The notation here is just used to give a general
idea of the iterative scheme.  Also, the details of the treatment for 
$\partial P_0/\partial t$ will be given below.\\ \\
Let's assume we want to form an iterative update to $h$ using
\begin{eqnarray}
\frac{\rho^{n+1}h^{n+1,l+1} - (\rho h)^n}{\Delta t} &=& -\nabla\cdot(\rho h\vb)^{n+\myhalf} + \frac{\partial P_0}{\partial t}\nonumber\\
&&+ \half\nabla\cdot\lambda^n\nabla T^n + \half\nabla\cdot\lambda^{n+1,l}\nabla T^{n+1,l+1}\nonumber\\
&& + \half\sum_k\nabla\cdot h_k^n F_k^n + \half\sum_k\nabla\cdot h_k^{n+1,l}F_k^{n+1,l}.\label{eq:temp form}
\end{eqnarray}
\\
We start with initial guesses $(T,h)^{n+1,0} = (T,h)^n$.\\ \\
We say that
\begin{equation}
T^{n+1,l+1} = T^{n+1,l} + \delta T\label{eq:delta T}
\end{equation}
Linearization gives
\begin{equation}
h^{n+1,l+1} = h^{n+1,l} + c_p^{n+1,l}\delta T\label{eq:delta h}
\end{equation}
\\
Substituting (\ref{eq:delta T}) and (\ref{eq:delta h}) into (\ref{eq:temp form}) gives:
\begin{eqnarray}
\frac{\rho^{n+1}(h^{n+1,l} + c_p^{n+1,l}\delta T) - (\rho h)^n}{\Delta t} &=& -\nabla\cdot(\rho h\vb)^{n+\myhalf} + \frac{\partial P_0}{\partial t}\nonumber\\
&&\hspace{-0.5in}+ \half\nabla\cdot\lambda^n\nabla T^n + \half\nabla\cdot\lambda^{n+1,l}(T^{n+1,l} + \delta T)\nonumber\\
&&\hspace{-0.5in} + \half\sum_k\nabla\cdot h_k^n F_k^n + \half\sum_k\nabla\cdot h_k^{n+1,l} F_k^{n+1,l}.
\end{eqnarray}
\\
We solve this implicitly for $\delta T$, then compute $(T,h)^{n+1,l+1}$
using (\ref{eq:delta T}) and (\ref{eq:delta h}).  Or should we compute $h^{n+1,l+1}$ from
the EOS?

\section{Algorithm Summary}
An inertial algorithm might look like this.\\ \\
Initialization:\\
\begin{enumerate}
\item Compute $\Fb$ and $\Qb$ at the initial time and project.
\item Loop over the following:
\begin{enumerate}
\item Predictor forward Euler step for thermodynamic variables.
\item Advance $P_0$.
\item If satisfied with thermodynamic drift, exit this loop.
\item Determine the volume discrepancy correction and redefine $S$.
\item Project to obtain an updated velocity field.
\end{enumerate}
\item Proceed to Corrector.\\
\end{enumerate}
Corrector:\\
\begin{enumerate}
\item Compute preliminary time-advanced $\Fb$ and $\Qb$.
\item Loop over the following
\begin{enumerate}
\item GMRES solve for preliminary time-advanced velocity.
\item Trapezoidal scalar corrector step for thermodynamic variables.
\item Advance $P_0$.
\item If satisfied with thermodynamic drift, exit this loop.
\item Determine the volume discrepancy correction and redefine $S$.  The weighting of the
correction may have to change since this velocity field is used in a trapezoidal corrector.
I'm also not sure if/how to relate this correction to the predictor correction.
\end{enumerate}
\item Proceed to New-Time Predictor.\\
\end{enumerate}
New-Time Predictor:\\
\begin{enumerate}
\item Compute updated time-advanced $\Fb$ and $\Qb$.
\item Loop over the following:
\begin{enumerate}
\item GMRES solve for updated time-advanced velocity.
\item Predictor forward Euler step for thermodynamic variables.
\item Advance $P_0$.
\item If satisfied with thermodynamic drift, exit this loop.
\item Determine the volume discrepancy correction and redefine $S$.
\end{enumerate}
\item Return to Corrector.
\end{enumerate}

\section{Algorithm Details}
Assume we are given initial $\rho_i$ and $h$ that are thermodynamically consistent with 
$P_0$.  If they are not, compute e.g., $h = h(\rho,P_0,\wb)$.  Initialize the velocity
to zero, although this shouldn't matter beyond solver tolerance as it will get 
overwritten during the initialization step.\\ \\
{\bf Step 0: Initialization}.\\ \\
This step is only performed once at the beginning of the simulation, so here $n=0$.
The goal of this step is to compute preliminary time-advanced thermodynamically
consistent pressure and scalars, $(P_0,\rho_i,(\rho h))^{*,n+1}$.  This will
require the computation of a velocity field, $\vb^n$, that preserves this
thermodynamic balance.  
We set the initial volume discrepancy correction to zero, 
$S_{\rm corr}^n = 0$, as well as the individual terms in the decomposition,
$S_{\rm corr}^n = \bar{S}_{\rm corr}^n + \delta S_{\rm corr}^n$.
Begin by computing fluxes, $\Fb^n, \Qb^n$ and then
loop over the following until the convergence criteria in {\bf Step 0d} is met:\\
\begin{itemize}
\item {\bf Step 0a:} Compute a pressure update:
\begin{equation}
P_0^{*,n+1} = P_0^n + \frac{\Delta t(\bar{S}^n + \bar{S}_{\rm corr}^n)}{\bar{\alpha}^n}
\end{equation}
\item {\bf Step 0b:} Compute the velocity field using
\begin{equation}
\nabla\cdot\vb^n = \delta S^n + \delta S_{\rm corr}^n - \delta\alpha^n\underbrace{\frac{\bar{S}^n + \bar{S}_{\rm corr}^n}{\bar{\alpha}^n}}_{\partial P_0/\partial t}.
\end{equation}
\item {\bf Step 0c:} Advance the densities using forward-Euler advective fluxes
and explicit mass diffusion:
\begin{equation}
\rho_i^{*,n+1} = \rho_i^n + \Delta t\left[-\nabla\cdot(\rho_i\vb)^n + \Fb^n\right].
\end{equation}
\item {\bf Step 0d:} Advance the enthalpy by iteratively looping over an energy solve.
The fundamental formulation is given by,
\begin{equation}
(\rho h)^{*,n+1} = (\rho h)^n + \Delta t\left[-\nabla\cdot(\rho h\vb)^n + \frac{\bar{S}^n + \bar{S}_{\rm corr}^n}{\bar{\alpha}^n} + \frac{1}{2}(\Qb^n + \Qb^{*,n+1})\right],
\end{equation}
but since we express $\Qb$ in terms of temperature gradients, we need an iterative
solve that linearizes the depences of $h$ on $T$ (summarized above, still need to
write this out here).\\
\item {\bf Step 0e:} If the thermodynamic drift is unacceptable, update the volume 
discrepancy correction,
\MarginPar{Temporal discretization of $\alpha$ may not be the best choice here.}
\begin{equation}
S_{\rm corr}^n = S_{\rm corr}^n + \alpha^n\left(\frac{P_{\rm EOS}^{*,n+1} - P_0^{*,n+1}}{\Delta t}\right)
\end{equation}
and return to {\bf Step 0a}.  Otherwise, proceed to the {\bf Scalar Corrector Step}.\\
\end{itemize}
{\bf Step 1: Scalar Corrector Step}.\\ \\
The goal of this step is to compute updated time-advanced thermodynamically consistent 
pressure and scalars, $(P_0,\rho_i,(\rho h))^{n+1}$.  This will require the computation
of a velocity field, $\vb^{*,n+1}$, that preserves this thermodynamic balance.
We set the volume discrepancy correction to zero,
$S_{\rm corr}^{*,n+1} = 0$, as well as the individual terms in the decomposition
$S_{\rm corr}^{*,n+1} = \bar{S}_{\rm corr}^{*,n+1} + \delta S_{\rm corr}^{*,n+1}$
Begin by computing preliminary time-advanced fluxes, $\Fb^{*,n+1}, \Qb^{*,n+1}$,
and then loop over the following until the convergence criteria in {\bf Step 1d} is met:\\
\begin{itemize}
\item {\bf Step 1a:} Pressure update.
\begin{equation}
P_0^{n+1} = P_0^n + \frac{\Delta t}{2}\frac{(\bar{S}^n + \bar{S}_{\rm corr}^n)}{\bar{\alpha}^n} + \frac{\Delta t}{2}\frac{(\bar{S}^{*,n+1} + \bar{S}_{\rm corr}^{*,n+1})}{\bar{\alpha}^{*,n+1}}
\end{equation}
\item {\bf Step 1b:} GMRES solve for the velocity, $\vb^{*,n+1}$, and perturbational
pressure, $\pi^{*,n+1}$.
\begin{equation}
\frac{\rho^{*,n+1}\vb^{*,n+1} - \rho^n\vb^n}{\Delta t} + \nabla\pi^{*,n+1} = \nabla\cdot(-\rho\vb\vb)^n + \frac{1}{2}\nabla\cdot(\taub^n + \taub^{*,n+1}) + \frac{1}{2}(\rho^n + \rho^{*,n+1})\gb,
\end{equation}
\begin{equation}
\nabla\cdot\vb^{*,n+1} = \delta S^{*,n+1} + \delta S_{\rm corr}^{*,n+1}
\end{equation}
\item {\bf Step 1c:} Advect-diffuse the thermodynamic variables using trapezoidal 
corrector advective fluxes, trapezoidal corrector mass diffusion, and implicit energy 
diffusion:
\begin{eqnarray}
\rho_i^{n+1} &=& \rho_i^n + \frac{\Delta t}{2}\left[-\nabla\cdot(\rho_i\vb)^n -\nabla\cdot(\rho_i\vb)^{*,n+1} + \Fb^n + \Fb^{*,n+1}\right],\nonumber \\
\\
(\rho h)^{n+1} &=& (\rho h)^n + \frac{\Delta t}{2}\left[-\nabla\cdot(\rho h\vb)^n -\nabla\cdot(\rho h\vb)^{*,n+1} + \frac{\bar{S}^n + \bar{S}_{\rm corr}^n}{\bar{\alpha}^n} + \frac{\bar{S}^{*,n+1}+ \bar{S}_{\rm corr}^{*,n+1}}{\bar{\alpha}^{*,n+1}} + \Qb^n + \Qb^{n+1}\right].\nonumber\\
\end{eqnarray}
\item {\bf Step 1d:} If the thermodynamic drift is unacceptable, update the volume 
discrepancy correction,
\begin{equation}
S_{\rm corr}^{*,n+1} = S_{\rm corr}^{*,n+1} + \frac{2}{\rho P_\rho}\left(\frac{P_{\rm EOS}^{n+1} - P_0^{n+1}}{\Delta t}\right)
\end{equation}
and return to {\bf Step 1a}.  I think the factor of 2 is required here since the new-time
velocity is used in a trapezoidal corrector with only a half-weighting.  Otherwise, 
proceed to the {\bf Scalar Predictor Step}.\\
\end{itemize}
{\bf Step 2: Scalar Predictor Step.}\\ \\
We are now in the next time step, so we increment $n$.
The goal of this step is to compute preliminary time-advanced thermodynamically
consistent pressure and scalars, $(P_0,\rho_i,(\rho h))^{*,n+1}$.  This will
require the computation of a velocity field, $\vb^n$, that preserves this
thermodynamic balance.  Note that this step is essentially the same as
{\bf Step 0}, except that in the computation of $\vb^n$ we use a GMRES
solver rather than a projection, since in essence we are correcting the velocity
field to second order by including a trapezoidal discretization of the advective fluxes.
We set the initial volume discrepancy correction to zero, 
$S_{\rm corr}^n = 0$, as well as the individual terms in the decomposition,
$S_{\rm corr}^n = \bar{S}_{\rm corr}^n + \delta S_{\rm corr}^n$.
Begin by computing fluxes, $\Fb^n, \Qb^n$ and then loop over the following until 
the convergence criteria in {\bf Step 2d} is met:\\
\begin{itemize}
\item {\bf Step 2a:} Compute a pressure update:
\begin{equation}
P_0^{*,n+1} = P_0^n + \frac{\Delta t(\bar{S}^n + \bar{S}_{\rm corr}^n)}{\bar{\alpha}^n}
\end{equation}
\item {\bf Step 2b:} GMRES solve for the velocity, $\vb^n$, and perturbational
pressure, $\pi^n$.
\begin{equation}
\frac{\rho^n\vb^n - \rho^{n-1}\vb^{n-1}}{\Delta t} + \nabla\pi^n = \half\nabla\cdot(-(\rho\vb\vb)^{n-1} - (\rho\vb\vb)^{*,n}) + \frac{1}{2}\nabla\cdot(\taub^{n-1} + \taub^n) + \frac{1}{2}(\rho^{n-1} + \rho^n)\gb,
\end{equation}
\begin{equation}
\nabla\cdot\vb^n = \delta S^{*,n} + \delta S_{\rm corr}^n
\end{equation}
\item {\bf Step 2c:} Advect-diffuse the thermodynamic variables using forward-Euler advective fluxes,
explicit mass diffusion, and implicit energy diffusion:
\begin{eqnarray}
\rho_i^{*,n+1} &=& \rho_i^n + \Delta t\left[-\nabla\cdot(\rho_i\vb)^n + \Fb^n\right], \\
(\rho h)^{*,n+1} &=& (\rho h)^n + \Delta t\left[-\nabla\cdot(\rho h\vb)^n + \frac{\bar{S}^n + \bar{S}_{\rm corr}^n}{\bar{\alpha}^n} + \frac{1}{2}(\Qb^n + \Qb^{*,n+1})\right].\nonumber\\
\end{eqnarray}
\item {\bf Step 2d:} If the thermodynamic drift is unacceptable, update the volume 
discrepancy correction,
\begin{equation}
S_{\rm corr}^n = S_{\rm corr}^n + \frac{1}{\rho P_\rho}\left(\frac{P_{\rm EOS}^{*,n+1} - P_0^{*,n+1}}{\Delta t}\right)
\end{equation}
and return to {\bf Step 2a}.  Otherwise, proceed to the {\bf Scalar Corrector Step}.\\
\end{itemize}
\end{document}
