\documentclass[final]{siamltex}

% for red MarginPars
\usepackage{color}

% for \boldsymbol
\usepackage{amsmath}

% for \mathfrak
\usepackage{amsfonts}

% total number of floats allowed on a page
\setcounter{totalnumber}{100}

% float page fractions
\renewcommand{\topfraction}{0.9}
\renewcommand{\bottomfraction}{0.9}
\renewcommand{\textfraction}{0.2}

% MarginPar
\setlength{\marginparwidth}{0.75in}
\newcommand{\MarginPar}[1]{\marginpar{\vskip-\baselineskip\raggedright\tiny\sffamily\hrule\smallskip{\color{red}#1}\par\smallskip\hrule}}

% for non-stacked fractions
\newcommand{\sfrac}[2]{\mathchoice
  {\kern0em\raise.5ex\hbox{\the\scriptfont0 #1}\kern-.15em/
   \kern-.15em\lower.25ex\hbox{\the\scriptfont0 #2}}
  {\kern0em\raise.5ex\hbox{\the\scriptfont0 #1}\kern-.15em/
   \kern-.15em\lower.25ex\hbox{\the\scriptfont0 #2}}
  {\kern0em\raise.5ex\hbox{\the\scriptscriptfont0 #1}\kern-.2em/
   \kern-.15em\lower.25ex\hbox{\the\scriptscriptfont0 #2}}
  {#1\!/#2}}

\def\1b {{\bf 1}}
\def\bb {{\bf b}}
\def\Fb {{\bf F}}
\def\gb {{\bf g}}
\def\mb {{\bf m}}
\def\Qb {{\bf Q}}
\def\vb {{\bf v}}
\def\wb {{\bf w}}
\def\Wb {{\bf W}}
\def\xb {{\bf x}}

\def\chib   {\boldsymbol{\chi}}
\def\deltab {\boldsymbol{\delta}}
\def\Gammab {\boldsymbol{\Gamma}}
\def\phib   {\boldsymbol{\phi}}
\def\Phib   {\boldsymbol{\Phi}}
\def\Psib   {\boldsymbol{\Psi}}
\def\Sigmab {\boldsymbol{\Sigma}}
\def\taub   {\boldsymbol{\tau}}
\def\zetab  {\boldsymbol{\zeta}}

\def\half   {\frac{1}{2}}
\def\myhalf {\sfrac{1}{2}}

\begin{document}

%==========================================================================
% Title
%==========================================================================
\title{Notes for Adding Energy Equation to the Implicit Low Mach Number Multispecies Code}

\maketitle

\section{Equations}
The equations for the momentum and densities:
\begin{eqnarray}
\frac{\partial\rho\vb}{\partial t} &=& - \nabla\cdot(\rho\vb\vb) - \nabla\pi + \nabla\cdot(\overline\taub + \tilde\taub) + \rho\gb,\label{eq:momentum}\\
\frac{\partial\rho_i}{\partial t} &=& -\nabla\cdot(\rho_i\vb) + \nabla\cdot(\overline\Fb_i + \tilde\Fb_i),\label{eq:mass}\\
\frac{\partial\rho h}{\partial t} &=& -\nabla\cdot(\rho h\vb) + \frac{DP}{Dt} + \nabla\cdot(\overline\Qb + \tilde\Qb),\label{eq:enthalpy}
\end{eqnarray}
where $\taub$, $\Fb$, and $\Qb$ are the stress tensors, mass fluxes, and heat fluxes 
respectively, divided into deterministic and stochastic components.
The deterministic part of the mass fluxes, $\overline{F}$, contains contributions from 
compositional gradients and temperature gradients (we do not consider barodiffusion here),
\begin{equation}
\overline{F} = -\rho\Wb\chib\left[\Gammab\nabla\xb + \zetab\frac{\nabla T}{T}\right].
\end{equation}
The deterministic stress tensor ignores the effect of bulk viscosity, 
$\overline\taub = \eta\bar\nabla\vb = \eta[\nabla\vb + (\nabla\vb)^T]$.
The deterministic heat fluxes have the form $\overline\Qb = (\lambda/c_p)\nabla h$,
where the specific heat at constant pressure is $c_p = \partial h/\partial T$.
If we wish to include differential diffusion effects, following 
Day and Bell, Combust.~Theor.~Model., 2000, one possible formulation is to add
the following term to the right-hand-side of equation (\ref{eq:enthalpy}):
\begin{equation}
\mathfrak{D} = \sum_k \nabla\cdot h_k\left(\rho\mathcal{D}_m - \frac{\lambda}{c_p}\right)\nabla w_k.
\end{equation}

\subsection{Velocity Constraint}
To derive a constraint on the velocity, we begin by writing the pressure as a function 
of $\rho, T$, and $w$ and differentiate along particle paths,
\begin{equation}
\frac{DP}{Dt} = P_\rho\frac{D\rho}{Dt} + P_T\frac{DT}{Dt} + \sum_kP_{w_k}\frac{Dw_k}{Dt}.\label{eq:particle paths}
\end{equation}
We now impose the constraint that $P$ is constant in space but varies in time, $P=P_0(t)$.
In this way we allow for pressure changes in closed containers, but are not allowing for 
pressure stratification due to a gravitational density stratification.  Thus, we replace 
the $DP/Dt$ with $\partial P_0/\partial t$ in both the enthalpy equation 
(\ref{eq:enthalpy}) and the constraint equation (\ref{eq:particle paths}).  Furthermore, 
by noting that by continuity $D\rho/Dt = -\rho\nabla\cdot\vb$, we can rewrite 
(\ref{eq:particle paths}) as
\begin{equation}
\nabla\cdot\vb = \frac{1}{\rho P_\rho}\left(-\frac{\partial P_o}{\partial t} + P_T\frac{DT}{Dt} + \sum_kP_{w_k}\frac{Dw_k}{Dt}\right).\label{eq:constraint1}
\end{equation}
To obtain an expression for $DT/Dt$, we apply differentiate the enthalpy, $h=h(P,T,w_k)$, 
along particle paths to obtain
\begin{equation}
\frac{DT}{Dt} = \frac{1}{c_p}\left(\frac{Dh}{Dt} - h_P\frac{\partial P_0}{\partial t} - \sum_kh_{w_k}\frac{Dw_k}{Dt}\right).\label{eq:DTDt1}
\end{equation}
Note that we have also replaced $DP/Dt$ with 
$\partial P_0/\partial t$ in this equation.  From equation (\ref{eq:mass}) and 
continuity we have,
\begin{equation}
\frac{Dw_k}{Dt} = \frac{1}{\rho}\nabla\cdot\Fb_k,\label{eq:DwDt}
\end{equation}
and from equation (\ref{eq:enthalpy}) we have,
\begin{equation}
\frac{Dh}{Dt} = \frac{1}{\rho}\left(\frac{\partial P_0}{\partial t} + \nabla\cdot\Qb\right).\label{eq:DhDt}
\end{equation}
Combining (\ref{eq:DTDt1}), (\ref{eq:DwDt}), and (\ref{eq:DhDt}) we have
\begin{equation}
\frac{DT}{Dt} = \frac{1}{\rho c_p}\left[\left(1 - \rho h_P\right)\frac{\partial P_0}{\partial t} + \nabla\cdot\Qb - \sum_kh_{w_k}\nabla\cdot\Fb_k\right].\label{eq:DTDt}
\end{equation}
Combining equations (\ref{eq:constraint1}) and (\ref{eq:DTDt}) gives
\footnote{These are the same equations in MAESTRO Paper III, Almgren et al., ApJ, 2008, except that paper includes a convective derivative of $P_0$ due to stratification}
\begin{equation}
\nabla\cdot\vb + \alpha\frac{\partial P_0}{\partial t} = \sum_k \beta_k \nabla\cdot\Fb_k + \gamma \nabla\cdot\Qb \equiv S,\label{eq:constraint}
\end{equation}
\begin{equation}
\alpha = -\left[\frac{(1-\rho h_p)P_T - \rho c_p}{\rho^2 c_p P_\rho}\right], \qquad
\beta_k = \frac{P_{w_k}}{\rho^2 P_\rho} - \frac{P_T h_{w_k}}{\rho^2 P_\rho c_p}, \qquad
\gamma = \frac{P_T}{\rho^2 P_\rho c_P}.
\end{equation}

\section{Maintaining the Equation of State}
There are two issues with maintaining the equation of state.  First, equation 
(\ref{eq:constraint}) represents a linearized, tangent-plane approximation to the 
velocity field required to advect the thermodynamic variables so that they are 
thermodynamically consistent with $P_0$.  For a general, non-linear EOS, the 
variables will fall out of thermodynamic equilibrium, requiring an iterative volume 
discrepancy approach to correct the drift.

Second, for closed system (e.g., solid wall boundary conditions on all sides), if one 
assumes that $P_0$ is constant, then equation (\ref{eq:constraint}) is only solvable 
if $S$ integrates to zero over the entire domain, i.e.,
\begin{equation}
S_{\rm int} \equiv \int_\Omega S d\Omega = 0.
\end{equation}
Physically, if $S_{\rm int} \ne 0$, then $P_0$ will adjust accordingly.  How to deal with
this numerically is an open problem.  One possibility is to split up $S$ into an average
and a perturbational component,
\begin{equation}
S = \bar{S} + \delta S,
\end{equation}
where $\bar{S}$ is the average value over the domain, and $\delta S$ are local 
perturbations off the average.  By definition, $(\delta S)_{\rm int}=0$ and we can
simultaneously solve $\nabla\cdot\vb = \delta S$ and a pressure update equation,
\begin{equation}
\alpha\frac{\partial P_0}{\partial t} = \bar{S}.
\end{equation}
After the thermodynamic variable update, we need to correct for the fact that the 
thermodynamic variables will not in general be thermodynamically consistent with the 
updated $P_0$.  While the argument above still applies for why the variables drift, there 
is now another effect in play in that $\delta S$ is linearized about the original $P_0$.  
In some sense we are chasing a moving target $P_0$, but it is our hope that a volume 
discrepancy type iteration will bring the thermodynamic variables into equilibrium in 
a small number of iterations.

\section{Algorithm Summary}
An inertial algorithm might look like this.\\ \\
Initialization:\\
\begin{enumerate}
\item Compute $\Fb$ and $\Qb$ at the initial time and project.
\item Loop over the following:
\begin{enumerate}
\item Predictor forward Euler step for thermodynamic variables.
\item Advance $P_0$.
\item If satisfied with thermodynamic drift, exit this loop.
\item Determine the volume discrepancy correction and redefine $S$.
\item Project to obtain an updated velocity field.
\end{enumerate}
\item Proceed to Corrector.\\
\end{enumerate}
Corrector:\\
\begin{enumerate}
\item Compute preliminary time-advanced $\Fb$ and $\Qb$.
\item Loop over the following
\begin{enumerate}
\item GMRES solve for preliminary time-advanced velocity.
\item Trapezoidal scalar corrector step for thermodynamic variables.
I'm not sure how the volume discrepancy correction from the predictor comes into play here.
\item Advance $P_0$.
\item If satisfied with thermodynamic drift, exit this loop.
\item Determine the volume discrepancy correction and redefine $S$.  The weighting of the
correction may have to change since this velocity field is used in a trapezoidal corrector.
I'm also not sure if/how to relate this correction to the predictor correction.
\end{enumerate}
\item Proceed to New-Time Predictor.\\
\end{enumerate}
New-Time Predictor:\\
\begin{enumerate}
\item Compute updated time-advanced $\Fb$ and $\Qb$.
\item Loop over the following:
\begin{enumerate}
\item GMRES solve for updated time-advanced velocity.
\item Predictor forward Euler step for thermodynamic variables.
\item Advance $P_0$.
\item If satisfied with thermodynamic drift, exit this loop.
\item Determine the volume discrepancy correction and redefine $S$.
\end{enumerate}
\item Return to Corrector.
\end{enumerate}

\subsection{Volume Discrepancy}
No matter what kind of temporal integrator you are using, any time you compute a
complete set of time-advance thermodynamic variables (whether they be preliminary
or final), they will not be consistent with the ambient pressure.  The strategy
is to wrap the entire time-advancement routine in an iterative loop that uses
volume discrepancy corrections to drive the time-advanced state onto the
equation of state.  We assume isentropic expansion/contraction of fluid elements
and relate this back to the pressure we wish to obtain.

We add a correction term that looks something like
\begin{equation}
\nabla\cdot\vb = S + \frac{1}{\rho P_\rho}\left(\frac{P_{\rm eos}^{\rm new} - P_0^{\rm new}}{\Delta t} + \vb\cdot\nabla P\right) \equiv S + S_{\rm corr}.
\end{equation}
Normally we omit the convective part, as it doesn't seem to have much effect with or
without it.  Since $S_{\rm corr}$ doesn't necessarily sum to zero, so in the 
corrector solve we have to split the update again,
\begin{equation}
\nabla\cdot\vb = \delta S + \delta S_{\rm corr},
\end{equation}
\begin{equation}
\alpha\frac{\partial P_0}{\partial t} = \bar{S} + \bar{S}_{\rm corr}.
\end{equation}
There needs to be some kind of iteration to make sure $P_{\rm eos}$ is sufficiently
close to $P_0^{\rm new}$.

\section{Algorithm Details}
Assume we are given initial $\rho_i$ and $h$ that are thermodynamically consistent with 
$P_0$.  If they are not, compute e.g., $h = h(\rho,P_0,w_i)$.  Initialize the velocity
to zero.\\ \\
{\bf Initialization.}\\ \\
During the initialization step we wish to compute a thermodynamically consistent
velocity field and pressure update for use in the subsequent time-advancement step.
For time step $n=0$, first set the initial volume discrepancy correction to zero, 
$S_{\rm corr}^{n,1} = 0$, as well as the individual terms in the decomposition,
$S_{\rm corr}^{n,1} = \bar{S}_{\rm corr}^{n,1} + \delta S_{\rm corr}^{n,1}$.
Then, loop over the following beginning with $k=1$:\\
\begin{itemize}
\item {\bf Step 0a:} Compute a pressure update:
\begin{equation}
P_0^{*,n+1,k} = P_0^n + \frac{\Delta t(\bar{S}^n + \bar{S}_{\rm corr}^{n,k})}{\alpha^n}
\end{equation}
\item {\bf Step 0b:} Compute the velocity field using
\begin{equation}
\nabla\cdot\vb^{n,k} = \delta S^n + \delta S_{\rm corr}^{n,k}
\end{equation}
\item {\bf Step 0c:} Advect-diffuse the thermodynamic variables using explicit mass 
diffusion and implicit energy diffusion:
\begin{eqnarray}
\rho_i^{*,n+1,k} &=& \rho_i^n + \Delta t\left[-\nabla\cdot(\rho_i^n\vb^{n,k}) + \Fb^n\right], \\
(\rho h)^{*,n+1,k} &=& (\rho h)^n + \Delta t\left[-\nabla\cdot(\rho h\vb) + \frac{\bar{S}^n}{\alpha^n} + \frac{1}{2}(\Qb^n + \Qb^{*,n+1,k})\right].
\end{eqnarray}
\item {\bf Step 0d:} If the thermodynamic drift is unacceptable, compute a volume 
discrepancy correction,
\begin{equation}
S_{\rm corr}^{n,k+1} = S_{\rm corr}^{n,k} + \frac{1}{\rho P_\rho}\left(\frac{P_{\rm EOS}^{*,n+1,k} - P_0^{*,n+1,k}}{\Delta t}\right)
\end{equation}
and return to {\bf Step 0a}.  Otherwise, define 
$(\rho_i,h)^{*,n+1} = (\rho_i,h)^{*,n+1,k}$ and $\vb^n = \vb^{n,k}$ and proceed to 
the {\bf Corrector Step}.\\
\end{itemize}
{\bf Corrector Step.}\\ \\
Begin by computing time-advanced fluxes, $\Fb^{*,n+1}, \Qb^{*,n+1}$ and loop over
the following in order to obtain a time-advanced velocity that will ultimately
allow for thermodynamic balance.  We set the volume discrepancy correction to zero,
$S_{\rm corr}^{*,n+1,1} = 0$, as well as the individual terms in the decomposition
$S_{\rm corr}^{*,n+1,1} = \bar{S}_{\rm corr}^{*,n+1,1} + \delta S_{\rm corr}^{*,n+1,1}$
Then, loop over the following beginning with $k=1$:\\
\begin{itemize}
\item {\bf Step 1a:} Pressure update.
\begin{equation}
P_0^{*,n+1,k} = P_0^n + \frac{\Delta t}{2}\frac{(\bar{S}^n + \bar{S}_{\rm corr}^{n,k})}{\alpha^n} + \frac{\Delta t}{2}\frac{(\bar{S}^{*,n+1} + \bar{S}_{\rm corr}^{*,n+1,k})}{\alpha^{*,n+1}}
\end{equation}
\item {\bf Step 1b:} GMRES solve.
\begin{equation}
\frac{\rho^{*,n+1}\vb^{*,n+1} - \rho^n\vb^n}{\Delta t} + \nabla\pi^{*,n+1} = \nabla\cdot(-\rho\vb\vb)^n + \frac{1}{2}\nabla\cdot(\taub^n + \taub^{*,n+1})
\end{equation}
\begin{equation}
\nabla\cdot\vb^{*,n+1} = \delta S^{*,n+1} + \delta S_{\rm corr}^{*,n+1,k}
\end{equation}
\item {\bf Step 1c:} Trapezoidal scalar corrector.
\begin{eqnarray}
\rho_i^{n+1,k} &=& \rho_i^n + \frac{\Delta t}{2}\left[-\nabla\cdot(\rho_i^n\vb^{n,k}) -\nabla\cdot(\rho_i^{*,n+1}\vb^{*,n+1}) + \Fb^n + \Fb^{*,n+1}\right],\nonumber \\
\\
(\rho h)^{n+1,k} &=& (\rho h)^n + \frac{\Delta t}{2}\left[-\nabla\cdot(\rho h\vb)^n -\nabla\cdot(\rho h\vb)^{*,n+1} + \frac{\bar{S}^n}{\alpha^n} + \frac{\bar{S}^{*,n+1}}{\alpha^{*,n+1}} + \Qb^n + \Qb^{n+1,k}\right].\nonumber\\
\end{eqnarray}
\item {\bf Step 1d:} Check thermodynamics and exit if satisfied.  The weighting of the
correction may have to change since this velocity field is used in a trapezoidal corrector.
I'm also not sure if/how to relate this correction to the predictor correction.
\end{itemize}


\end{document}
