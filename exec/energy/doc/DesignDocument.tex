\documentclass[final]{siamltex}

% for red MarginPars
\usepackage{color}

% for \boldsymbol
\usepackage{amsmath}

% for \mathfrak
\usepackage{amsfonts}

% total number of floats allowed on a page
\setcounter{totalnumber}{100}

% float page fractions
\renewcommand{\topfraction}{0.9}
\renewcommand{\bottomfraction}{0.9}
\renewcommand{\textfraction}{0.2}

% MarginPar
\setlength{\marginparwidth}{0.75in}
\newcommand{\MarginPar}[1]{\marginpar{\vskip-\baselineskip\raggedright\tiny\sffamily\hrule\smallskip{\color{red}#1}\par\smallskip\hrule}}

% for non-stacked fractions
\newcommand{\sfrac}[2]{\mathchoice
  {\kern0em\raise.5ex\hbox{\the\scriptfont0 #1}\kern-.15em/
   \kern-.15em\lower.25ex\hbox{\the\scriptfont0 #2}}
  {\kern0em\raise.5ex\hbox{\the\scriptfont0 #1}\kern-.15em/
   \kern-.15em\lower.25ex\hbox{\the\scriptfont0 #2}}
  {\kern0em\raise.5ex\hbox{\the\scriptscriptfont0 #1}\kern-.2em/
   \kern-.15em\lower.25ex\hbox{\the\scriptscriptfont0 #2}}
  {#1\!/#2}}

\def\1b {{\bf 1}}
\def\bb {{\bf b}}
\def\Fb {{\bf F}}
\def\gb {{\bf g}}
\def\mb {{\bf m}}
\def\Qb {{\bf Q}}
\def\vb {{\bf v}}
\def\wb {{\bf w}}
\def\Wb {{\bf W}}
\def\xb {{\bf x}}

\def\chib   {\boldsymbol{\chi}}
\def\deltab {\boldsymbol{\delta}}
\def\Gammab {\boldsymbol{\Gamma}}
\def\phib   {\boldsymbol{\phi}}
\def\Psib   {\boldsymbol{\Psi}}
\def\Sigmab {\boldsymbol{\Sigma}}
\def\taub   {\boldsymbol{\tau}}
\def\zetab  {\boldsymbol{\zeta}}

\def\Hext {H_{\rm ext}}

\def\half   {\frac{1}{2}}
\def\myhalf {\sfrac{1}{2}}

\begin{document}

%==========================================================================
% Title
%==========================================================================
\title{A New Low Mach Number Hydrodynamics Algorithm}

\author{A. Nonaka\footnotemark[1],
        J. B. Bell\footnotemark[1], and
        A. Donev\footnotemark[2]}

\renewcommand{\thefootnote}{\fnsymbol{footnote}}

\footnotetext[1]{Center for Computational Sciences and Engineering,
                 Lawrence Berkeley National Laboratory, 
                 Berkeley, CA 94720, USA}
\footnotetext[2]{Courant Institute of Mathematical Sciences,
                 New York University,
                 New York, NY 10012, USA}

\maketitle

\begin{abstract}
We present a projection method for modeling unsteady, low Mach number multicomponent
flow for a generalized equation of state.
Due to the linearization of the equation of state used when deriving
constraint manifolds, previously it has
not been possible to simultaneously conserve mass and energy while remaining in 
thermodynamic equilibrium with the spatially-constant ambient pressure.
Our method is fully conservative and introduces a new iterative scheme that 
corrects thermodynamic drift to a desired tolerance.  
We use a staggered grid formulation for momentum and a GMRES algorithm to
solve the viscous-projection Stokes system without splitting artifacts.
For confined regions, our iterative scheme also allows for thermodynamically 
consistent evolution in the presence of time-varying ambient pressure.
Our method provides a development path for low Mach number reacting flow and 
stochastic flow described by fluctuating hydrodynamics equations.
We present results for mixtures of ideal noble gases with external heating
as well as high-pressure and temperature ternary mixture using the 
Soave-Redlich-Kwong equation of state.
\end{abstract}

\section{Introduction}
 This work is based low Mach number formulations proposed in
\cite{RehmBaum,MajdaSethian}, but here our equation set is suitable for a
generalized equation of state, is fully conservative, and includes an iterative
scheme to correct thermodynamic drift.
The basic model is a system of advection-diffusion equations
subject to a constraint on the velocity field that arises by differentiating
the equation of state along particle paths under to the assumption
that the thermodynamic pressure remain spatially constant.  Physically, this
manifests itself as an instananeous acoustic equilibration, thus eliminating
pressure waves.  Numerically, this allows for an advective, rather than acoustic,
time step constraint.

Our numerical framework is the staggered grid, finite volume approach first
introduced in \cite{LowMachExplicit,LowMachImplicit,LowMachMulti} for
multicomponent miscible liquids with incompressible components.
We treat the viscosity implicitly without splitting the pressure
update, relying on a recently-developed variable-coefficient multigrid-preconditioned
Stokes solver \cite{StokesPreconditioners}.
For now, mass diffusion in treated explicitly making our
approach suitable for large Schmidt numbers.  Implicit mass diffusion is a subject
of future work.

\section{Equations}
The governing conservation equations for momentum, mass, and energy for our system are:
\begin{eqnarray}
\frac{\partial\rho\vb}{\partial t} &=& - \nabla\cdot(\rho\vb\vb) - \nabla\pi + \nabla\cdot\taub + \rho\gb,\label{eq:momentum}\\
\frac{\partial\rho_i}{\partial t} &=& -\nabla\cdot(\rho_i\vb) - \nabla\cdot\Fb_i + E_i,\label{eq:mass}\\
\frac{\partial\rho h}{\partial t} &=& -\nabla\cdot(\rho h\vb) + \frac{DP}{Dt} - \nabla\cdot\Qb - \sum_k \nabla\cdot h_k \Fb_k + \rho\Hext,\label{eq:enthalpy}
\end{eqnarray}
where $\rho$ is the density, $\vb$ is the velocity, $\pi$ is the dynamic 
(or perturbational) pressure, $\gb$ is the gravitational vector, $\rho_i=\rho w_i$ is
the density of fluid component $i$ with $\rho = \sum_k\rho_k$ and $w_i$ the 
concentrations (mass fractions), $P$ is the total
pressure, $h$ is the specific enthalpy, $h_k = \partial h/\partial w_k$,
$\Hext$ is an external heating term, $\taub$ is the stress tensor,
$\Fb_i$ are the mass fluxes, $E_i$ are external sources/sinks that sum to zero,
and $\Qb$ is the heat flux.
Starting from the fully compressible equations and expanding in Mach number shows 
that the leading order pressure terms in Mach number must be constant in space. 
Thus, the perturbational pressure $\pi$ is $O(M^2)$.
The low Mach number system is then obtained by ignoring the effect of terms of 
order $M^2$ or higher in pressure  on the thermodynamic state of the system.
The equation of state then becomes a constraint on the evolution; namely,
the evolution of the system is constrained so that
\begin{equation}
P_0(t) = P(\rho,\wb,T),\label{eq:EOS}
\end{equation}
where $T=T(\rho,\wb,h)$ is specified by the equation of state.\\

The viscous stress tensor,
$\taub = \eta[\nabla\vb + (\nabla\vb)^T]$,
ignores the effect of bulk viscosity, which can be aborbed into $\pi$.
The heat flux has the form $\Qb = -\lambda\nabla T$.
The mass fluxes contains contributions from 
compositional and temperature gradients,
\begin{equation}
\Fb = -\rho\Wb\chib\left[\Gammab\nabla\xb + (\phib - \wb)\frac{\nabla P}{n k_B T} + \zetab\frac{\nabla T}{T}\right],
\end{equation}
where $\Wb$ is the diagonal matrix of concentrations,
$\chib$ is the diffusion matrix, $\xb$ are the mole fractions,
$\phib$ are the volume fractions (which for an ideal gas is equivalent to mole fractions),
$k_B$ is Boltzmann's constant,
$T$ is the temperature, and $\zetab$ are thermodiffusion coefficients.  The total
number density is the sum of the partial number densities, $n=\sum_kn_k$ with
$n_i=\rho_i/W_i$ and $W_i$ the molecular mass of species $i$.

Equations (\ref{eq:momentum}), (\ref{eq:mass}), (\ref{eq:enthalpy})
with $DP/Dt$ in the enthalpy equation replaced by $dP_0/dt$, and (\ref{eq:EOS})
form the actual system that we would like to solve.
This corresponds to evolution subject to a constraint.  From the analysis below it 
will become apparent that this system is an index 3 differential algebraic system.
Before looking in detail at the structure of the system, it is, perhaps, helpful to 
consider a physical interpretation of the
system.  The perturbational pressure $\pi$ constrains velocity field so that
the evolution so that the evolution of $h$ and $\rho_k$
preserves a thermodynamic pressure that is a function of $t$ only.
When the system is open, the pressure
equilibrates to the ambient pressure and $P_0$ is independent of time and known.
When the system is closed
$P_0$ represents the global pressure
required force the fluid to occupy the available volume.\\

We also note that $\rho, h, T, P$ are linked by the equation of state in a sense that given
$\wb$ and any two of these variables, the equation of state uniquely specifies the other
two variables.  In particular, we will think of enthalpy as $h = h(P,\wb,T)$ when deriving
the divergence constraint from the equation of state.
We view these relations as defining temperature in terms of enthalpy.

\subsection{Velocity Constraint and Thermodynamic Pressure Update}

Directly attacking the evolution of the constrained system is not tractable.
For DAE's the standard approach to understanding the structure is to differentiate the constraint.
If we start on the constraint and recast the evolution in terms of the derivative of the constraint
being satisfied, the resulting evolution of the system is analytically equivalent.  In the theory
of DAE's the number of times the system needs to be differentiated to recast it as a pure initial
value problem is referred to as the index.
The derivative of the constraint can also be cast as a characterization of the tangent plane to the
constraint manifold but it is not clear what to do with that observation.\\

The key things we want to get a handle on are how the two different elements that go into the
constraint interact to control the evolution and whether they are, in some sense, uniquely specified.\\

We begin by differentiating the 
right-hand side of equation (\ref{eq:EOS}) along particle paths,
\begin{equation}
\frac{DP}{Dt} = P_\rho\frac{D\rho}{Dt} + P_T\frac{DT}{Dt} + \sum_kP_{w_k}\frac{Dw_k}{Dt}.\label{eq:particle paths}
\end{equation}
This statement is always true, but in our low Mach number model we wish to enforce
that the pressure is spatially constant, yet can vary in time.

Next, by noting that by continuity $D\rho/Dt = -\rho\nabla\cdot\vb$, 
we can rewrite 
(\ref{eq:particle paths}) as
\begin{equation}
\nabla\cdot\vb = \frac{1}{\rho P_\rho}\left(-\frac{dP_o}{dt} + P_T\frac{DT}{Dt} + \sum_kP_{w_k}\frac{Dw_k}{Dt}\right).\label{eq:constraint1}
\end{equation}
To obtain an expression for $DT/Dt$, we differentiate the enthalpy, $h=h(T,P,\wb)$, 
along particle paths to obtain
\begin{equation}
\frac{Dh}{Dt} = 
c_p \frac{DT}{Dt} 
+ h_P\frac{DP}{Dt} + \sum_k h_{k}\frac{Dw_k}{Dt}  \;\;  ,
\end{equation}
where $c_p = \partial h/\partial T$.  This gives
\begin{equation}
\frac{DT}{Dt} = \frac{1}{c_p}\left(\frac{Dh}{Dt} - h_P\frac{dP}{Dt} - \sum_kh_{k}\frac{Dw_k}{Dt}\right).\label{eq:DTDt1}
\end{equation}
From equation (\ref{eq:mass}) and continuity we have,
\begin{equation}
\frac{Dw_k}{Dt} = \frac{1}{\rho}\left(-\nabla\cdot\Fb_k + E_k\right),\label{eq:DwDt}
\end{equation}
and from equation (\ref{eq:enthalpy}) we have,
\begin{equation}
\frac{Dh}{Dt} = \frac{1}{\rho}\left(\frac{DP}{Dt} - \nabla\cdot\Qb - \sum_k\nabla\cdot h_k\Fb_k + \rho\Hext\right).\label{eq:DhDt}
\end{equation}
Combining (\ref{eq:DTDt1}), (\ref{eq:DwDt}), and (\ref{eq:DhDt}) we have
\begin{equation}
\frac{DT}{Dt} = \frac{1}{\rho c_p}\left[\left(1 - \rho h_P\right)\frac{DP}{Dt} - \nabla\cdot\Qb - \sum_k\nabla\cdot h_k\Fb_k - \sum_kh_k(-\nabla\cdot\Fb_k + E_k) + \rho\Hext\right].\label{eq:DTDt}
\end{equation}
Combining equations (\ref{eq:constraint1}) and (\ref{eq:DTDt})
and using $P_{w_k}/P_\rho = -\rho_{w_k}$ and $P_T/P_\rho = -\rho_T$ gives
\begin{equation}
\nabla\cdot\vb + \alpha\frac{DP}{Dt} = \sum_k \frac{1}{\rho}\beta_k \nabla\cdot\Fb_k + \frac{1}{\rho c_p}\theta\left[\nabla\cdot\Qb + \sum_k\left(\nabla\cdot h_k\Fb_k - h_k\nabla\cdot\Fb_k + h_kE_k\right) - \rho\Hext\right] \equiv S,\label{eq:constraint}
\end{equation}
\begin{equation}
\alpha = \frac{1}{\rho P_\rho} + \frac{(1-\rho h_p)}{\rho c_p}\theta, \quad
\beta_k = \frac{1}{\rho}\rho_{w_k}, \quad
\theta = \frac{1}{\rho}\rho_T.\label{eq:thermo deriv}
\end{equation}
Note that $\beta_k$ and $\theta$
are the solutal and thermal expansion coefficients, respectively.

In \cite{MAESTROI}, it was shown that
\begin{equation}
\alpha = \frac{1}{\Gamma_1 P},
\end{equation}
where $\Gamma_1 = \left(d\ln{p}/d\ln{\rho}\right)_{\rm ad}$ is the first 
adiabatic exponent.  This is related to the ratio of specific heats, 
$\gamma$, via
\begin{equation}
\gamma = \frac{c_p}{c_v} = \frac{\Gamma_1}{\chi_\rho},
\end{equation}
where
\begin{equation}
\chi_\rho \equiv \left(\frac{\partial\ln{P}}{\partial\ln{\rho}}\right)_T = \frac{\rho}{P}P_\rho
\end{equation}
is the ``density exponent in the pressure equation of state.''
Note that for ideal gases, $\chi_\rho=1, \Gamma_1=\gamma$, and therefore
\begin{equation}
\alpha = \frac{1}{\gamma\rho P_\rho} = \frac{1}{\gamma P}
\end{equation}

In our low Mach number model we replace $DP/Dt$ with $dP_0/ dt$ in the 
constraint (\ref{eq:constraint}) and enthalpy (\ref{eq:enthalpy}) equations.
(Since we are currently assuming $P_0$ is constant in space,
we are not allowing for pressure stratification due to gravity.)
%\begin{equation}
%\frac{dP_0}{dt} = P_\rho\frac{D\rho}{Dt} + P_T\frac{DT}{Dt} + \sum_kP_{w_k}\frac{Dw_k}{Dt}.\label{eq:particle paths2}
%\end{equation}
%As noted above we also make the
%substitution $DP/Dt = dP_0/ d t$ in the enthalpy equation.

\subsection{Comparison to LMC and MAESTRO}
It is worth comparing this formulation to other low Mach number constraints.
This model is equivalent to equation (7) in \cite{DayBell:2000},
but with three exceptions.  First,
we are not accounting for reactions, so $\dot\omega=0$.  Second, here we have 
{\it not} made the substitution, 
$-\nabla\cdot h_k\Fb_k + h_k\nabla\cdot\Fb_k = -\Fb\cdot\nabla h_k$, so the coefficients
$\beta_k$ and $\theta$ are slightly different (although the overall expression is
analytically equivalent).  Third, the time-derivative of $P_0$ is no longer zero,
so we gain an additional term.  Notably, the coefficient in front of the pressure 
derivative matches that in the MAESTRO algorithm \cite{MAESTROIII},
except that paper includes a convective derivative of $P_0$ due to stratification.

\subsection{Confined Domains}
The unknowns in equation (\ref{eq:constraint}) are $\vb$ and $dP_0/d t$.
For open domains where $\vb$ is allowed to be non-zero on one or more walls $P_0$
is constant and the constraint equation reduces to
\begin{equation}
\nabla\cdot\vb = S.
\end{equation}
This is a generalization of standard low Mach number approaches (at least our version) 
and fits into
a framework we have dealt with before.\\

For confined domains with no-flow walls the constraint is somewhat more complex.
Equation (\ref{eq:constraint}) then represents the derivative of the original
constraint.  We now want to see how
decomposition of this equation into mean and fluctuating (not in the stochastic sense)
components allows us to recover the evolution of $P_0$ and $\pi$.\\

For a closed system, $\nabla \cdot \vb = g$ only has a solution if $\int_\Omega g~d\Omega = 0$
We can use this fact to simultaneously solve for the unknowns
$\vb$ and $dP_0/dt$.
To do this, we split up $S$ into an average and a perturbational component,
\begin{equation}
S = \bar{S} + \delta S.
\end{equation}
We also
split $\alpha$ into an average and a perturbational component,
\begin{equation}
\alpha = \bar{\alpha} + \delta\alpha.
\end{equation}
By definition,
\begin{equation}
\int_{\Omega} \delta S ~\partial\Omega = \int_{\Omega} \delta \alpha ~\partial\Omega = 0.\label{eq:zero int}
\end{equation}
So equation (\ref{eq:constraint}) can be rewritten as,
\begin{equation}
\nabla\cdot\vb + \bar{\alpha}\frac{dP_0}{d t} 
= \bar{S} + \delta S -
\delta\alpha\frac{dP_0}{d t}.\label{eq:constraint2}
\end{equation}
Since $\nabla \cdot \vb$ must integrate to zero and $P_0$ must be only a function of time,
equation (\ref{eq:constraint2}) can be uniquely decomposed to give
\begin{equation}
\frac{dP_0}{dt} = \frac{\bar{S}}{\bar{\alpha}}.\label{eq:P0}
\end{equation}
and
\begin{eqnarray}
\nabla\cdot\vb &=& \delta S - \delta\alpha\frac{dP_0}{dt} \nonumber\\
&=& \delta S - \delta\alpha\frac{\bar{S}}{\bar{\alpha}}.\label{eq:constraint3}
\end{eqnarray}
Thus, the original system is analytically equivalent to solving the system (\ref{eq:momentum}), (\ref{eq:mass}), and (\ref{eq:enthalpy})
subject to (\ref{eq:P0}) and (\ref{eq:constraint3}).

\section{Equation of State Drift}
Consider equation (\ref{eq:particle paths}), but with
$DP/Dt$ replaced with $\partial P_0/\partial t$,
where we analytically enforce that
thermodynamic pressure will evolve to be consistent with the evolution of $P_0$,
\begin{equation}
\frac{dP_0}{dt} = P_\rho\frac{D\rho}{Dt} + P_T\frac{DT}{Dt} + \sum_kP_{w_k}\frac{Dw_k}{Dt}.
\end{equation}
This equation was shown to be analytically equivalent to equation (\ref{eq:constraint}):
\begin{equation}
\nabla\cdot\vb + \alpha\frac{dP_0}{dt} = S.
\end{equation}

This divergence constraint represents a linearized approximation for the velocity 
field required to ensure that the thermodynamic variables remain thermodnamically
consistent with $P_0$.  Since in general the EOS is non-linear, the updated 
thermodynamic variables will not
satisfy this condition.  We propose an iterative strategy based on
volume discrepancy schemes used
in other conservative, finite-volume low Mach number algorithms
\cite{Pember:1998,XRB}, that
uses a modified divergence constraint designed to drive the updated 
thermodynamic variables closer to thermodynamic equilibrium with $P_0$ with each iteration.
For confined domains, we also iteratively update $P_0$ as well.  To our knowledge,
such approaches have only been used for open domains.\\

After computing a velocity field with the GMRES solver, and updating the thermodynamic
variables with advective and diffusive fluxes, the EOS will no longer be satisfied.
The amount by which the EOS is not satisfied can be easily quantified by defining
the drift, $\Delta P = P_{\rm EOS}^{n+1} - P_0^{n+1}$ with $P_{\rm EOS} = P(\rho,\wb,T)$.\\

If we were able to ``redo'' the velocity computation, the rate at which we want the 
pressure to change in each cell is no longer given
by simply $dP_0/dt$, but is now given by:
\begin{equation}
\frac{DP}{Dt} = \frac{dP_0}{dt} - \frac{\Delta P}{\Delta t}.\label{eq:DPDt}
\end{equation}
We include the $\Delta P/\Delta t$ ``volume discrepancy'' correction term here because 
now we have a sense for how numerical truncation error causes the local thermodynamic pressure in 
each cell to drift without such a correction.  Physically, by including this term, we are enforce 
an additional amount of expansion or contraction within each cell in order to modify 
the pressure.  

By substituting (\ref{eq:DPDt}) into (\ref{eq:particle paths}),
and following the same derivation, we end up with,
\begin{equation}
\nabla\cdot\vb + \alpha\frac{dP_0}{dt} = S + \underbrace{\frac{1}{\rho P_\rho}\frac{\Delta P}{\Delta t}}_{S_{\rm corr}}.
\end{equation}
Similar to above, in order for this sytem to be solvable we split the update into
equations for $P_0$ and $\vb$,
\begin{equation}
\frac{dP_0}{dt} = \frac{\bar{S} + \bar{S}_{\rm corr}}{\bar{\alpha}}
\end{equation}
\begin{equation}
\nabla\cdot\vb = \delta S + \delta S_{\rm corr} - \delta\alpha\left(\frac{\bar{S} + \bar{S}_{\rm corr}}{\bar{\alpha}}\right).
\end{equation}
At the beginning of each time step, we initialize $S_{\rm corr}$ based on the current drift of the data at $t^n$.
As we iterate, we increment (rather than reset) $\delta S_{\rm corr}$ based on how much 
the current solution drifts using $\delta S_{\rm corr}$ the previous iteration.

\section{Heat Flux Integration Strategy}
Recall the form of the enthalpy equation we are interested in:
\begin{equation}
\frac{\partial(\rho h)}{\partial t} = -\nabla\cdot(\rho h\vb) + \frac{dP_0}{dt} + \nabla\cdot\lambda\nabla T - \sum_k\nabla\cdot h_k\Fb_k + \rho\Hext.
\end{equation}
In our time-advancement scheme, we want to handle the thermal diffusion semi-implicitly.
The details are given in Section {\ref{Sec:Time-Advancement Strategy}} but for ease
of exposition we outline the heat flux strategy here.
We use an iterative method that
linearizes the relationship between how $h$ changes with respect to changes in $T$.
Note that we have already updated the densities, but unfortunately the time-advanced
mass fluxes also depend on the enthalpy.  In the full time-advancement algorithm,
the idea shall be to iterate these fluxes as well, with an initial guess coming from
an earlier step in the algorithm.  The notation here is just used to give a general
idea of the iterative scheme.  Also, the details of the treatment for 
$dP_0/dt$ will be given below.\\ \\
Let's assume we want to form an iterative update to $h$ using
\begin{eqnarray}
\frac{\rho^{n+1}h^{n+1,l+1} - (\rho h)^n}{\Delta t} &=& -\nabla\cdot(\rho h\vb) + \frac{dP_0}{dt}\nonumber\\
&&+ \half\nabla\cdot\lambda^n\nabla T^n + \half\nabla\cdot\lambda^{n+1,l}\nabla T^{n+1,l+1}\nonumber\\
&& - \half\sum_k\nabla\cdot h_k^n \Fb_k^n - \half\sum_k\nabla\cdot h_k^{n+1,l}\Fb_k^{n+1,l}\nonumber\\
&& + \half(\rho\Hext)^n + \half(\rho\Hext)^{n+1,l}.\label{eq:temp form}
\end{eqnarray}
\\
We start with initial guesses $(T,h)^{n+1,l=1} = (T,h)^n$.\\ \\
We say that
\begin{equation}
T^{n+1,l+1} = T^{n+1,l} + \delta T\label{eq:delta T}
\end{equation}
Linearization gives
\begin{equation}
h^{n+1,l+1} = h^{n+1,l} + c_p^{n+1,l}\delta T\label{eq:delta h}
\end{equation}
Substituting (\ref{eq:delta T}) and (\ref{eq:delta h}) into (\ref{eq:temp form}) gives:
\begin{eqnarray}
\frac{\rho^{n+1}(h^{n+1,l} + c_p^{n+1,l}\delta T) - (\rho h)^n}{\Delta t} &=& -\nabla\cdot(\rho h\vb) + \frac{dP_0}{dt}\nonumber\\
&&\hspace{-0.5in}+ \half\nabla\cdot\lambda^n\nabla T^n + \half\nabla\cdot\lambda^{n+1,l}\nabla(T^{n+1,l} + \delta T)\nonumber\\
&&\hspace{-0.5in} - \half\sum_k\nabla\cdot h_k^n \Fb_k^n - \half\sum_k\nabla\cdot h_k^{n+1,l} \Fb_k^{n+1,l}\nonumber\\
&&\hspace{-0.5in} + \half(\rho\Hext)^n + \half(\rho\Hext)^{n+1,l}.
\end{eqnarray}
We solve this implicitly for $\delta T$, then compute $T^{n+1,l+1}$
using (\ref{eq:delta T}) and compute $h^{n+1,l+1}$ using the EOS.

\section{Overview of Time Stepping Strategy}
Our unknowns are $(\rho_i,h,\vb,\pi,P_0)^{n+1}$, and $S_{\rm corr}$.  A quantity can
be split into average and perturbational components using
\begin{equation}
\phi = \bar\phi + \delta\phi, \quad \int_\Omega \delta\phi~d\Omega = 0,
\end{equation}
where $\bar\phi$ is constant in space.\\ \\
The discretizations we would like to solve are as follows.  To compute $\rho_i^{n+1}$ we use
\begin{equation}
\frac{\rho_i^{n+1} - \rho_i^n}{\Delta t} = \half\left[-\nabla\cdot(\rho_i\vb)^n -\nabla\cdot(\rho_i\vb)^{n+1} - \nabla\cdot\Fb_i^n - \nabla\cdot\Fb_i^{n+1} + E_i^n + E_i^{n+1}\right].
\end{equation}
To compute $h^{n+1}$, we use
\begin{eqnarray}
\frac{(\rho h)^{n+1} - (\rho h)^n}{\Delta t} &=& -\half\nabla\cdot(\rho h\vb)^n - \half\nabla\cdot(\rho h\vb)^{n+1} + \frac{\Delta P}{\Delta t}\nonumber\\
&&+ \half\nabla\cdot\lambda^n\nabla T^n + \half\nabla\cdot\lambda^{n+1}\nabla T^{n+1}\nonumber\\
&&- \half\sum_k\nabla\cdot h_k^n\Fb_k^n - \half\sum_k\nabla\cdot h_k^{n+1}\Fb_k^{n+1}\nonumber\\
&&+ \half(\rho\Hext)^n + \half(\rho\Hext)^{n+1},
\end{eqnarray}
To compute $\vb^{n+1}$ and the Lagrance multiplier $\pi^{n+\myhalf}$ we use
\begin{equation}
\frac{(\rho\vb)^{n+1} - (\rho\vb)^n}{\Delta t} + \nabla\pi^{n+\myhalf} = \half\left[-\nabla\cdot(\rho\vb\vb)^n - \nabla\cdot(\rho\vb\vb)^{n+1}\right] + \frac{1}{2}\nabla\cdot(\taub^n + \taub^{n+1}) + \frac{1}{2}(\rho^n + \rho^{n+1})\gb,
\end{equation}
\begin{equation}
\nabla\cdot\vb^{n+1} = \delta S^{n+1} + \delta S_{\rm corr} - \delta\alpha^{n+1}\left(\frac{\bar{S}^{n+1} + \bar{S}_{\rm corr}}{\bar{\alpha}^{n+1}}\right).\label{eq:div_constraint}
\end{equation}
To compute $P_0^{n+1}$ we use
\begin{equation}
\frac{P_0^{n+1} - P_0^n}{\Delta t} \equiv \frac{\Delta P}{\Delta t} = \frac{\bar S + \bar S_{\rm corr}}{\bar\alpha},\label{eq:P0_update}
\end{equation}
We seek $S_{\rm corr}$ that causes the evolution of the thermodynamic variables to stay on the equation of 
state.  For gases, this means,
\begin{equation}
P(\rho,w_k,T)^{n+1} = P_0^{n+1},
\end{equation}
and for our liquid model,
\begin{equation}
\sum_i\frac{\rho_i}{\bar\rho_i(T,P)} = 1.
\end{equation}
We don't want to solve this implicitly, so we use an iterative approach that lags some
of the time-advanced quantities while updating $S_{\rm corr}$.

Note that for liquids, we are taking the limit that $\Gamma_1 \rightarrow 0$, or equivalently,
$\alpha\rightarrow\infty$.  Thus, $P_0$ is constant, and therefore the right-hand-side of equation 
(\ref{eq:P0_update}) is zero, and the term proportional to $\delta\alpha$ in 
(\ref{eq:div_constraint}) % is zero.

% The form of the system given above can fit into the framework of low Mach number algorithms we have
% developed based on the generalized Stokes solver. 
% The additional issue that arises is that since the core numerical algorithm is based on the derivative
% of the constraint, we can numerically drift off of the actual algebraic constraint. Thus, we augment the
% basic algorithm with iterative 
% ``volume discrepancy'' scheme described below to solve 
% (\ref{eq:momentum}), (\ref{eq:mass}), and (\ref{eq:enthalpy})
% while enforcing condition (\ref{eq:EOS}) to a specified tolerance.\\

At the most basic level, the time-stepping strategy is an iterative loop over the following:\\
\begin{itemize}
\item Advance thermodynamic variables $\rho,w_k,h,P_0$.
\item Advance velocity using a Stokes solver.
\item Update $S_{\rm corr}$ so the EOS is more closely maintained in the next iteration.
\end{itemize}

\section{Time-Advancement Strategy}\label{Sec:Time-Advancement Strategy}

\subsection{Initialization}
We are given $\vb^{\rm init}$ and $(\rho,\wb,h,T,P_0,\pi)^0$,
where the thermodynamic variables are in equilibrium
and $\pi^0=0$.  We compute $\vb^0$ by enforcing,
\begin{equation}
\nabla\cdot\vb^0 = \delta S^0 - \delta\alpha^0\frac{\bar{S}^0}{\bar{\alpha}^0} \equiv S_{\rm proj}^0
\end{equation}
This involves solving the ellptic equation,
\begin{equation}
\nabla\cdot\frac{1}{\rho^0}\nabla\Phi = \nabla\cdot\vb^{\rm init} - S_{\rm proj}^0,
\end{equation}
and velocity update,
\begin{equation}
\vb^0 = \vb^{\rm init} - \frac{1}{\rho^0}\nabla\Phi.
\end{equation}

\subsection{Time-Advancement Scheme}
We are given $(\vb,\rho,\wb,h,T,P_0,\pi)^n$.
Note that in general that the thermodynamic variables are not in 
equilibrium.
The time-advancement consists of:\\ \\
{\bf Step 1:}\\ \\
Compute $(S,\alpha,\Fb,\eta,\lambda,h_k)^n$ from $(\rho,\wb,h,P_0,T)^n$.
\MarginPar{The routines in the code take in all these variables and mix the usage.  More details need to be given here.}
Split $(S,\alpha)^n$ into average and perturbational components:
\begin{equation}
S^n = \bar{S}^n + \delta S^n,
\end{equation}
\begin{equation}
\alpha^n = \bar{\alpha}^n + \delta\alpha^n.
\end{equation}
{\bf Step 2:}\\ \\
Compute a volume discrepancy correction:
\begin{equation}
S_{\rm corr} = \frac{1}{(\rho P_\rho)^n}\left(\frac{P(\rho,\wb,T)^n - P_0^n}{\Delta t}\right),
\end{equation}
and split this into average and perturbational components,
\begin{equation}
S_{\rm corr} = \bar{S}_{\rm corr} + \delta S_{\rm corr}.
\end{equation}
Note: it's possible that setting $S_{\rm corr}=0$ here may cause the iterations to converge more
quickly, and that will be easily testable.
Note that if the thermodynamic variables are in equilibrium, then $S_{\rm corr}=0$ regardless.\\ \\
Set $(\vb,\rho,\wb,h,T,P_0)^{n+1,m=1} = (\vb,\rho,\wb,h,T,P_0)^n$, 
and  $(S,\alpha,\Fb,\eta,c_p,\lambda,h_k)^{n+1,m=1} = (S,\alpha,\Fb,\eta,c_p,\lambda,h_k)^n$.\\ \\
{\bf Step 3}\\ \\
Split $(S,\alpha)^{n+1,1}$ into average and perturbational components:
\begin{equation}
S^{n+1,1} = \bar{S}^{n+1,1} + \delta S^{n+1,1},
\end{equation}
\begin{equation}
\alpha^{n+1,1} = \bar{\alpha}^{n+1,1} + \delta\alpha^{n+1,1}.
\end{equation}
Iterate over an outer loop from $m=1$ over {\bf Steps 4-8}:\\ \\
{\bf Step 4}\\ \\
Pressure update:
\begin{equation}
P_0^{n+1,m+1} = P_0^n + \frac{\Delta t}{2}\left(\frac{\bar{S}^n + \bar{S}_{\rm corr}}{\bar{\alpha}^n} + \frac{\bar{S}^{n+1,m} + \bar{S}_{\rm corr}}{\bar{\alpha}^{n+1,m}}\right).
\end{equation}
{\bf Step 5}\\ \\
Recompute densities with trapezoidal-rule advective and diffusive fluxes.
\begin{equation}
\rho_i^{n+1,m+1} = \rho_i^n + \frac{\Delta t}{2}\left[-\nabla\cdot(\rho_i\vb)^n -\nabla\cdot(\rho_i\vb)^{n+1,m} - \nabla\cdot\Fb^n - \nabla\cdot\Fb^{n+1,m} + E^n + E^{n+1,m}\right],
\end{equation}
\begin{equation}
\rho^{n+1,m+1} = \sum_k\rho_k^{n+1,m+1}.
\end{equation}
{\bf Step 6}\\ \\
Recompute enthalpy semi-implicitly.  This involves a linearized iteration 
for $\delta T$.  In this fundamental formulation, the unknowns 
are $h^{n+1,m+1}$ and $T^{n+1,m+1}$:
\begin{eqnarray}
\frac{(\rho h)^{n+1,m+1} - (\rho h)^n}{\Delta t} &=& \half\left[-\nabla\cdot(\rho h\vb)^n -\nabla\cdot(\rho h\vb)^{n+1,m} + \frac{\bar{S}^n + \bar{S}_{\rm corr}}{\bar{\alpha}^n} + \frac{\bar{S}^{n+1,m}+ \bar{S}_{\rm corr}}{\bar{\alpha}^{n+1,m}}\right]\nonumber\\
&&+ \half\nabla\cdot\lambda^n\nabla T^n + \half\nabla\cdot\lambda^{n+1,m+1}\nabla T^{n+1,m+1}\nonumber\\
&&- \half\sum_k\nabla\cdot h_k^n\Fb_k^n - \half\sum_k\nabla\cdot h_k^{n+1,m+1}\Fb_k^{n+1,m}\nonumber\\
&&+ \half(\rho\Hext)^n + \half(\rho\Hext)^{n+1,m+1},
\end{eqnarray}
but since the heat fluxes are expressed in terms of temperature gradients, we
need an iterative approach that linearizes the dependence of $h$ on $T$,
\begin{eqnarray}
\frac{\rho^{n+1,m+1}(h^{n+1,m+1,l} + c_p^{n+1,m+1,l}\delta T) - (\rho h)^n}{\Delta t} &=& \half\left[-\nabla\cdot(\rho h\vb)^n -\nabla\cdot(\rho h\vb)^{n+1,m} + \frac{\bar{S}^n + \bar{S}_{\rm corr}}{\bar{\alpha}^n} + \frac{\bar{S}^{n+1,m}+ \bar{S}_{\rm corr}}{\bar{\alpha}^{n+1,m}}\right]\nonumber\\
&&\hspace{-1in} + \half\nabla\cdot\lambda^n\nabla T^n + \half\nabla\cdot\lambda^{n+1,m+1,l}\nabla(T^{n+1,m+1,l} + \delta T)\nonumber\\
&&\hspace{-1in} - \half\sum_k\nabla\cdot h_k^n \Fb_k^n - \half\sum_k\nabla\cdot h_k^{n+1,m+1,l} \Fb_k^{n+1,m},\nonumber\\
&&\hspace{-1in} + \half(\rho\Hext)^n + \half(\rho\Hext)^{n+1,m+1},
\end{eqnarray}
which can be rearranged to solve for $\delta T$,
\begin{eqnarray}
\left(\frac{\rho^{n+1,m+1}c_p^{n+1,m+1,l}}{\Delta t} - \half\nabla\cdot\lambda^{n+1,m+1,l}\nabla\right)\delta T &=& \frac{(\rho h)^n - \rho^{n+1,m+1}h^{n+1,m+1,l}}{\Delta t}\nonumber\\
&&\hspace{-1.25in} + \half\left[-\nabla\cdot(\rho h\vb)^n - \nabla\cdot(\rho h\vb)^{n+1,m} + \frac{\bar{S}^n + \bar{S}_{\rm corr}^n}{\bar{\alpha}^n} + \frac{\bar{S}^{n+1,m} + \bar{S}_{\rm corr}}{\bar{\alpha}^{n+1,m}}\right]\nonumber\\
&&\hspace{-1.25in} + \half\nabla\cdot\lambda^n\nabla T^n + \half\nabla\cdot\lambda^{n+1,m+1,l}\nabla T^{n+1,m+1,l}\nonumber\\
&&\hspace{-1.25in} - \half\sum_k\nabla\cdot h_k^n \Fb_k^n - \half\sum_k\nabla\cdot h_k^{n+1,m+1,l} \Fb_k^{n+1,m}\nonumber\\
&&\hspace{-1.25in} + \half(\rho\Hext)^n + \half(\rho\Hext)^{n+1,m+1}.\label{eq:deltaT}
\end{eqnarray}
To proceed, we set $(h,T,\lambda,c_p,h_k)^{n+1,m+1,l=1} = (h,T,\lambda,c_p,h_k)^{n+1,m}$
and iterate over an inner loop from $l=1$ over {\bf Steps 6a-c}:\\ \\
{\bf Step 6a}\\ \\
Solve equation (\ref{eq:deltaT}) implicitly for $\delta T$.\\ \\
{\bf Step 6b}\\ \\
Set $T^{n+1,m+1,l+1} = T^{n+1,m+1,l} + \delta T$ and compute 
$h^{n+1,m+1,l+1} = h(\rho^{n+1,m+1},\wb^{n+1,m+1},T^{n+1,m+1,l+1}$).
{\bf Step 6c}\\ \\
If this is the last iteration, set $(T,h)^{n+1,m+1} = (T,h)^{n+1,m+1,l+1}$
and compute $\eta^{n+1,m+1}$ from $(\rho^{n+1,m+1},\wb^{n+1,m+1},T^{n+1,m+1,l+1})$.
Otherwise,
compute $(\lambda,c_p,h_k)^{n+1,m+1,l+1}$ from 
$(\rho^{n+1,m+1},\wb^{n+1,m+1},T^{n+1,m+1,l+1})$,
increment $l$, and return to {\bf Step 6a}.\\ \\
{\bf Step 7}\\ \\
Compute $(S,\alpha)^{n+1,m+1}$ from the ``$n+1,m+1$'' state and split them 
into average and perturbational components:
\begin{equation}
S^{n+1,m+1} = \bar{S}^{n+1,m+1} + \delta S^{n+1,m+1},
\end{equation}
\begin{equation}
\alpha^{n+1,m+1} = \bar{\alpha}^{n+1,m+1} + \delta\alpha^{n+1,m+1},
\end{equation}
Then perform a GMRES solve for $(\vb,\pi)^{n+1,m+1}$.
\begin{eqnarray}
\frac{(\rho\vb)^{n+1,m+1} - (\rho\vb)^n}{\Delta t} + \nabla\pi^{n+1,m+1} &=& \half\left[-\nabla\cdot(\rho\vb\vb)^n - \nabla\cdot(\rho\vb\vb)^{n+1,m}\right] + \frac{1}{2}\nabla\cdot(\taub^n + \taub^{n+1,m+1})\nonumber\\
&& + \frac{1}{2}(\rho^n + \rho^{n+1,m+1})\gb,\label{eq:Stokes}
\end{eqnarray}
\begin{equation}
\nabla\cdot\vb^{n+1,m+1} = \delta S^{n+1,m+1} + \delta S_{\rm corr} - \delta\alpha^{n+1,m+1}\left(\frac{\bar{S}^{n+1,m+1} + \bar{S}_{\rm corr}}{\bar{\alpha}^{n+1,m+1}}\right).
\end{equation}
If we express the viscous stress tensor in operator form, e.g., 
$\mathcal{A}_0^n\vb = \nabla\cdot[\eta^n(\nabla\vb + (\nabla\vb)^T)]$, 
then (\ref{eq:Stokes}) can be 
expressed in a familiar Stokes operator form,
\begin{eqnarray}
\left(\frac{\rho^{n+1,m+1}}{\Delta t} - \half\mathcal{A}_0^{n+1,m+1}\right)\vb^{n+1,m+1} + \nabla\pi^{n+1,m+1} &=& \frac{\rho^n\vb^n}{\Delta t} - \half\nabla\cdot\left[(\rho\vb\vb)^n + (\rho\vb\vb)^{n+1,m}\right]\nonumber\\
&&\hspace{-0.25in} + \half\mathcal{A}_0^n\vb^n + \half(\rho^n + \rho^{n+1,m+1})\gb.
\end{eqnarray}
\MarginPar{These notes about solving in homogeneous delta form don't really have to make it into the paper.
They are more to help me with implementation.
We could mention that we do it, and just cite the binary paper.}
{\color{red}
In practice, we solve for a modified increment to the velocity field, $\deltab\vb$.  In particular,
we split the updated velocity into
$\vb^{n+1,m+1} = \bar{\vb}^n + \deltab\vb$, where $\bar{\vb}^n$ is the velocity at $t^n$, but with modified boundary
conditions consistent with $\vb^{n+1}$, 
and $\deltab\vb$ has homogeneous boundary conditions of the same type 
(Dirichlet or Neumann).
We also split $\pi^{n+1,m+1} = \pi^n + \delta\pi$ in the same way.
Rewriting the Stokes system above we have:
\begin{eqnarray}
\left(\frac{\rho^{n+1,m+1}}{\Delta t} - \half\mathcal{A}_0^{n+1,m+1}\right)\deltab\vb + \nabla\delta\pi &=& \frac{\rho^n\vb^n - \rho^{n+1,m+1}\bar{\vb}^n}{\Delta t} - \nabla\pi^n\nonumber\\
&&\hspace{-1.5in} - \half\left[\nabla\cdot(\rho\vb\vb)^n + \nabla\cdot(\rho\vb\vb)^{n+1,m}\right] + \half(\mathcal{A}_0^n\vb^n+\mathcal{A}_0^{n+1,m+1}\bar{\vb}^n) + \half(\rho^n + \rho^{n+1,m+1})\gb,\nonumber\\
\end{eqnarray}
\begin{equation}
-\nabla\cdot\delta\vb = \nabla\cdot\bar{\vb}^n - \left[\delta S^{n+1,m+1} + \delta S_{\rm corr} - \delta\alpha^{n+1,m+1}\left(\frac{\bar{S}^{n+1,m+1} + \bar{S}_{\rm corr}}{\bar{\alpha}^{n+1,m+1}}\right)\right].
\end{equation}
}% end red color block
{\bf Step 8}\\ \\
Update volume discrepancy
\begin{equation}
S_{\rm corr} = S_{\rm corr} + \frac{2}{(\rho P_\rho)^{n+1,m+1}}\left(\frac{P(\rho,\wb,T)^{n+1,m+1} - P_0^{n+1,m+1}}{\Delta t}\right)
\end{equation}
Note: I've found this factor of two to be more effective at reducing the drift
since we are using trapezoidal advective fluxes.  If the drift is unacceptable,
increment $m$ and return to {\bf Step 4}.

\section{Ideal Gas Test Problem}
Our first test will be a mixture of four noble gases, He-Ne-Ar-Kr,
using an ideal gas equation of state and a hard sphere model for transport.
An ideal multi-component gas has an equation of state,
\begin{equation}
P = \rho\mathcal{R}T\sum_k\frac{w_k}{m_k},
\end{equation}
with universal gas constant $\mathcal{R} = 8.31451\times 10^7$ [erg/(K-mol)] 
(all units are CGS) and $m_k$ is the molecular mass of species $k$ in [g/mol].
The enthalpy, mass fractions, and temperature are related by:
\begin{equation}
h = \sum_k w_k h_k(T).
\end{equation}
For an ideal gas, the thermodynamic derivatives in (\ref{eq:thermo deriv})
reduce to,
\begin{eqnarray}
P_\rho = \mathcal{R} T\sum_k\frac{w_k}{m_k}, \quad
P_T = \rho\mathcal{R}\sum_k\frac{w_k}{m_k}, \quad
P_{w_k} = \frac{\rho\mathcal{R}T}{m_k},
\end{eqnarray}
and simply the coefficients in the constraint equation,
\begin{equation}
\beta_k = \frac{1}{\rho}\left(\frac{\bar{m}}{m_k} - \frac{h_m}{c_p T}\right), \quad
\theta = \frac{1}{\rho c_p T},
\end{equation}
where $\bar{m} = (\sum_k w_k/m_k)^{-1}$ is the mixture-averaged (mean) molecular mass.  Since for an ideal gas, $\Gamma_1 = \gamma = c_p/c_v$, we have
$\alpha = 1/(\gamma P)$.

In this hard sphere model, we use $c_p = c_v = \sum_k 3k_b/2m_k$.
The transport coefficients $\eta,\lambda$ and $\chi$ are evaluated using the 
dilute gas formulation in \cite{Hirschfelder54}, with the latter found numerically
using an efficient iterative method from \cite{Giovangigli99}.
Thermodiffusion coefficients $\zeta$ are evaluated using the formulation in \cite{Valk}.
Refer to Table I in \cite{LLNS} for the molecular diameters used to compute
the transport coefficients.

The problem setup is a three-dimensional enclosed domain (solid impermeable 
walls on each side) of 1~cm$^3$ initially at room temperature and pressure
($T=273$~K, $P=1.01325\times 10^6$~Ba).  The initial mass fraction of each component
is $w_k = 0.25$.  There is an external heating of $\Hext=10^{10}e^{-100r^2}$~erg/(g-s),
where $r$ is the distance to the center of the domain.  We perform a convergence
study using three different resolutions, with spatial and temporal refinement
factors of 2 between simulations.  For the finest simulation, we use a grid
spacing of $\Delta x = 1/128$ and a fixed time step
of (check)~s, which corresponds to a mass diffusion CFL of (check), and
a maximum advective CFL of (check) over the simulation.
We run each simulation to a total time of (check).
We note that the ambient pressure has increased to (check; may need to increase heating
to get some kind of change).  Maybe show plots of final state, particularly temperature.

Show convergence for $\rho$, $h$, $w_i$, $P$, $\vb$.

\section{Soave-Redlich-Kwong Test Problem}

Nonreacting mixture of H$_2$-O$_2$-N$_2$ at high temperature and pressure
using a Soave-Redlich-Kwong equation of state \cite{RedlichKwong,Soave}
for multicomponent flow \cite{GMD11}.  These equations are valid without
additional coefficient truncation for temperatures below the lowest component 
crossing temperature (roughly 1000~K for N$_2$) \cite{GMD11}.
\begin{equation}
P = \frac{\mathcal{R}T}{v-b(\wb)}\sum_k\frac{w_k}{m_k} - \frac{a(T,\wb)}{v(v+b(\wb))},
\end{equation}
with specific volume $v=1/\rho$.  We will denote $T_{c,i}$ and $P_{c,i}$ as the critical
temperature and pressure for component $i$.
\begin{equation}
a(T,\wb) = \sum_i\sum_j w_iw_j\sqrt{a_i(T)a_j(T)}
\end{equation}
\begin{equation}
a_i(T) = 0.42748\frac{R^2T_{c,i}^2}{m_i^2 P_{c,i}}\underbrace{\left[1 + \underbrace{\left(0.48508 + 1.5517\bar\omega_i - 0.151613\bar\omega_i^2\right)}_{s_i}\left(1-\sqrt{\frac{T}{T_{c,i}}}\right)\right]^2}_{\alpha_i(T)},
\end{equation}
with $\bar\omega_i$ the acentric factor for component $i$.
\begin{equation}
b(\wb) = \sum_i w_i b_i, \quad b_i = 0.08664\frac{RT_{c,i}}{m_i P_{c,i}}
\end{equation}

The acentric factors, $\bar\omega$, for H$_2$, O$_2$, and N$_2$, respectively, 
are (-0.220,0.022,0.040.)  Thus, $s_i=$(0.13637,0.51914,0.54691),
the critical temperatures are $T_{c}=$ (33.20,154.6,126.2)~K, 
and the critical pressures are ($1.3\times 10^7,5.05\times 10^7,3.35\times 10^7$~Ba.

\subsection{Thermodynamic Derivatives}
To evaluate $P_\rho$, use the chain rule, $P_\rho = P_v v_\rho = -v^2 P_v$:
\begin{equation}
P_\rho = \frac{v^2}{(v-b(\wb))^2}\mathcal{R}T\sum_k\frac{w_k}{m_k} - \frac{a(T,\wb)v^2(2v+b(\wb))}{[v(v+b(\wb))]^2}
\end{equation}
\begin{equation}
P_T = \frac{\mathcal{R}}{v-b(\wb)}\sum_k\frac{w_k}{m_k} - \frac{\frac{\partial a(T,\wb)}{\partial T}}{v(v+b(\wb))}
\end{equation}
with
\begin{equation}
\frac{\partial a(T,\wb)}{\partial T} = \half \sum_i\sum_j w_i w_j\left(\sqrt{\frac{a_i(T)}{a_j(T)}}\frac{\partial a_i(T)}{\partial T} + \sqrt{\frac{a_j(T)}{a_i(T)}}\frac{\partial a_j(T)}{\partial T}\right)
\end{equation}
\begin{equation}
\frac{\partial a_i(T)}{\partial T} = -0.42748\frac{R^2T_{c,i}^2s_i}{m_i^2 P_{c,i}}\sqrt{\frac{\alpha_i(T)}{T_{c,i}T}}
\end{equation}
\begin{equation}
P_{w_k} = \left(\frac{\mathcal{R}T}{v-b(\wb)}\right)\frac{1}{m_k}
+\frac{\mathcal{R}Tb_k\sum_i\frac{w_i}{m_i}}{[v-b(\wb)]^2}
- \frac{v(v+b(\wb))\frac{\partial a(T,\wb)}{\partial w_k} - a(T,\wb)v\frac{\partial b(\wb)}{\partial w_k}}{[v(v+b(\wb))]^2},
\end{equation}
with
\begin{equation}
\frac{\partial a(T,\wb)}{\partial w_k} = 2\sum_i w_i\sqrt{a_i(T)a_k(T)},
\quad
\frac{\partial b(\wb)}{\partial w_k} = b_k.
\end{equation}

\section{Temperature-Dependent Density Example}
Consider an equation of state where the pure component densities are a function
of temperature,
\begin{equation}
\sum_i\frac{\rho_i}{\bar\rho_i(T)} = 1.
\end{equation}
Then,
\begin{equation}
\rho = \left(\sum_i\frac{w_i}{\bar\rho_i(T)}\right)^{-1},
\end{equation}
\begin{equation}
\rho_{w_k} = -\frac{1}{\left(\sum_i\frac{w_i}{\bar\rho_i(T)}\right)^2}\frac{1}{\bar\rho_k}
= -\frac{\rho^2}{\bar\rho_k},
\end{equation}
\begin{equation}
\rho_T = \frac{1}{\left(\sum_i\frac{w_i}{\bar\rho_i(T)}\right)^2}\sum_i\left(\frac{w_i}{\rho_i(T)^2}\frac{\partial\rho_i(T)}{\partial T}\right)
\end{equation}


\section{Implementation Notes}
The molecular weights, diameters, and assumption of hard sphere dynamics
 are all that is needed to specify the transport in
{\tt energy\_EOS\_module\_HS.f90}.  The functions of interest are
(not all of these are actually being used at the moment):\\
\begin{itemize}

\item {\tt CKCPBS} computes $c_p(\wb,T)$.  In this EOS this is not dependent on temperature.\\

\item {\tt CKCVBS} computes $c_v(\wb,T)$.  In this EOS this is not dependent on temperature.\\

\item {\tt CKUBMS} computes the mean internal energy, $e(\wb,T)$.\\

\item {\tt get\_t\_given\_ey} computes $T(e,\wb)$.\\

\item {\tt CKPY} computes $P(\rho,\wb,T)$.\\

\item {\tt CKHMS} computes $h_m(T)$.  Note $h = \sum_k w_k h_k$.\\

\item {\tt CKYTX} computes mole fractions, $\xb$, from mass fractions, $\wb$.\\

\item {\tt CKYTCR} computes molar concentrations from $\rho,T$ and $\wb$.
In this EOS this is not dependent on temperature.\\

\item {\tt CKXTY} mass fractions, $\wb$, from mole fractions, $\xb$.\\

\item {\tt CKMMWY} computes the mixture-averaged molecular mass, $\bar{m}$, given $\wb$.\\

\item {\tt CKRHOY} computes $\rho(P,\wb,T)$.\\

\item {\tt CKCPMS} computes the individual specific heats for each component,
$c_p(T)$.  In this EOS this is not dependent on temperature.\\

\end{itemize}

The subroutine {\tt ideal\_mixture\_transport} takes $\rho,T,P,\wb$, and $\xb$ 
as inputs and computes the following:\\
\begin{itemize}

\item {\tt eta} is the dynamic viscosity, $\eta$.\\

\item {\tt zeta} is the bulk viscosity, $\kappa$.\\

\item {\tt diff\_ij} is the diffusion matrix, $\chi$.\\

\item {\tt kappa} is the thermal conductivity, $\lambda$.\\

\item {\tt chitil} are the thermodiffusion coefficients, $\zeta$.
                   (I modified the routine to make this so).\\

\end{itemize}

\section{Conclusions}

Our staggered grid formulation allows for the including of thermal fluctuations
in a way that satisfies discrete fluctuation-dissipation balance through an
exact projection as well as matching stochastic and diffusive operators.
The inclusion of thermal fluctuations is straightforward, following in our previous work.
We defer the inclusion of fluctuations to a future paper, as
the methodology here is applicable to low Mach number models regardless of framework.

\bibliographystyle{unsrt}
\bibliography{DesignDocument}

\end{document}
