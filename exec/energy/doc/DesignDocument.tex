\documentclass[final]{siamltex}

% for red MarginPars
\usepackage{color}

% for \boldsymbol
\usepackage{amsmath}

% for \mathfrak
\usepackage{amsfonts}

% total number of floats allowed on a page
\setcounter{totalnumber}{100}

% float page fractions
\renewcommand{\topfraction}{0.9}
\renewcommand{\bottomfraction}{0.9}
\renewcommand{\textfraction}{0.2}

% MarginPar
\setlength{\marginparwidth}{0.75in}
\newcommand{\MarginPar}[1]{\marginpar{\vskip-\baselineskip\raggedright\tiny\sffamily\hrule\smallskip{\color{red}#1}\par\smallskip\hrule}}

% for non-stacked fractions
\newcommand{\sfrac}[2]{\mathchoice
  {\kern0em\raise.5ex\hbox{\the\scriptfont0 #1}\kern-.15em/
   \kern-.15em\lower.25ex\hbox{\the\scriptfont0 #2}}
  {\kern0em\raise.5ex\hbox{\the\scriptfont0 #1}\kern-.15em/
   \kern-.15em\lower.25ex\hbox{\the\scriptfont0 #2}}
  {\kern0em\raise.5ex\hbox{\the\scriptscriptfont0 #1}\kern-.2em/
   \kern-.15em\lower.25ex\hbox{\the\scriptscriptfont0 #2}}
  {#1\!/#2}}

\def\1b {{\bf 1}}
\def\bb {{\bf b}}
\def\Fb {{\bf F}}
\def\gb {{\bf g}}
\def\mb {{\bf m}}
\def\Qb {{\bf Q}}
\def\vb {{\bf v}}
\def\wb {{\bf w}}
\def\Wb {{\bf W}}
\def\xb {{\bf x}}

\def\chib   {\boldsymbol{\chi}}
\def\deltab {\boldsymbol{\delta}}
\def\Gammab {\boldsymbol{\Gamma}}
\def\phib   {\boldsymbol{\phi}}
\def\Phib   {\boldsymbol{\Phi}}
\def\Psib   {\boldsymbol{\Psi}}
\def\Sigmab {\boldsymbol{\Sigma}}
\def\taub   {\boldsymbol{\tau}}
\def\zetab  {\boldsymbol{\zeta}}

\def\half   {\frac{1}{2}}
\def\myhalf {\sfrac{1}{2}}

\begin{document}

%==========================================================================
% Title
%==========================================================================
\title{Notes for Adding Energy Equation to the Implicit Low Mach Number Multispecies Code}

\maketitle

\section{Equations}
The governing conservation equations for momentum, mass, and energy for our system are:
\begin{eqnarray}
\frac{\partial\rho\vb}{\partial t} &=& - \nabla\cdot(\rho\vb\vb) - \nabla\pi + \nabla\cdot\taub + \rho\gb,\label{eq:momentum}\\
\frac{\partial\rho_i}{\partial t} &=& -\nabla\cdot(\rho_i\vb) + \nabla\cdot F_i,\label{eq:mass}\\
\frac{\partial\rho h}{\partial t} &=& -\nabla\cdot(\rho h\vb) + \frac{DP}{Dt} + \nabla\cdot\Qb,\label{eq:enthalpy}
\end{eqnarray}
where
$\taub = \overline\taub + \tilde\taub$,
$\Fb = \overline\Fb + \tilde\Fb$, and 
$\Qb = \overline\Qb + \tilde\Qb$ are the stress tensors, mass fluxes, and heat fluxes 
respectively, divided into deterministic and stochastic components.
These equations are closed by the following relation derived from low Mach number asymptotics,
which states that the thermodynamic pressure must remain spatially constant,
\begin{equation}
P_0 = P(\rho,\wb,T),\label{eq:EOS}
\end{equation}
with
\begin{equation}
T = T(\rho,\wb,h)
\end{equation}
The deterministic stress tensor ignores the effect of bulk viscosity, 
$\overline\taub = \eta\bar\nabla\vb = \eta[\nabla\vb + (\nabla\vb)^T]$.
The deterministic heat fluxes have the form 
$\overline\Qb = \lambda\nabla T + \sum_k\nabla\cdot h_k F_k$.
The deterministic part of the mass fluxes, $\overline{\Fb}$, contains contributions from 
compositional gradients, barodiffusion, and temperature gradients,
\begin{equation}
\overline{F} = -\rho\Wb\chib\left[\Gammab\nabla\xb + (\phib - \wb)\frac{\nabla P}{n k_B T} + \zetab\frac{\nabla T}{T}\right].
\end{equation}

\subsection{Velocity Constraint and Thermodynamic Pressure Update}
To derive a constraint on the velocity, we begin by differentiating equation (\ref{eq:EOS})
along particle paths,
\begin{equation}
\frac{DP_0}{Dt} = P_\rho\frac{D\rho}{Dt} + P_T\frac{DT}{Dt} + \sum_kP_{w_k}\frac{Dw_k}{Dt}.\label{eq:particle paths}
\end{equation}
Since $P_0$ is constant in space but varies in time, we make the substitution
$DP/Dt = \partial P_0(t)/\partial t$ in both the enthalpy equation 
(\ref{eq:enthalpy}) and the constraint equation (\ref{eq:particle paths}).
This will allow for pressure changes in closed containers, but not allow for
pressure stratification due to a gravitational density stratification.

Next, by noting that by continuity $D\rho/Dt = -\rho\nabla\cdot\vb$, we can rewrite 
(\ref{eq:particle paths}) as
\begin{equation}
\nabla\cdot\vb = \frac{1}{\rho P_\rho}\left(-\frac{\partial P_o}{\partial t} + P_T\frac{DT}{Dt} + \sum_kP_{w_k}\frac{Dw_k}{Dt}\right).\label{eq:constraint1}
\end{equation}
To obtain an expression for $DT/Dt$, we apply differentiate the enthalpy, $h=h(P,T,w_k)$, 
along particle paths to obtain
\begin{equation}
\frac{DT}{Dt} = \frac{1}{c_p}\left(\frac{Dh}{Dt} - h_P\frac{\partial P_0}{\partial t} - \sum_kh_{w_k}\frac{Dw_k}{Dt}\right).\label{eq:DTDt1}
\end{equation}
Note that we have also replaced $DP/Dt$ with 
$\partial P_0/\partial t$ in this equation.  From equation (\ref{eq:mass}) and 
continuity we have,
\begin{equation}
\frac{Dw_k}{Dt} = \frac{1}{\rho}\nabla\cdot\Fb_k,\label{eq:DwDt}
\end{equation}
and from equation (\ref{eq:enthalpy}) we have,
\begin{equation}
\frac{Dh}{Dt} = \frac{1}{\rho}\left(\frac{\partial P_0}{\partial t} + \nabla\cdot\Qb\right).\label{eq:DhDt}
\end{equation}
Combining (\ref{eq:DTDt1}), (\ref{eq:DwDt}), and (\ref{eq:DhDt}) we have
\begin{equation}
\frac{DT}{Dt} = \frac{1}{\rho c_p}\left[\left(1 - \rho h_P\right)\frac{\partial P_0}{\partial t} + \nabla\cdot\Qb - \sum_kh_{w_k}\nabla\cdot\Fb_k\right].\label{eq:DTDt}
\end{equation}
Combining equations (\ref{eq:constraint1}) and (\ref{eq:DTDt}) gives
\footnote{These are the same equations in MAESTRO Paper III, Almgren et al., ApJ, 2008, except that paper includes a convective derivative of $P_0$ due to stratification}
\begin{equation}
\nabla\cdot\vb + \alpha\frac{\partial P_0}{\partial t} = \sum_k \beta_k \nabla\cdot\Fb_k + \gamma \nabla\cdot\Qb \equiv S,\label{eq:constraint}
\end{equation}
\begin{equation}
\alpha = -\left[\frac{(1-\rho h_p)P_T - \rho c_p}{\rho^2 c_p P_\rho}\right], \qquad
\beta_k = \frac{P_{w_k}}{\rho^2 P_\rho} - \frac{P_T h_{w_k}}{\rho^2 P_\rho c_p}, \qquad
\gamma = \frac{P_T}{\rho^2 P_\rho c_P}.
\end{equation}
The unknowns in equation (\ref{eq:constraint}) are $\vb$ and $\partial P_0/\partial t$.
For open containers where $\vb$ is allowed to be non-zero on one or more wallsm $P_0$
is constant and the constraint equation reduces to
\begin{equation}
\nabla\cdot\vb = S.
\end{equation}
For closed chambers with no-flow walls, this system only has a solution if
\begin{equation}
\int_{\Omega} \left(S - \alpha\frac{\partial P_0}{\partial t}\right)\partial\Omega = 0.
\end{equation}
We can use this fact to simultaneously solve for the unknowns
$\vb$ and $\partial P_0/\partial t$.
To do this, we split up $S$ into an average and a perturbational component,
\begin{equation}
S = \bar{S} + \delta S,
\end{equation}
as well as split up $\alpha$ into an average and a perturbational component,
\begin{equation}
\alpha = \bar{\alpha} + \delta\alpha.
\end{equation}
By definition,
\begin{equation}
\int_{\Omega} \delta S ~\partial\Omega = \int_{\Omega} \delta \alpha ~\partial\Omega = 0.\label{eq:zero int}
\end{equation}
So equation (\ref{eq:constraint}) can be rewritten as,
\begin{equation}
\nabla\cdot\vb = \bar{S} + \delta S - \bar{\alpha}\frac{\partial P_0}{\partial t} - \delta\alpha\frac{\partial P_0}{\partial t}.\label{eq:constraint2}
\end{equation}
Recall that the right-hand-side of equation (\ref{eq:constraint2}) must integrate to zero.
The second and fourth terms integrate to zero by equation (\ref{eq:zero int}) and the fact that
$\partial P_0/\partial t$ is a constant.
The first and third terms are non-zero constants,
so in order for the entire RHS to integrate to zero we must have
\begin{equation}
\bar{S} - \bar{\alpha}\frac{\partial P_0}{\partial t} = 0.\label{eq:P0}
\end{equation}
Thus, equation (\ref{eq:P0}) is the evolution equation for $P_0$ and the constraint equation reduces to
\begin{equation}
\nabla\cdot\vb = \delta S - \delta\alpha\frac{\partial P_0}{\partial t}.\label{eq:constraint3}
\end{equation}
Altogether, the system we are solving consists of equations (\ref{eq:momentum}),  (\ref{eq:mass}), and (\ref{eq:enthalpy})
subject to (\ref{eq:P0}), (\ref{eq:constraint3}).

\section{Equation of State Drift}

\section{Heat Flux Formulation}
Here's the basic idea.  Recall (\ref{eq:enthalpy}):
\begin{equation}
\frac{\partial(\rho h)}{\partial t} = -\nabla\cdot(\rho h\vb) + \frac{DP}{Dt} + \nabla\cdot\lambda\nabla T + \sum_k\nabla\cdot h_kFb_k.
\end{equation}
We want to handle the thermal diffusion semi-implicitly.  We use an iterative method that
linearizes the relationship between how $h$ changes with respect to changes in $T$.\\ \\
Let's assume we want to form an iterative update to $h$ using
\begin{eqnarray}
\frac{\rho^{n+1}h^{n+1,l+1} - (\rho h)^n}{\Delta t} &=& -\nabla\cdot(\rho h\vb)^{n+\myhalf}\nonumber\\
&&+ \half\nabla\cdot\lambda^n\nabla T^n + \half\nabla\cdot\lambda^{n+1,l}T^{n+1,l+1}\nonumber\\
&& + \half\sum_k\nabla\cdot h_k^n\Fb_k^n + \half\sum_k\nabla\cdot h_k^{n+1,l}\Fb^{n+1}.\label{eq:temp form}
\end{eqnarray}
\\
We start with initial guesses $(T,h)^{n+1,0} = (T,h)^n$.\\ \\
We say that
\begin{equation}
T^{n+1,l+1} = T^{n+1,l} + \delta T\label{eq:delta T}
\end{equation}
Linearization gives
\begin{equation}
h^{n+1,l+1} = h^{n+1,l} + c_p^{n+1,l}\delta T\label{eq:delta h}
\end{equation}
\\
Substituting (\ref{eq:delta T}) and (\ref{eq:delta h}) into (\ref{eq:temp form}) gives:
\begin{eqnarray}
\frac{\rho^{n+1}(h^{n+1,l} + c_p^{n+1,l}\delta T) - (\rho h)^n}{\Delta t} &=& -\nabla\cdot(\vb\rho h)^{n+\myhalf}\nonumber\\
&&+ \half\nabla\cdot\lambda^n\nabla T^n + \half\nabla\cdot\lambda^{n+1,l}(T^{n+1,l} + \delta T)\nonumber\\
&& + \half\sum_k\nabla\cdot h_k^n\Fb_k^n + \half\sum_k\nabla\cdot h_k^{n+1,l}\Fb_k^{n+1}.
\end{eqnarray}
\\
We solve this implicitly for $\delta T$, then compute $(T,h)^{n+1,l+1}$
using (\ref{eq:delta T}) and (\ref{eq:delta h}).\\

FIXME: One consequence of using this formulation is that we can use a slightly different, 
analytically equivalent formulation for the constraint:
\begin{eqnarray}
\nabla\cdot\vb &=& \frac{1}{\rho c_p T}\left(\nabla\cdot\lambda\nabla T + \sum_k\nabla\cdot h_k\Fb_k\right)\nonumber\\
&& + \frac{1}{\rho}\sum_k\left(\frac{\overline{W}}{W_k} - \frac{h_k}{c_p T}\right)\left(\nabla\cdot\Fb_k + \dot\omega_k\right) \equiv S.
\end{eqnarray}

\section{Algorithm Summary}
An inertial algorithm might look like this.\\ \\
Initialization:\\
\begin{enumerate}
\item Compute $\Fb$ and $\Qb$ at the initial time and project.
\item Loop over the following:
\begin{enumerate}
\item Predictor forward Euler step for thermodynamic variables.
\item Advance $P_0$.
\item If satisfied with thermodynamic drift, exit this loop.
\item Determine the volume discrepancy correction and redefine $S$.
\item Project to obtain an updated velocity field.
\end{enumerate}
\item Proceed to Corrector.\\
\end{enumerate}
Corrector:\\
\begin{enumerate}
\item Compute preliminary time-advanced $\Fb$ and $\Qb$.
\item Loop over the following
\begin{enumerate}
\item GMRES solve for preliminary time-advanced velocity.
\item Trapezoidal scalar corrector step for thermodynamic variables.
\item Advance $P_0$.
\item If satisfied with thermodynamic drift, exit this loop.
\item Determine the volume discrepancy correction and redefine $S$.  The weighting of the
correction may have to change since this velocity field is used in a trapezoidal corrector.
I'm also not sure if/how to relate this correction to the predictor correction.
\end{enumerate}
\item Proceed to New-Time Predictor.\\
\end{enumerate}
New-Time Predictor:\\
\begin{enumerate}
\item Compute updated time-advanced $\Fb$ and $\Qb$.
\item Loop over the following:
\begin{enumerate}
\item GMRES solve for updated time-advanced velocity.
\item Predictor forward Euler step for thermodynamic variables.
\item Advance $P_0$.
\item If satisfied with thermodynamic drift, exit this loop.
\item Determine the volume discrepancy correction and redefine $S$.
\end{enumerate}
\item Return to Corrector.
\end{enumerate}

\subsection{Volume Discrepancy}
No matter what kind of temporal integrator you are using, any time you compute a
complete set of time-advance thermodynamic variables (whether they be preliminary
or final), they will not be consistent with the ambient pressure.  The strategy
is to wrap the entire time-advancement routine in an iterative loop that uses
volume discrepancy corrections to drive the time-advanced state onto the
equation of state.  We assume isentropic expansion/contraction of fluid elements
and relate this back to the pressure we wish to obtain.

We add a correction term that looks something like
\begin{equation}
\nabla\cdot\vb = S + \frac{1}{\rho P_\rho}\left(\frac{P_{\rm eos}^{\rm new} - P_0^{\rm new}}{\Delta t} + \vb\cdot\nabla P\right) \equiv S + S_{\rm corr}.
\end{equation}
Normally we omit the convective part, as it doesn't seem to have much effect with or
without it.  Since $S_{\rm corr}$ doesn't necessarily sum to zero, so in the 
corrector solve we have to split the update again,
\begin{equation}
\nabla\cdot\vb = \delta S + \delta S_{\rm corr},
\end{equation}
\begin{equation}
\alpha\frac{\partial P_0}{\partial t} = \bar{S} + \bar{S}_{\rm corr}.
\end{equation}
There needs to be some kind of iteration to make sure $P_{\rm eos}$ is sufficiently
close to $P_0^{\rm new}$.

\section{Algorithm Details}
Assume we are given initial $\rho_i$ and $h$ that are thermodynamically consistent with 
$P_0$.  If they are not, compute e.g., $h = h(\rho,P_0,\wb)$.  Initialize the velocity
to zero, although this shouldn't matter beyond solver tolerance as it will get 
overwritten during the initialization step.\\ \\
{\bf Initialization}.\\ \\
This step is only performed once at the beginning of the simulation, so here $n=0$.
The goal of this step is to compute preliminary time-advanced thermodynamically
consistent pressure and scalars, $(P_0,\rho_i,(\rho h))^{*,n+1}$.  This will
require the computation of a velocity field, $\vb^n$, that preserves this
thermodynamic balance.  
We set the initial volume discrepancy correction to zero, 
$S_{\rm corr}^n = 0$, as well as the individual terms in the decomposition,
$S_{\rm corr}^n = \bar{S}_{\rm corr}^n + \delta S_{\rm corr}^n$.
Begin by computing fluxes, $\Fb^n, \Qb^n$ and then
loop over the following until the convergence criteria in {\bf Step 0d} is met:\\
\begin{itemize}
\item {\bf Step 0a:} Compute a pressure update:
\begin{equation}
P_0^{*,n+1} = P_0^n + \frac{\Delta t(\bar{S}^n + \bar{S}_{\rm corr}^n)}{\alpha^n}
\end{equation}
\item {\bf Step 0b:} Compute the velocity field using
\begin{equation}
\nabla\cdot\vb^n = \delta S^n + \delta S_{\rm corr}^n
\end{equation}
\item {\bf Step 0c:} Advect-diffuse the thermodynamic variables using forward-Euler advective fluxes,
explicit mass diffusion, and implicit energy diffusion:
\begin{eqnarray}
\rho_i^{*,n+1} &=& \rho_i^n + \Delta t\left[-\nabla\cdot(\rho_i\vb)^n + \Fb^n\right], \\
(\rho h)^{*,n+1} &=& (\rho h)^n + \Delta t\left[-\nabla\cdot(\rho h\vb)^n + \frac{\bar{S}^n + \bar{S}_{\rm corr}^n}{\alpha^n} + \frac{1}{2}(\Qb^n + \Qb^{*,n+1})\right].\nonumber\\
\end{eqnarray}
\item {\bf Step 0d:} If the thermodynamic drift is unacceptable, update the volume 
discrepancy correction,
\begin{equation}
S_{\rm corr}^n = S_{\rm corr}^n + \frac{1}{\rho P_\rho}\left(\frac{P_{\rm EOS}^{*,n+1} - P_0^{*,n+1}}{\Delta t}\right)
\end{equation}
and return to {\bf Step 0a}.  Otherwise, proceed to the {\bf Scalar Corrector Step}.\\
\end{itemize}
{\bf Scalar Corrector Step}.\\ \\
The goal of this step is to compute updated time-advanced thermodynamically consistent 
pressure and scalars, $(P_0,\rho_i,(\rho h))^{n+1}$.  This will require the computation
of a velocity field, $\vb^{*,n+1}$, that preserves this thermodynamic balance.
We set the volume discrepancy correction to zero,
$S_{\rm corr}^{*,n+1} = 0$, as well as the individual terms in the decomposition
$S_{\rm corr}^{*,n+1} = \bar{S}_{\rm corr}^{*,n+1} + \delta S_{\rm corr}^{*,n+1}$
Begin by computing preliminary time-advanced fluxes, $\Fb^{*,n+1}, \Qb^{*,n+1}$,
and then loop over the following until the convergence criteria in {\bf Step 1d} is met:\\
\begin{itemize}
\item {\bf Step 1a:} Pressure update.
\begin{equation}
P_0^{n+1} = P_0^n + \frac{\Delta t}{2}\frac{(\bar{S}^n + \bar{S}_{\rm corr}^n)}{\alpha^n} + \frac{\Delta t}{2}\frac{(\bar{S}^{*,n+1} + \bar{S}_{\rm corr}^{*,n+1})}{\alpha^{*,n+1}}
\end{equation}
\item {\bf Step 1b:} GMRES solve for the velocity, $\vb^{*,n+1}$, and perturbational
pressure, $\pi^{*,n+1}$.
\begin{equation}
\frac{\rho^{*,n+1}\vb^{*,n+1} - \rho^n\vb^n}{\Delta t} + \nabla\pi^{*,n+1} = \nabla\cdot(-\rho\vb\vb)^n + \frac{1}{2}\nabla\cdot(\taub^n + \taub^{*,n+1}) + \frac{1}{2}(\rho^n + \rho^{*,n+1})\gb,
\end{equation}
\begin{equation}
\nabla\cdot\vb^{*,n+1} = \delta S^{*,n+1} + \delta S_{\rm corr}^{*,n+1}
\end{equation}
\item {\bf Step 1c:} Advect-diffuse the thermodynamic variables using trapezoidal 
corrector advective fluxes, trapezoidal corrector mass diffusion, and implicit energy 
diffusion:
\begin{eqnarray}
\rho_i^{n+1} &=& \rho_i^n + \frac{\Delta t}{2}\left[-\nabla\cdot(\rho_i\vb)^n -\nabla\cdot(\rho_i\vb)^{*,n+1} + \Fb^n + \Fb^{*,n+1}\right],\nonumber \\
\\
(\rho h)^{n+1} &=& (\rho h)^n + \frac{\Delta t}{2}\left[-\nabla\cdot(\rho h\vb)^n -\nabla\cdot(\rho h\vb)^{*,n+1} + \frac{\bar{S}^n + \bar{S}_{\rm corr}^n}{\alpha^n} + \frac{\bar{S}^{*,n+1}+ \bar{S}_{\rm corr}^{*,n+1}}{\alpha^{*,n+1}} + \Qb^n + \Qb^{n+1}\right].\nonumber\\
\end{eqnarray}
\item {\bf Step 1d:} If the thermodynamic drift is unacceptable, update the volume 
discrepancy correction,
\begin{equation}
S_{\rm corr}^{*,n+1} = S_{\rm corr}^{*,n+1} + \frac{2}{\rho P_\rho}\left(\frac{P_{\rm EOS}^{n+1} - P_0^{n+1}}{\Delta t}\right)
\end{equation}
and return to {\bf Step 1a}.  I think the factor of 2 is required here since the new-time
velocity is used in a trapezoidal corrector with only a half-weighting.  Otherwise, 
proceed to the {\bf Scalar Predictor Step}.\\
\end{itemize}
{\bf Scalar Predictor Step.}\\ \\
We are now in the next time step, so we increment $n$.
The goal of this step is to compute preliminary time-advanced thermodynamically
consistent pressure and scalars, $(P_0,\rho_i,(\rho h))^{*,n+1}$.  This will
require the computation of a velocity field, $\vb^n$, that preserves this
thermodynamic balance.  Note that this step is essentially the same as
{\bf Step 0}, except that in the computation of $\vb^n$ we use a GMRES
solver rather than a projection, since in essence we are correcting the velocity
field to second order by including a trapezoidal discretization of the advective fluxes.
We set the initial volume discrepancy correction to zero, 
$S_{\rm corr}^n = 0$, as well as the individual terms in the decomposition,
$S_{\rm corr}^n = \bar{S}_{\rm corr}^n + \delta S_{\rm corr}^n$.
Begin by computing fluxes, $\Fb^n, \Qb^n$ and then loop over the following until 
the convergence criteria in {\bf Step 2d} is met:\\
\begin{itemize}
\item {\bf Step 2a:} Compute a pressure update:
\begin{equation}
P_0^{*,n+1} = P_0^n + \frac{\Delta t(\bar{S}^n + \bar{S}_{\rm corr}^n)}{\alpha^n}
\end{equation}
\item {\bf Step 2b:} GMRES solve for the velocity, $\vb^n$, and perturbational
pressure, $\pi^n$.
\begin{equation}
\frac{\rho^n\vb^n - \rho^{n-1}\vb^{n-1}}{\Delta t} + \nabla\pi^n = \half\nabla\cdot(-(\rho\vb\vb)^{n-1} - (\rho\vb\vb)^{*,n}) + \frac{1}{2}\nabla\cdot(\taub^{n-1} + \taub^n) + \frac{1}{2}(\rho^{n-1} + \rho^n)\gb,
\end{equation}
\begin{equation}
\nabla\cdot\vb^n = \delta S^{*,n} + \delta S_{\rm corr}^n
\end{equation}
\item {\bf Step 2c:} Advect-diffuse the thermodynamic variables using forward-Euler advective fluxes,
explicit mass diffusion, and implicit energy diffusion:
\begin{eqnarray}
\rho_i^{*,n+1} &=& \rho_i^n + \Delta t\left[-\nabla\cdot(\rho_i\vb)^n + \Fb^n\right], \\
(\rho h)^{*,n+1} &=& (\rho h)^n + \Delta t\left[-\nabla\cdot(\rho h\vb)^n + \frac{\bar{S}^n + \bar{S}_{\rm corr}^n}{\alpha^n} + \frac{1}{2}(\Qb^n + \Qb^{*,n+1})\right].\nonumber\\
\end{eqnarray}
\item {\bf Step 2d:} If the thermodynamic drift is unacceptable, update the volume 
discrepancy correction,
\begin{equation}
S_{\rm corr}^n = S_{\rm corr}^n + \frac{1}{\rho P_\rho}\left(\frac{P_{\rm EOS}^{*,n+1} - P_0^{*,n+1}}{\Delta t}\right)
\end{equation}
and return to {\bf Step 2a}.  Otherwise, proceed to the {\bf Scalar Corrector Step}.\\
\end{itemize}
\end{document}
